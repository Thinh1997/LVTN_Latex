\chapmoi{HỆ THỐNG CUNG CẤP ĐIỆN}
\section{TỔNG QUAN}
\textbf{Nhiệm vụ thết kế cung cấp điện:}

Mục tiêu chính của thiết kế cung cấp điện là đảm bảo cho tòa nhà văn phòng tiêu thụ luôn luôn đủ điện năng với chất lượng nằm trong phạm vi cho phép. Phải thỏa mãn những yêu cầu sau:
\begin{itemize}
	\item Vốn đầu tư nhỏ, chú ý đến tiết kiệm được ngoại tệ quý và vật tư hiếm.
	\item Đảm bảo độ tin cậy cung cấp điện cao tùy theo tính chất tiêu thụ.
	\item Chi phí vận hành hàng năm thấp.
	\item Đảm bảo an toàn cho người và thiết bị.
	\item Đảm bảo tính kinh tế.
	\item Thuận tiện cho vận hành, sửa chữa v.v…
\end{itemize}

Ngoài ra, khi thiết kế cung cấp điện phải chú ý đến những yêu cầu khác như:
\begin{itemize}
	\item Dự báo được khả năng phát triển phụ tải sau này.
	\item Rút ngắn thời gian xây xựng.
\end{itemize}

Ngày nay, điện năng được sử dụng rất rộng rãi trong các ngành như: điện tử, giao thông vận tải.v.v.. Do đó mà vai trò của điện đối với đời sống xã hội, điện năng được xem là chỉ tiêu, là thước đo về sự phát triển của một quốc gia.

\section{TÍNH TOÁN PHỤ TẢI ĐIỆN CHO TÒA NHÀ}
\subsection{TÍNH TOÁN PHỤ TẢI CHIẾU SÁNG}
\subsubsection{Đèn hiệu:}
Yêu cầu: Lựa chọn, thiết kế và cung cấp các vật phản quang, các phụ kiện và thiết bị kiểm soát theo sự giới thiệu của nhà sản xuất và cho phép các loại đèn đạt chất lượng thực hiện theo tài liệu kỹ thuật được xuất bản của nhà sản xuất.
\subsubsection{Chiếu sáng nhân tạo:}
Hệ thống chiếu sáng trong nhà được tính toán đủ ánh sáng khi không có chiếu sáng tự nhiên mà vẫn đảm bảo mọi hoạt động bình thường của con người trong công trình. Trong trường hợp khẩn cấp xảy ra vẫn có hệ thống chiếu sáng sự cố được bố trí dọc đường thoát nạn các loại đèn này có nguồn acquy riêng duy trì tối thiểu 2h.

Chiếu sáng ngoài nhà: Hệ thống chiếu sáng ngoài toà công trình được thiết kế với độ sáng 100 Lux. Toàn bộ dùng loại đèn pha có chao chụp phản quan bóng Sodiumnua 250$ W $ loại sử dụng tranpormer
\begin{itemize}
	\item Văn phòng độ rọi từ 300-400 Lux.
	\item Sảnh độ rọi từ 150-200 Lux.
	\item Khu hành lang, cầu thang, kho, vệ sinh độ rọi tối thiểu 100 Lux và được trang bị đèn khẩn cấp để thoát hiểm.
\end{itemize}

Điện áp sử dụng cho đèn là 220v nếu sử dụng đèn có điện áp 380v thì phải có dây nối đất. Với các loại đèn chiếu sáng sự cố, cục bộ, cầm tay ở trong các phòng nguy hiểm hoặc rất nguy hiểm thì điện áp nhỏ hơn 42v và nhỏ hơn 12v đối với các phòng ẩm ướt, chật chội dễ bị chạm vào những bề mặt kim loại lớn có nối đất.
\subsubsection{Thiết kế chiếu sáng:}
\textbf{CÁC CÔNG THỨC SẼ SỬ DỤNG TÍNH TOÁN CHIẾU SÁNG}
\begin{itemize}[label = $\blacktriangleright$]
	\item Chiều cao treo đèn: $H_{tt} = H - H' - H_{lv}$
	\item Phân bố bóng đèn: $K = \dfrac{a\times b}{H_{tt}\times (a+b)}$
	\item Tỷ số treo: $j =\dfrac{H'}{H' + H_{tt}}$
	\item Hệ số chiếu sáng: $U = u_{d}\times \eta_{d} + u_{i}\times \eta_{i}$
	\item Quang thông tổng cộng: {\Large $\Phi$}$_{\Sigma} = \dfrac{E_{tc}\times S\times d}{U}$
	\item Xác định số lượng bóng đèn: $N_{bd} = ${\Large $\dfrac{\Phi_{\Sigma}}{\Phi_{bd}}$}
	\item Kiểm tra độ rọi trung bình trên bề mặt làm việc: $E_{tb} = \dfrac{N_{bd}\times \Phi_{2}\times U}{S\times d}$
\end{itemize}





