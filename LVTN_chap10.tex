\chapmoi{TÍNH TOÁN HỆ THỐNG PHÂN PHỐI NƯỚC LẠNH}
\section{PHƯƠNG PHÁP TÍNH}
Để tính toán đường ống phân phối nước lạnh đến các tầng và đến các dàn lạnh ta làm theo các bước sau:
\subsection{BƯỚC 1}
-- Tính lưu lượng nước lạnh qua từng đoạn ống:
\begin{equation*}
	G = \dfrac{Q_{0}}{C\times\Delta t}
\end{equation*}

Trong đó:
\begin{itemize}
	\item $Q_{0}$: là năng suất lạnh yêu cầu của không gian cần điều hòa.
	\item $\Delta t$:  là độ chênh nhiệt độ của nước vào và ra khỏi dàn lạnh.
	\item $C$: Nhiệt dung riêng của nước ở nhiệt độ trung bình. 
\end{itemize}

\subsection{BƯỚC 2}
Chọn vận tốc sơ bộ: $\omega$ = (0.5$\div$2)m/s

\subsection{BƯỚC 3}
Xác định kích thước đường kính trong của đoạn ống:
\begin{equation*}
	d_{tr} = \sqrt{\dfrac{4\times G}{\pi\times\omega\times\rho}}
\end{equation*}

Trong đó:
\begin{itemize}
	\item $G$: là lưu lượng nước trong đường ống, kg/s
	\item $\omega$: vận tốc nước chuyển động trong ống, m/s
	\item $\rho$: là khối lượng riêng của nước ở nhiệt độ trung bình.
\end{itemize}

\subsection{BƯỚC 4}
Chọn đường kính danh nghĩa và thống số $d_{tr}$, $d_{ng}$ tương ứng.

\subsection{BƯỚC 5}
Tính lại vận tốc thực trong ống theo đường kính trong vừa chọn:
\begin{equation*}
	\omega = \dfrac{4\times G}{\rho\times\pi\times d_{tr}^2}
\end{equation*}

\section{TÍNH TOÁN ỐNG CHÍNH CẤP NƯỚC CHO CÁC TẦNG}

