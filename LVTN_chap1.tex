\fancyhead[L]{\leftmark}
%Trang bìa
\chapmoi{TỔNG QUAN VỀ CÔNG TRÌNH THIẾT KẾ}

%nội dung
	\section{KHÁI QUÁT VỀ CÔNG TRÌNH}
	\subsection{VỊ TRÍ CÔNG TRÌNH}
	- Nằm toạ lạc trên đường Trần Duy Hưng (\emph{địa chỉ cụ thể là: \textbf{117 Trần Duy Hưng, Trung Hoà, Cầu Giấy, Hà Nội}}) và có vị trí chiến lược trong việc phát triển Hà Nội.
	
	 - Nằm gần khu dân cư và trung tâm trọng yếu như Khu đô thị Trung Hoà - Nhân Chính, trung tâm Hội nghị Quốc gia, trung tâm Triển lãm Quốc gia, trường THPT Chuyên Amsterdam ...

\begin{figure}[H]
	\centering
	\includegraphics[width=\textwidth]{Google_maps.png}
	\caption{Vị trí công trình thông qua Google Maps}
\end{figure}
	
	\subsection{MỤC ĐÍCH SỬ DỤNG}
	- Với diện tích khu đất 19 689m$^{2}$, diện tích xây dựng là 7 799m$^{2}$, do Tập đoàn Charm Vit, Hàn Quốc phát triển với mức phát triển với mức đầu tư trên 120 triệu USD, bao gồm một toà tháp văn phòng hạng A 27 tầng, một toà tháp khách sạn 5 sao 27 tầng và một khu trung tâm thương mại cao cấp 5 tầng.

\begin{figure}[H]
	\centering
	\includegraphics[width=1\textwidth]{ban-ve-khach-san-5-sao-Ha-noi-Plaza_3.jpg}
	\caption{Phối cảnh công trình}
\end{figure}

	- Khách sạn có quy mô 27 tầng và 2 tầng hầm, chiều cao trên 100m với tổng diện tích đất 19 689m$^{2}$, diện tích sàn xây dựng là 150 000m$^{2}$.
	
	- Tầng 1 đến tầng 4 dùng vào các hoạt động dịch vụ như: trung tâm thương mại, siêu thị và nhà hàng ăn uống Âu và Á,...
	
	- Ngoài ra, Ha Noi Plaza còn có hệ thống 1 phòng họp có chứa được khoảng 800 người, có sân golf tập, bể bơi trong nhà và ngoài trời và một số dịch vụ công cộng...
		
	- Từ tầng 5 đến tầng 27 với diện tích 101 104m$^{2}$ và được làm văn phòng cho thuê với diện tích 53 443m$^{2}$ (riêng tầng 26 và tầng 27 được dùng làm nhà hàng).	
		
	- Khu mua sắm 5 tầng này có diện tích lên tới 15 000m$^{2}$, sẽ là trung tâm thương mại cao cấp đầu tiên trong khu vực.
	
	\subsection{TÓM TẮT CÁC TẦNG CỦA TOÀ NHÀ}
	\begin{itemize}
	\setlength\itemsep{1mm}
		\item \emph{Tầng 1}: Là không gian sang trọng với đồ trang sức cao cấp, mỹ phẩm, nước hoa và đồng hồ.
	
		\item \emph{Tầng M}: Là nơi tập trung các gian hàng thời trang, phụ kiện thời trang, đồ da thương hiệu quốc tế cho cả nam lẫn nữ.
	
		\item \emph{Tầng 2}: Là các gian hàng thời trang công sở, trang phục hàng ngày thương hiệu mạnh Việt Nam, đồ lưu niệm, trang phục và dụng cụ thể thao.
	
		\item \emph{Tầng 3}: Là khu mua sắm cho mẹ và bé, đồ trang trí nội thất, đồ gia dụng, chăn nệm, đồ điện tử cao cấp.
	
		\item \emph{Tầng 4}: Là khu ẩm thực với 17 quầy food court đa dạng, 2 quán cà phê và một nhà hàng rộng 400m$^{2}$.
	\end{itemize}
	
	- \textbf{Đặc biệt}, tại tầng M sẽ là một siêu thị mini rộng gần 400m$^{2}$, tầng 2 là khu vui chơi giải trí dành cho các bé và các máy trò chơi cho thanh thiếu niên, tầng sẽ là một showroom trang trí nội thất sang trọng.
	
	\section{KHÍ HẬU}
	- Hà Nội là khí hậu nhiệt đới gió mùa ẩm, mùa hè nóng, mưa nhiều và mùa đông lạnh, ít mưa. Thời tiết tại đây được chia làm 2 mùa: \textbf{mùa mưa} (\emph{từ tháng 4 đến tháng 10}) và \textbf{mùa khô} (\emph{từ tháng 11 đến tháng 3}).
	
	- Mùa nóng bắt đầu từ tháng 5 đến tháng 8, khí hậu nóng ẩm vào đầu mùa và cuối mùa mưa nhiều, khô ráo vào tháng 9 và tháng 10, mùa lạnh bắt đầu từ tháng 11 đến tháng 3 năm sau.
	
	- Từ cuối tháng 11 đến nửa đầu tháng 2 rét và hành khô, từ nửa cuối tháng 2 đến hết tháng 3 lạnh và mưa phùn kéo dài từng đợt, trong khoảng tháng 9 đến giữa tháng 11, Hà Nội có những ngày thu với tiết trời mát mẻ.
	
	- Nhiệt độ trung bình mùa đông là 16.4$^{\circ}$C, trung bình mùa hạ 29.2$^{\circ}$C (lúc cao nhất lên tới 42.8$^{\circ}$C). Nhiệt độ trung bình cả năm 23.6$^{\circ}$C, lượng mưa trung bình hàng năm vào mức 1800mm đến 2000mm, do chịu ảnh hưởng của hiệu ứng đô thị và là vùng khí hậu có độ ẩm cao nên những đợt nắng nóng, nhiệt độ cảm nhận thực tế luôn cao hơn mức đo đạc, có thể lên tới 50$^{\circ}$C.
	
	- Lượng bức xạ tổng cộng trung bình hằng năm ở Hà Nội là 122.8 $ kcal/cm^{2} $ với 1641 giờ nắng và nhiệt độ không khí trung bình hằng năm là 23.6$^{\circ}$C, cao nhất là tháng 6 (29.8$^{\circ}$C), thấp nhất là tháng 1 (17.2$^{\circ}$C). Hà Nội có độ ẩm \& lượng mưa khá lớn. Độ ẩm tương đối lớn trung bình hàng năm là 79\%. Lượng mưa trung bình hàng năm là 1800mm và mỗi năm có khoảng 144 ngày mưa.
	
	\section{CẤP ĐIỆN - NĂNG LƯỢNG CHO TOÀ NHÀ}
	Hệ thống điện nặng là hệ thống điện chính của tòa nhà bao gồm hệ thống Điện Động Lực và hệ thống Điện Điều Khiển. Sử dụng nguồn điện chính 3 Pha 380 Volt hoặc 1 pha 220 Volt.
	
	- Nguồn Cấp Điện Chính:
	Trạm Biến Áp Điện Lực + Tủ Tụ Bù ==> ATS + Máy Phát ==> UPS lưu Điện ==> Tải sử dụng Trực tiếp.
	
	- Tải Sử Dụng Trực Tiếp: Từng căn hộ sử dụng điện 1 pha, Máy Bơm Cấp Thoát Nước, Thang Máy, Điều Hòa v.v...
	
	\begin{figure}[H]
		\centering
		\caption{Sơ Đồ Hệ Thống Cơ Điện cho toà nhà}
		\includegraphics[scale=0.6]{sodohethongcodien.jpg}
	\end{figure}
	