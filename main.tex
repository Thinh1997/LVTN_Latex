\documentclass[a4paper]{report}

\usepackage{times} %dung de su dung font chu Times New Roman

\usepackage{multicol} %dung de su dung khi dung excel sang table Latex

\usepackage{multirow} %dung de su dung khi dung excel sang table latex

\usepackage{bigstrut} %su dung thay package booktabs

\usepackage{multido} %dung de ve may dotted lines

\usepackage[utf8]{vietnam} %dung de viet tieng Viet

\usepackage{fancyhdr} %dung de tao ra cac style kho giay

\usepackage[Rejne]{fncychap}
\makeatletter
\def\@makeschapterhead#1{%
	\vspace*{50\p@}%
	{\parindent\z@\raggedright\normalfont\interlinepenalty\@M
		\DOTIS{#1}%
		\vskip-30\p@ 
		\kern-.6\p@
		\hrule
		\vskip-80\p@}%
}
\ChNumVar{\fontsize{48}{18}\selectfont}
\ChNameVar{\centering\fontsize{30}{18}\selectfont}
\ChRuleWidth{1mm}
\makeatother

\usepackage[a4paper,left=3cm, right=2cm, top=2cm, bottom=2cm, includefoot, includehead, headsep=0.5cm]{geometry} %dung de dinh dang kho giay

\usepackage{amsmath} %dung de tao ra cac phuong trinh toan hoc

\usepackage{fontawesome5} %package dùng để sử dụng icon để trang trí này nọ cho đẹp

\usepackage{wasysym} %dùng để sử dụng icon wasysysm
\usepackage{amssymb} %dùng để sử dụng icon amssymb

\usepackage{graphicx} %dung de su dung khi insert image 
\graphicspath{{D:/TOT NGHIEP/LVTN/Pic/}}

\usepackage{tocloft}
\setlength{\cftbeforechapskip}{1em}
\setlength{\cftbeforesecskip}{0.5em}
\setlength{\cftbeforesubsecskip}{0.5em}
\let\oldcftchappresnum\cftchappresnum
\renewcommand*{\cftchappresnum}{CHƯƠNG \oldcftchappresnum}
\setlength{\cftchapnumwidth}{3cm}
\cftsetindents{section}{1cm}{0.7cm} %chinh khoang cach trong TOC
\cftsetindents{subsection}{1.5cm}{1.5cm}
\addtocontents{toc}{~\hfill\textbf{Trang}\par}

\usepackage[hidelinks, unicode]{hyperref} %dùng để làm toc có thể click được

\usepackage{float} %dung de tao, di chuyen cac object nhu table, image...

\usepackage{adjustbox} %dung de dieu chinh scale nhung table qua to

\usepackage{subcaption} %dùng để tạo subfigure

\usepackage{lscape} %dung de xoay ngang to giay

\usepackage{scrextend} %dung de dieu chinh size font chu
\changefontsizes{13pt}

\usepackage{indentfirst} %dùng để sau khi section thì dòng đầu tiên thục vào n cm
\setlength{\parindent}{1cm}
\renewcommand{\baselinestretch}{1.2} %dùng để giãn dòng 1.2

\widowpenalty=1000
\clubpenalty=1000

\usepackage{wrapfig} %dùng để wrap các text xung quanh table hoặc figure(image)

\usepackage{enumitem} %dùng để thiết kế lại các định dạng khi sử dụng list hoặc item, enumerate

\usepackage{anyfontsize} %dùng cho mọi kích cỡ chữ

\usepackage[dvipsnames]{xcolor} %dung de highlight chu viet hoac doi mau chu
\usepackage{listings} %dùng để add code. Tạo vùng code.
\definecolor{codegreen}{rgb}{0,0.6,0}
\definecolor{codegray}{rgb}{0.5,0.5,0.5}
\definecolor{codepurple}{rgb}{0.58,0,0.82}
\definecolor{codebackcolour}{rgb}{0.95,0.95,0.92}

\lstdefinestyle{mystyle}{
	backgroundcolor=\color{codebackcolour},   
	commentstyle=\color{codegreen},
	keywordstyle=\color{magenta},
	numberstyle=\tiny\color{codegray},
	stringstyle=\color{codepurple},
	basicstyle=\ttfamily\footnotesize,
	breakatwhitespace=false,         
	breaklines=true,                 
	captionpos=b,                    
	keepspaces=true,                 
	numbers=left,                    
	numbersep=5pt,                  
	showspaces=false,                
	showstringspaces=false,
	showtabs=false,                  
	tabsize=2
}
\lstset{style=mystyle}

\usepackage{tcolorbox} %dung de tao cac colorbox

\usepackage[perpage]{footmisc} %dùng để reset số footcite mỗi trang.

\usepackage[backend=bibtex, style=verbose-trad2]{biblatex} %dùng để tạo "TÀI LIỆU THAM KHẢO"
\citetrackerfalse
\interfootnotelinepenalty=10000 %dung de tranh tinh trang footcite qua trang khac
\addbibresource{thamkhao.bib}
\renewcommand*{\thefootnote}{(\arabic{footnote})} %dùng để viết footcite có trong ngoặc đơn

\usepackage{titlesec}
\titleformat{\section}
{\selectfont\fontsize{12}{8}\bfseries\color{Red}}{\thesection}{1em}{}
\titleformat{\subsection}
{\selectfont\fontsize{12}{8}\bfseries\color{Cerulean}}{\thesubsection}{1em}{}
\titleformat{\subsubsection}
{\selectfont\fontsize{12}{8}\bfseries\color{Green}}{\thesubsubsection}{1em}{}
\titlespacing*{\section}
{0cm}{0.2cm}{0.2cm}
\titlespacing*{\subsection}
{0cm}{0.2cm}{0.2cm}
\titlespacing*{\subsubsection}
{0cm}{0.2cm}{0.2cm}

\setlist[itemize]{leftmargin=2.2cm}
\setlist[enumerate]{leftmargin=2.2cm}

%Phan nay la su dung package fancyhdr
\pagestyle{fancy}
\fancyhf{}
\renewcommand{\footrulewidth}{1pt}
\renewcommand*{\thesection}{\Alph{section}.}
\renewcommand*{\thesubsection}{\hspace{0.75cm}\Roman{subsection}.}
\renewcommand*{\thesubsubsection}{\hspace{1cm}\arabic{subsubsection}.}

% Đầu trang có số trang, số chương & chân trang có tên đề tài
\fancyhead[L]{\leftmark}
\fancyhead[R]{Trang \thepage}
\fancyfoot[C]{IoT HVAC}

%Tạo 1 lệnh để tạo chương mới, trình độ này thượng thừa vl ra
\newcommand{\chapmoi}[1]{
	\chapter{\textbf{#1}}
	\begin{center}
		{\Huge \APLminus}
		{\Huge \faBookOpen}
		{\Huge \APLminus}
	\end{center}
	\newpage}

\newcommand{\dottedline}[1]{
	\par\nobreak
	\noindent\rule{0pt}{1.5\baselineskip}
	\multido{}{#1}{\noindent\makebox[\linewidth]{\dotfill}\endgraf}
	\bigskip}

%Đây chính là phần soạn thảo Latex, add các file vào để compile
%Khi compile thì compile file chính là đủ.
\begin{document}
	\fancyhead[L]{LỜI NÓI ĐẦU}
\begin{center}
	\textbf{{\Large \textcolor{red}{LỜI NÓI ĐẦU}}}
\end{center}

Nước ta hiện đang là một trong những quốc gia có mức độ công nghiệp hoá và hiện đại hoá đang phát triển rất mạnh mẽ trong những năm gần đây. Khi mà đời sống tăng cao thì nhu cầu về điều hoà càng cao, có thể nói hầu như trong tất cả các cao ốc, văn phòng, khách sạn, bệnh viện, nhà hàng, một số phân xưởng,… đã và đang xây dựng đều trang bị hệ thống điều hòa không khí. Mục đích của việc điều hòa không khí là tạo ra môi trường vi khí hậu thích hợp cho điều kiện sinh lý của con người và nâng cao độ tin cậy hoạt động của các trang thiết bị công nghệ.

Tuy nhiên, nếu chỉ đơn thuần là việc điều hoà không khí và vận hành thôi thì trong thời đại kỹ thuật số hiện nay là chưa đủ. Nhu cầu cấp tiến hơn cho các hệ thống lớn là ngoài việc vận hành phải đảm bảo được an toàn, ổn định thì còn phải đạt được tiêu chí tiết kiệm điện, giảm giá thành vận hành hệ thống. Hệ thống càng lớn thì mức độ tiết kiệm càng được để ý nhiều hơn. 

Với đề tài ``IoT HVAC''. Để thực nhiện đề tài này, chúng em đã vận dụng kiến thức, kinh nghiệp làm việc và các tài liệu liên quan để tính toán, thiết kế dưới sự hướng dẫn tận tình của thầy Đỗ Trí Nhựt.

Vì đây là lần đầu tiên thực hiện việc tính toán, thiết kế cho một công trình lớn, hơn nữa kiến thức chuyên môn còn hạn chế và chưa có kinh nghiệm thực tế nên sẽ có nhiều thiếu sót. Kính mong được sự chỉ bảo cũng như góp ý quý báo của quý thầy cô để chúng em có được kinh nghiệm và tiến bộ sau này. Chúng em xin chân thành cảm ơn!

\begin{flushright}
	Hồ Chí Minh, tháng 02 năm 2020
	
	Sinh viên thực hiện
	
	Nguyễn Phúc Thịnh
	
	Nguyễn Thành Nhân
	
	Nguyễn Vũ Trường
\end{flushright}

	%Trang nhan xet cua giao vien huong dan
\newpage
\fancyhead[L]{NHẬN XÉT CỦA GIÁO VIÊN}
\begin{center}
	\textbf{{\large \textcolor{red}{NHẬN XÉT CỦA GIÁO VIÊN HƯỚNG DẪN}}}
\end{center}
\dottedline{33}

%Trang nhan xet cua giao vien phan bien
\newpage
\fancyhead[L]{NHẬN XÉT CỦA GIÁO VIÊN}
\begin{center}
	\textbf{{\large \textcolor{red}{NHẬN XÉT CỦA GIÁO VIÊN PHẢN BIỆN}}}
\end{center}
\dottedline{33}

%Trang viet loi cam on
\newpage
\fancyhead[L]{LỜI CẢM ƠN}
\begin{center}
	\textbf{{\Large \textcolor{red}{LỜI CẢM ƠN}}}
\end{center}

	Trong suốt hơn 4 năm học tập và rèn luyện dưới giảng đường trường Đại Học Văn Lang với lòng yêu nghề và sự tận tâm, hết lòng vì sinh viên của các thầy cô trong khoa Kỹ Thuật Nhiệt đã giúp sinh viên tích lũy được rất nhiều kiến thức chuyên ngành và kỹ năng cần thiết trong cuộc sống.
	
	Lời đầu tiên chúng em xin được gửi đến cha mẹ, gia đình lời cảm ơn và lòng biết ơn sâu sắc vì những lời động viên, ủng hộ và cổ vũ tinh thần trong suốt quá trình thực hiện và hoàn thành bài luận văn nay.
	
	Nhóm em xin được gửi lời cảm ơn chân thành tới trường Đại Học Văn Lang, phòng đào tạo, văn phòng khoa và đặc biệt là Ths. Đỗ Trí Nhựt đã trực tiếp hướng dẫn, giúp đỡ nhóm trong suốt quá trình hoàn thành đồ án tốt nghiệp với đề tài ``ỨNG DỤNG IoT DÀNH CHO HVAC'' hay còn gọi tắt là ``IoT HVAC''.
	
	Xin chân thành cảm ơn tới Ts. Lê Hùng Tiến trưởng khoa Kỹ Thuật Nhiệt đã tận tâm giúp đỡ sinh viên trong suốt quá trình học tập và tích lũy được nhiều kiến thức trong trường học, trường đời. 
	
\begin{flushright}
	Nhóm em xin chân thành cảm ơn!		
\end{flushright}
	\fancyhead[L]{MỤC LỤC}
\chapter*{MỤC LỤC}
\renewcommand{\contentsname}{}
\tableofcontents %đây là lệnh tạo mục lục
\setcounter{secnumdepth}{3} %report không có đánh số cho subsubsection nên lệnh này dùng đánh số cho subsubsection
	\fancyhead[L]{\leftmark}
%Trang bìa
\chapmoi{TỔNG QUAN VỀ CÔNG TRÌNH THIẾT KẾ}

%nội dung
	\section{KHÁI QUÁT VỀ CÔNG TRÌNH}
	\subsection{VỊ TRÍ CÔNG TRÌNH}
	- Nằm toạ lạc trên đường Trần Duy Hưng (\emph{địa chỉ cụ thể là: \textbf{117 Trần Duy Hưng, Trung Hoà, Cầu Giấy, Hà Nội}}) và có vị trí chiến lược trong việc phát triển Hà Nội.
	
	 - Nằm gần khu dân cư và trung tâm trọng yếu như Khu đô thị Trung Hoà - Nhân Chính, trung tâm Hội nghị Quốc gia, trung tâm Triển lãm Quốc gia, trường THPT Chuyên Amsterdam ...

\begin{figure}[H]
	\centering
	\includegraphics[width=0.7\textheight]{Google_maps.png}
	\caption{Vị trí công trình thông qua Google Maps}
\end{figure}
	
	\subsection{MỤC ĐÍCH SỬ DỤNG}
	- Với diện tích khu đất 19 689m$^{2}$, diện tích xây dựng là 7 799m$^{2}$, do Tập đoàn Charm Vit, Hàn Quốc phát triển với mức phát triển với mức đầu tư trên 120 triệu USD, bao gồm một toà tháp văn phòng hạng A 27 tầng, một toà tháp khách sạn 5 sao 27 tầng và một khu trung tâm thương mại cao cấp 5 tầng.

\begin{figure}[H]
	\centering
	\includegraphics[width=0.7\textheight]{ban-ve-khach-san-5-sao-Ha-noi-Plaza_3.jpg}
	\caption{Phối cảnh công trình}
\end{figure}

	- Khách sạn có quy mô 27 tầng và 2 tầng hầm, chiều cao trên 100m với tổng diện tích đất 19 689m$^{2}$, diện tích sàn xây dựng là 150 000m$^{2}$.
	
	- Tầng 1 đến tầng 4 dùng vào các hoạt động dịch vụ như: trung tâm thương mại, siêu thị và nhà hàng ăn uống Âu và Á,...
	
	- Ngoài ra, Ha Noi Plaza còn có hệ thống 1 phòng họp có chứa được khoảng 800 người, có sân golf tập, bể bơi trong nhà và ngoài trời và một số dịch vụ công cộng...
		
	- Từ tầng 5 đến tầng 27 với diện tích 101 104m$^{2}$ và được làm văn phòng cho thuê với diện tích 53 443m$^{2}$ (riêng tầng 26 và tầng 27 được dùng làm nhà hàng).	
		
	- Khu mua sắm 5 tầng này có diện tích lên tới 15 000m$^{2}$, sẽ là trung tâm thương mại cao cấp đầu tiên trong khu vực.
	
	\subsection{TÓM TẮT CÁC TẦNG CỦA TOÀ NHÀ}
	\begin{itemize}
	\setlength\itemsep{1mm}
		\item \emph{Tầng 1}: Là không gian sang trọng với đồ trang sức cao cấp, mỹ phẩm, nước hoa và đồng hồ.
	
		\item \emph{Tầng M}: Là nơi tập trung các gian hàng thời trang, phụ kiện thời trang, đồ da thương hiệu quốc tế cho cả nam lẫn nữ.
	
		\item \emph{Tầng 2}: Là các gian hàng thời trang công sở, trang phục hàng ngày thương hiệu mạnh Việt Nam, đồ lưu niệm, trang phục và dụng cụ thể thao.
	
		\item \emph{Tầng 3}: Là khu mua sắm cho mẹ và bé, đồ trang trí nội thất, đồ gia dụng, chăn nệm, đồ điện tử cao cấp.
	
		\item \emph{Tầng 4}: Là khu ẩm thực với 17 quầy food court đa dạng, 2 quán cà phê và một nhà hàng rộng 400m$^{2}$.
	\end{itemize}
	
	- \textbf{Đặc biệt}, tại tầng M sẽ là một siêu thị mini rộng gần 400m$^{2}$, tầng 2 là khu vui chơi giải trí dành cho các bé và các máy trò chơi cho thanh thiếu niên, tầng sẽ là một showroom trang trí nội thất sang trọng.
	
	\section{KHÍ HẬU}
	- Hà Nội là khí hậu nhiệt đới gió mùa ẩm, mùa hè nóng, mưa nhiều và mùa đông lạnh, ít mưa. Thời tiết tại đây được chia làm 2 mùa: \textbf{mùa mưa} (\emph{từ tháng 4 đến tháng 10}) và \textbf{mùa khô} (\emph{từ tháng 11 đến tháng 3}).
	
	- Mùa nóng bắt đầu từ tháng 5 đến tháng 8, khí hậu nóng ẩm vào đầu mùa và cuối mùa mưa nhiều, khô ráo vào tháng 9 và tháng 10, mùa lạnh bắt đầu từ tháng 11 đến tháng 3 năm sau.
	
	- Từ cuối tháng 11 đến nửa đầu tháng 2 rét và hành khô, từ nửa cuối tháng 2 đến hết tháng 3 lạnh và mưa phùn kéo dài từng đợt, trong khoảng tháng 9 đến giữa tháng 11, Hà Nội có những ngày thu với tiết trời mát mẻ.
	
	- Nhiệt độ trung bình mùa đông là 16.4$^{\circ}$C, trung bình mùa hạ 29.2$^{\circ}$C (lúc cao nhất lên tới 42.8$^{\circ}$C). Nhiệt độ trung bình cả năm 23.6$^{\circ}$C, lượng mưa trung bình hàng năm vào mức 1800mm đến 2000mm, do chịu ảnh hưởng của hiệu ứng đô thị và là vùng khí hậu có độ ẩm cao nên những đợt nắng nóng, nhiệt độ cảm nhận thực tế luôn cao hơn mức đo đạc, có thể lên tới 50$^{\circ}$C.
	
	- Lượng bức xạ tổng cộng trung bình hằng năm ở Hà Nội là 122.8 $ kcal/cm^{2} $ với 1641 giờ nắng và nhiệt độ không khí trung bình hằng năm là 23.6$^{\circ}$C, cao nhất là tháng 6 (29.8$^{\circ}$C), thấp nhất là tháng 1 (17.2$^{\circ}$C). Hà Nội có độ ẩm \& lượng mưa khá lớn. Độ ẩm tương đối lớn trung bình hàng năm là 79\%. Lượng mưa trung bình hàng năm là 1800mm và mỗi năm có khoảng 144 ngày mưa.
	
	\section{CẤP ĐIỆN - NĂNG LƯỢNG CHO TOÀ NHÀ}
	Hệ thống điện nặng là hệ thống điện chính của tòa nhà bao gồm hệ thống Điện Động Lực và hệ thống Điện Điều Khiển. Sử dụng nguồn điện chính 3 Pha 380 Volt hoặc 1 pha 220 Volt.
	
	- Nguồn Cấp Điện Chính:
	Trạm Biến Áp Điện Lực + Tủ Tụ Bù ==> ATS + Máy Phát ==> UPS lưu Điện ==> Tải sử dụng Trực tiếp.
	
	- Tải Sử Dụng Trực Tiếp: Từng căn hộ sử dụng điện 1 pha, Máy Bơm Cấp Thoát Nước, Thang Máy, Điều Hòa v.v...
	
	\begin{figure}[H]
		\centering
		\caption{Sơ Đồ Hệ Thống Cơ Điện cho toà nhà}
		\includegraphics[scale=0.6]{sodohethongcodien.jpg}
	\end{figure}
	
	%Trang bìa
\chapmoi{SƠ LƯỢC HỆ THỐNG CƠ ĐIỆN}

	\section{KỸ THUẬT LẠNH ỨNG DỤNG ĐIỀU HOÀ KHÔNG KHÍ}
	\subsection{GIỚI THIỆU VỀ KỸ THUẬT LẠNH}
	Kỹ thuật lạnh đã ra đời từ rất lâu và được sử dụng trong rất nhiều ngành nghề kỹ thuật khác nhau: trong công nghiệp chế biến \& bảo quản thực phẩm, công nghiệp hoá chất, công nghiệp rượu bia, kỹ thuật sấy nhiệt độ thấp, công nghiệp dầu mỏ, chế tạo vật liệu, dụng cụ, xử lý hạt giống, v.v...
	
	\subsection{ĐIỀU HOÀ KHÔNG KHÍ}
	\subsubsection{Giới thiệu về Hệ Thống Điều Hoà Không Khí}
	Hệ thống điều hòa không khí hiện đại đầu tiên được phát triển vào năm 1902 bởi một kỹ sư trẻ tên là \textbf{Willis Haviland Carrier}. 
	
	\begin{wrapfigure}{r}{0.3\textwidth}
		\includegraphics[width=0.9\linewidth]{WillisHavilandCarrier.jpg}
		\caption{Chân dung ông Willis Haviland Carrier}
	\end{wrapfigure}

	Ban đầu hệ thống được thiết kế để làm giảm độ ẩm của không khí trong xưởng in của một công ty xuất bằng cách thổi nó qua ống ướp lạnh. Không khí được làm mát khi nó đi qua các đường ống lạnh và trở lên khô hơn. Quá trình làm giảm độ ẩm trong nhà máy đã tạo ra một lợi ích phụ là giảm nhiệt độ không khí và một công nghệ mới đã được sinh ra. Đó là công nghệ điều hòa không khí.
	
	Khi một chất lỏng chuyển thành khí (trong một quá trình được gọi là \emph{chuyển đổi pha}), nó hấp thụ nhiệt của môi trường xung quanh. Điều hòa không khí khai thác tính năng này của giai đoạn chuyển đổi bằng cách buộc các hợp chất hóa học đặc biệt để bay hơi và ngưng tụ hơn và hơn nữa trong một hệ thống khép kín của cuộn dây.
	
	\subsubsection{Nguyên lý hoạt động}
	Hệ thống làm lạnh không khí gồm có một máy nén khí bơm gas (môi chất lạnh) áp suất cao đến dàn nóng (outdoor), tại đây khí gas dưới áp suất lớn sẽ hóa lỏng và tỏa nhiệt ra môi trường bên ngoài nhờ quạt gió (gia dụng) hoặc tháp nước (công nghiệp), hoặc bình ngưng.

	Sau đó gas dưới dạng lỏng tuần hoàn đến van tiết lưu (van này có tác dụng tạo chênh lệch áp suất cần thiết cho hệ thống). Ở đây gas từ dạng lỏng ấp suất cao sẽ được tiết lưu về dạng khí áp suất thấp, nhiệt độ thấp phun vào dàn lạnh (indoor) và thu nhiệt từ môi trường cần làm lạnh nhờ hệ thống quạt và các tấm lược gió trên các ống dẫn gas. Sau đó gas ở trạng thái khí được máy nén hút về để bơm tiếp một chu trình mới.	
	
	\begin{figure}[H]
		\centering
		\includegraphics[width=0.9\linewidth]{cau-tao-va-nguyen-ly-hoat-dong-cua-may-lanh-4.jpg} 
		\caption{Sơ đồ mô tả nguyên lý hoạt động của hệ thống điều hoà}
	\end{figure}

	\subsection{PHÂN LOẠI HỆ THỐNG ĐIỀU HOÀ}
	\subsubsection{PHÂN LOẠI THEO ĐẶC TÍNH}
	\textbf{Theo mức độ quan trọng của hệ thống điều hòa không khí}
	\begin{enumerate}[leftmargin=0.5cm]
		\setlength\itemsep{1mm}
		\item \textbf{Hệ thống điều hòa không khí cấp I:} Là hệ thống điều hoà có khả năng duy trì các thông số vi khí hậu trong nhà với mọi phạm vi thông số ngoài trời, ngay tại cả ở những thời điểm khắc nghiệt nhất trong năm về mùa Hè lẫn mùa Đông.
		\item \textbf{Hệ thống điều hòa không khí cấp II:} Là hệ thống điều hoà có khả năng duy trì các thông số vi khí hậu trong nhà với sai số không qúa 200 giờ trong 1 năm, tức tương đương khoảng 8 ngày trong 1 năm. Điều đó có nghĩa trong 1 năm ở những ngày khắc nghiệt nhất về mùa Hè và mùa Đông hệ thống có thể có sai số nhất định, nhưng số lượng những ngày đó cũng chỉ xấp xỉ 4 ngày trong một mùa.
		\item \textbf{Hệ thống điều hòa không khí cấp III:} Là hệ thống điều hoà có khả năng duy trì các thông số tính toán trong nhà với sai số không quá 400 giờ trong 1 năm, tương đương 17 ngày.
	\end{enumerate}
	
	\textbf{Theo phương pháp xử lý nhiệt ẩm}
	\begin{enumerate}[leftmargin=0.5cm]
		\setlength\itemsep{1mm}
		\item \textbf{Hệ thống điều hòa kiểu khô:} Không khí được xử lý nhiệt ẩm nhờ các thiết bị trao đổi nhiệt kiểu bề mặt. Đặc điểm của việc xử lý không khí qua các thiết bị trao đổi nhiệt kiểu bề mặt là không có khả năng làm tăng dung ẩm của không khí. Quá trình xử lý không khí qua các thiết bị trao đổi nhiệt kiểu bề mặt tuỳ thuộc vào nhiệt độ bề mặt mà dung ẩm không đổi hoặc giảm. Khi nhiệt độ bề mặt thiết bị nhỏ hơn nhiệt độ đọng sương ts của không khí đi qua thì hơi ẩm trong nó sẽ ngưng tụ lại trên bề mặt của thiết bị, kết quả dung ẩm giảm. Trên thực tế, quá trình xử lý luôn luôn làm giảm dung ẩm của không khí.
		\item \textbf{Hệ thống điều hòa không khí kiểu ướt:} Không khí được xử lý qua các thiết bị trao đổi nhiệt hỗn hợp. Trong thiết bị này không khí sẽ hỗn hợp với nước phun đã qua xử lý để trao đổi nhiệt ẩm. Kết quả quá trình trao đổi nhiệt ẩm có thể làm tăng, giảm hoặc duy trì không đổi dung ẩm không khí.
	\end{enumerate}
	
	\textbf{Theo đặc điểm của khâu xử lý nhiệt}
	\begin{enumerate}[leftmargin=0.5cm]
		\setlength\itemsep{1mm}
		\item \textbf{Hệ thống điều hòa cục bộ:} Là hệ thống điều hoà không khí trong một không gian hẹp, thường là phòng. Kiểu điều hoà cục bộ trên thực tế chủ yếu sử dụng các máy điều hoà dạng cửa sổ, máy điều hoà kiểu rời (2 mảnh) và máy điều hoà ghép.
		\item \textbf{Hệ thống điều hòa phân tán:} Máy điều hoà VRV do hãng Daikin của Nhật phát minh đầu tiên. Hiện nay hầu hết các hãng đã sản xuất các máy điều hoà VRV và đặt dưới các tên gọi khác nhau , nhưng về mặt bản chất thì không có gì khác.
		
	+ Tên gọi VRV xuất phát từ các chữ đầu tiếng Anh : \textit{Variable Refrigerant Volume}, nghĩa là hệ thống điều hoà có khả năng điều chỉnh lưu lượng môi chất tuần hoàn và qua đó có thể thay đổi công suất theo phụ tải bên ngoài.

	+ Máy điều hoà VRV ra đời nhằm khắc phục nhược điểm của máy điều hoà dạng rời là độ dài đường ống dẫn ga, chênh lệch độ cao giữa dàn nóng, dàn lạnh và công suất lạnh bị hạn chế. Với máy điều hoà VRV cho phép có thể kéo dài khoảng cách giữa dàn nóng và dàn lạnh lên đến 100m và chênh lệch độ cao đạt 50m. Công suất máy điều hoà VRV cũng đạt giá trị công suất trung bình.
	
		\item \textbf{Hệ thống điều hòa trung tâm:} Hệ thống điều hoà trung tâm là hệ thống mà khâu xử lý không khí thực hiện tại một trung tâm sau đó được dẫn theo hệ thống kênh dẫn gió đến các hộ tiêu thụ. Hệ thống điều hoà trung tâm trên thực tế là máy điều hoà dạng tủ, ở đó không khí được xử lý nhiệt ẩm tại tủ máy điều hoà rồi được dẫn theo hệ thống kênh dẫn đến các phòng.
		
		\begin{figure}[H]
			\centering
			\includegraphics[width=0.7\textwidth]{VRV.png}
			\caption{Điều hoà trung tâm VRV}
		\end{figure}

	Tuy nhiên hệ thống này có kênh gió quá lớn (80.000 $BTU/h$ trở lên) nên chỉ có thể sử dụng trong các toà nhà có không gian lắp đặt lớn. Đối với hệ thống điều hoà trung tâm do xử lý nhiệt ẩm tại một nơi duy nhất nên chỉ thích hợp cho các phòng lớn, đông người. Đối với các toà nhà làm việc, khách sạn, công sở,... là các đối tượng có nhiều phòng nhỏ với các chế độ hoạt động khác nhau, không gian lắp đặt bé, tính đồng thời làm việc không cao thì hệ thống này không thích hợp.
	\end{enumerate}
	
	\textbf{Theo đặc điểm của môi chất giải nhiệt}
	\begin{enumerate}[leftmargin=0.5cm]
		\setlength\itemsep{1mm}
		\item \textbf{Giải nhiệt bằng gió:} 
		
		$ \bullet $ Tất cả các máy điều hoà công suất nhỏ đều giải nhiệt bằng không khí, các máy điều hoà công suất trung bình có thể giải nhiệt bằng gió hoặc nước, hầu hết các máy công suất lớn đều giải nhiệt bằng nước.
		
		$ \bullet $ Máy lạnh trung tâm sản xuất ra gió lạnh và cấp tới các không gian điều hoà qua các kênh dẫn gió. Lúc này, gió đóng vai trò trao đổi và thực hiện quá trình tăng giảm nhiệt, ẩm của không gian kín. Kết thúc công đoạn này, gió lạnh lại tuần hoàn về máy lạnh trung tâm qua một kênh dẫn gió khác (hoặc hồi trực tiếp về buồng máy) và tiếp tục một chu trình mới.
		
		$ \bullet $ Hệ thống này bao gồm: máy lạnh trung tâm, các kênh dẫn gió và phân phối gió lạnh, thiết bị giải nhiệt dàn ngưng…
		
		\break
		\item \textbf{Giải nhiệt bằng nước:} 
		
		$ \bullet $ Để nâng cao hiệu quả giải nhiệt các máy công suất lớn sử dụng nước để giải nhiệt cho thiết bị ngưng tụ. Đối với các hệ thống này đòi hỏi trang bị đi kèm là hệ thống bơm, tháp giải nhiệt và đường ống dẫn nước.
		
		$ \bullet $ Nước lạnh sản xuất ra tại các máy lạnh trung tâm được cấp tới các dàn trao đổi nhiệt đặt tại các không gian điều hoà. Lúc này, nước đóng vai trò trao đổi nhiệt, thực hiện quá trình tăng giảm giảm nhiệt và độ ẩm theo bên trong không gian. Kết thúc công đoạn này, nước lại tuần hoàn về máy lạnh trung tâm và tiếp tục một chu trình mới.
		
		$ \bullet $ Hệ thống này phù hợp với những yêu cầu điều hoà cho các không gian khác nhau có chế độ nhiệt độ – độ ẩm khác nhau.( ở mỗi không gian riêng biệt ta có thể lựa chọn một nhiệt độ – độ ẩm tuỳ thích, tuỳ thuộc vào cách khống chế tại không gian đó)
		
		$ \bullet $ Yêu cầu về không gian lắp đặt cho hệ thống này không cao lắm. Khoảng cách giữa trần giả và đáy dầm khoảng từ 100 – 200 $ mm $ là có thể thực hiện được.
	\end{enumerate}

\begin{figure}[H]
\begin{subfigure}{0.5\textwidth}
\begin{center}
		\includegraphics[width=0.9\linewidth]{429_thap_giai_nhiet_liang_chi_lbc_15rt_.jpg} 
	\caption{Tháp giải nhiệt bằng nước}
\end{center}
\end{subfigure}
\begin{subfigure}{0.5\textwidth}
\begin{center}
		\includegraphics[width=0.9\linewidth]{Chiller_giai_nhiet_gio.jpg}
	\caption{Chiller sử dụng giải nhiệt gió}
\end{center}
\end{subfigure}
\caption{Các loại giải nhiệt}
\end{figure}
	
	\textbf{Theo khả năng xử lý nhiệt}
	\begin{enumerate}[leftmargin=0.5cm]
		\setlength\itemsep{1mm}
		\item \textbf{Điều hoà một chiều lạnh:} Máy chỉ có khả năng làm lạnh về mùa Hè về mua đông không có khả năng sưởi ấm.
		
		\item \textbf{Điều hoà hai chiều lạnh:} Máy có hệ thống van đảo chiều cho phép hoán đổi chức năng của các dàn nóng và lạnh về các mùa khác nhau. Mùa Hè bên trong nhà là dàn lạnh, bên ngoài là dàn nóng về mùa đông sẽ hoán đổi ngược lại.
	\end{enumerate}
	
\begin{figure}[H]
	\centering
	\includegraphics[width=0.9\linewidth]{vi-vn-daikin-fthf25ravmv-1.jpg}
	\caption{Mô tả máy điều hoà dân dụng 2 chiều}
\end{figure}
	
	\textbf{Theo đặc điểm máy nén}
	\begin{enumerate}[leftmargin=0.5cm]
		\setlength\itemsep{1mm}
		\item \textbf{Máy nén PISTON}
		
		$ \bullet $ \textbf{Phân loại theo số lượng piston:} 1 piston, 2 piston, 3 piston, 4 piston v.v...
		
		$ \bullet $ \textbf{Phân loại theo hình dạng:} 1 piston thường là loại kín đặt ngập trong dầu, sơn màu đen, hình dáng hinh trụ tròn và mập lùn hơn so với loại xoắn ốc. Loại 2 piston trở lên thì công suất lớn hơn, nữa kín, thường piston đặt lệch nhau trong một mặt phẳng, ta phân biệt qua số lượng mặt bích hình thoi tròn và dẹp trên thân máy. Thường là gia công đúc, nên máy có hình thoi theo kiểu khối.
		
		$ \ast $ \textbf{Ưu điểm:} tỉ số nén cao, dùng cho máy nén nhiều cấp độ bay hơi sâu. Loại năng suất nhỏ dùng cho hầu hết là: tủ lạnh, máy điều hòa cục bộ v.v...
		
		$ \ast $ \textbf{Khuyết điểm:} hiệu suất thấp, ồn không hiệu quả đối với dãy công suất lớn.
		
\begin{figure}[H]
	\centering
	\includegraphics[width=0.9\linewidth]{bitzer1.jpg}
	\caption{Bên trong máy nén PISTON}
\end{figure}		
		
		\item \textbf{Máy nén Xoắn Ốc} 
		
		Năng suất lạnh \textit{trung bình} và \textit{vừa} (3 $ HP $ đến 60 $ HP $ = 4 máy xoắn ốc ghép song song).
		
		$ \bullet $ \textbf{Phân loại theo kích cỡ máy nén:} 3, 4, 5, 8, 10, 12, 15 $ HP $.
		
		$ \bullet $ \textbf{Phân loại theo hình dáng:} máy theo kiểu kín, đặt ngập trong dầu, máy hình trụ đứng và tròn 2 đầu, sơn đen hình thon và cao gấp đôi máy nén piston một cấp.
		
		$ \bullet $ \textbf{Cấu tạo của máy nén xoắn ốc gồm:} 1 scroll đứng yên và một phần scroll chuyển động theo đĩa lệnh tâm.
		
		–- Không có van hút và van đẩy nên tạo được ưu điểm lọai được áp suất rơi gây ra bởi các van nên tăng hiệu suất năng lượng của chu trình.
		
		–- Không tồn tại không gian chết – Hiệu suất thể tích tăng gần 100%.
		
		–- Rất ít chi tiết chuyển động – Tỉ lệ hư hỏng máy nén giảm tối đa.
		
		–- Việc nén gas liên tục trong các túi của scroll – Vận tốc xoay luôn được giữ ở mức thấp.
		
		–- Dải công suất rộng lớn dễ dàng cho khách hàng lựa chọn phù hợp với yêu cầu.
		
		–- Hệ số hiệu quả làm lạnh COP lớn.
		
		$ \bullet $ \textbf{Đặc tính khởi động tải tối ưu:}
		
		–- Máy nén scroll có ưu điểm khởi động giảm tải ngay cả khi áp suất hệ thống không cân bằng.
		
		–- Khi máy nén ngừng thì các scroll được tách ra và áp suất lúc này cân bằng.
		
		–- Khi máy nén khởi động trở lại, nó không ở điều kiện giảm tải. Vì áp suất sẽ tăng dần cho đến khi vượt quá áp suất đẩy làm van mở và thiết lập lại sự liên tục của hệ thống.

\begin{figure}[H]
	\begin{subfigure}{0.5\textwidth}
		\includegraphics[width=0.9\linewidth]{may-nen-xoan-oc.jpg} 
		\caption{Máy nén xoắn ốc}
	\end{subfigure}
	\begin{subfigure}{0.5\textwidth}
		\includegraphics[width=0.9\linewidth]{may-nen-khi-xoan-oc-ben-trong.jpg}
		\caption{Cấu tạo bên trong của máy nén xoắn ốc}
	\end{subfigure}
\end{figure}
		
		$ \ast $ \textbf{Ưu điểm \& đặc điểm vận hành:} 
		
		–- Có khả năng tránh được hiện tượng ngập lỏng và cho phép một lượng nhỏ chất bẩn rắn đi qua mà không làm hư hỏng phần SCROLL.
		
		–- Để tránh được hiện tượng ngập lỏng và cặn bẩn được là nhờ vào khả năng tương thích trục và tương thích bán kính trong máy nén SCROLL. Trong nhiều trường hợp không cần bình tách lỏng hoặc bình chứa lỏng lắp trên đường hút. Khi cần thiết thì máy nén scroll chỉ cần sấy cacte.
		
		–- Khả năng tương thích theo bán kính. Khi có lỏng hay chất bẩn thí scroll tách ra cho phép chúng đi qua tự do không làm hư hỏng máy nén.
		
		–- Khả năng tương thích theo trục. Khi quá tải, scroll cố định tách lên phía trên scroll quay để làm sạch khỏi máy nén bất kì lượng lỏng thừa nào.
		
		–- Hạn chế tối đa sự rung động.
		
		\item \textbf{Máy nén Trục Vít}
		
		Hiệu suất cao ở dãy năng suất lạnh lớn (40 $ tons $ đến 900 $ tons $).
		
		$ \bullet $ \textbf{Phân loại máy nén trục vít:} Vít đơn, vít đôi.
		
		Trong ngành lạnh thường chọn theo máy nén vít đôi, các công ty như Daikin (Nhật), vilter (USA) v.v... thì sử dụng vít đơn.

\begin{figure}[H]
	\begin{subfigure}{0.5\textwidth}
		\includegraphics[width=0.9\linewidth]{maynentrucvitcautao.jpg}
		\caption{Cấu tạo bên trong của máy nén trục vít}
	\end{subfigure}
	\begin{subfigure}{0.5\textwidth}
	\includegraphics[width=0.9\linewidth]{maynentrucvit.png} 
	\caption{Máy nén trục vít}
	\end{subfigure}
\end{figure}			
		$ \ast $ \textbf{Ưu điểm:} 
		
		–- Độ tin cậy cao, tuổi thọ cao
		
		–- Kích thước nhỏ gọn
		
		–- Không có các chi tiết chuyển động tịnh tiến và quán tính kèm theo
		
		–- Hầu như không có hiện tượng va đập thủy lực
		
		–- Trong cùng một máy nén có thể thực hiện 2 hay nhiều cấp nén
		
		–- Các chỉ tiêu năng lượng và thể tích ổn định trong thời gian vận hành lâu dài…
		
		$ \ast $ \textbf{Khuyết điểm:} 
		
		–- Việc chế tạo đòi hỏi phải có trình độ cao
		
		–- Dầu bôi trơn cho may nén phải là dâu chuyên dụng
		
		–- Để phun dầu vào máy nén cần phải tiêu tốn 1 công nhất định
		
		\break
		\item \textbf{Máy nén Ly Tâm} 
		
		- Máy nén ly tâm nhỏ gọn với cấu trúc \textit{đơn giản, ít bộ phận chuyển động, đáng tin cậy, bền, chi phí hoạt động thấp}; dễ dàng để thực hiện nhiều mức độ nén, và hàng loạt các quá trình bay hơi nhiệt độ làm mát trung gian, dễ dàng, \textit{cho phép tiêu thụ điện năng thấp hơn}; dầu bôi trơn tua-bin ly tâm có rất ít quá trình trao đổi nhiệt vì hiệu quả sinh nhiệt giữa các chi tiết máy nhỏ, nên máy có hiệu quả năng suất lạnh cao.
		
		$ \bullet $ \textbf{Phân loại máy nén ly tâm:} Có 2 loại (COP thường 6.0 trở lên)
		
		\textbf{Loại ly tâm thường} dùng một cánh quạt ly tâm, ứng dụng lực ly tâm (giống với máy bơm nước ly tâm), với cánh quạt lớn hơn rất nhiều so với máy nén turbocor ly tâm, ứng dụng với dãy công suất rất lớn từ 500 tons trở lên, thường những tập đoàn lớn sử dụng công nghệ này với tải rất lớn: Trane, carrier, york, climaveneta, dunham bush v.v...
		
		\textbf{Loại ly tâm turbo} thường dùng 2 cánh quạt ly tâm với động cơ một chiều và dể dàng thay đổi tốc độ nhờ điều chỉnh điện áp cấp tùy theo năng suất lạnh tương ứng, hảng danfoss đang sử dụng công nghệ này đó là máy nén Frictionless centrifugal (Danfoss Turbocor), sử dụng năng lượng hiệu quả cao nhờ thiết kế công nghệ từ. Công suất từ 60 đến 300 tons. Loại turborco: công suất nhỏ. Loại ly tâm trực tiếp công suất lớn.

\begin{figure}[H]
	\centering
	\includegraphics[width=0.6\linewidth]{maynenlytam.jpg}
	\caption{Máy nén ly tâm trong công nghiệp}
\end{figure}			
		
		$ \ast $ \textbf{Ưu điểm:} 
		
		–- Kích thước và trọng lượng nhỏ , đặc biệt với năng suất lạng rất lớn
		
		–- Cấu tạo đơn giản vận hành tin cậy và tuổi thọ kéo dài
		
		–- Có thể truyền động trực tiếp từ động cơ quay nhanh
		
		–- Cân bằng tốt cho nền móng nhẹ nhàng có thể đặt trực tiếp lên các thiết bị khác
		
		–- Dòng tác nhân lạnh ra khỏi máy nén một cách đồng đều , không có dầu bôi trơn trong máy nén tăng hệ số truỵền nhiệt
		
		–- Có thể nén tiết lưu nhiều cấp 
		
		$ \ast $ \textbf{Khuyết điểm:} 
		
		–- Hiệu suất thấp hơn đối với các máy nhỏ và trung bình
		
		–- Cần có bộ tăng tốc khi có sử dụng động cơ điện
		
		Các công ty sử dụng công nghệ turborco như danfoss, sử dụng động cơ một chiều để thay đổi tốc độ và tăng hiệu suất máy nén.
		
	\end{enumerate}
	
	\subsubsection{PHÂN LOẠI THEO CÁCH LẮP ĐẶT}
	\begin{enumerate}[leftmargin=0.5cm]
		\item \textbf{Điều hoà không khí dạng cửa sổ (WINDOW TYPE)} 
		
		Máy lạnh dạng cửa sổ thường được lắp đặt trên các tường trông giống như các cửa sổ nên được gọi là máy lạnh dạng cửa sổ. Dàn lạnh và dàn nóng được nằm chung trong 1 khối. Máy lạnh dạng cửa sổ là máy lạnh có công suất nhỏ nằm trong khoảng 7.000 $ BTU/h $ đến 24.000 $ BTU/h $ với các model chủ yếu sau có công suất sau 7.000 $ BTU/h $, 9.000 $ BTU/h $, 12.000 $ BTU/h $ và 18.000 $ BTU/h $. 
		
		Tuỳ theo hãng máy mà số model có thể nhiều hay ít. Hiện nay loại máy này không còn sản xuất nữa và thay vào đó là loại máy treo tường 2 khối.
		
\begin{figure}[H]
	\centering
	\includegraphics[width=0.75\linewidth]{may-dieu-hoa-loai-cua-so.jpg}
	\caption{Điều hoà cửa sổ}
\end{figure}
		
		\item \textbf{Điều hoà không khí loại 2 cục}
		
		Máy lạnh treo tường gồm 2 cụm dàn nóng và dàn lạnh được bố trí tách rời nhau. Giữa dàn nóng và dàn lạnh được nối với nhau bằng các ống đồng dẫn gas và dây điện điều khiển. Máy nén thường đặt ở bên trong cụm dàn nóng, và được điều khiển làm việc từ dàn lạnh của máy thông qua bộ điều khiển Remote. Máy lạnh loại treo tường thường có công suất từ 9.000 $ BTU/h $ đến 48.000 $ BTU/h $, bao gồm chủ yếu các model sau : 9.000 $ BTU/h $, 12.000 $ BTU/h $,18.000 $ BTU/h $, 24.000 $ BTU/h $, 36.000 $ BTU/h $, 48.000 $ BTU/h $. 
		
		Tuỳ theo từng hãng chế tạo máy mà số model mỗi chủng loại có khác nhau. Nhưng theo thị trường hiện nay thì chỉ có loại từ 9000 $ BTU/h $ đến 24.000 $ BTU/h $.
		
\begin{figure}[H]
	\centering
	\includegraphics[width=0.9\linewidth]{phan-loai-he-thong-dieu-hoa-khong-khi-2-.jpg}
	\caption{Điều hoà loại cục nóng \& cục lạnh}
\end{figure}
		
		\item \textbf{Điều hoà không khí loại Multi - Split}
		
		Máy điều hòa loại Multi-Split về thực chất là máy điều hoà gồm \textbf{1 dàn nóng} và từ \textbf{2 - 4 dàn lạnh}. Mỗi cụm dàn lạnh được gọi là một hệ thống. Thường các hệ thống hoạt động độc lập. Mỗi dàn lạnh hoạt động không phụ thuộc vào các dàn lạnh khác. 
		
		Các máy điều hoà ghép có thể có các dàn lạnh chủng loại khác nhau. Máy điều hòa dạng ghép có những đặc điểm và cấu tạo tương tự máy điều hòa kiểu rời.
		
		Tuy nhiên do dàn nóng chung nên tiết kiệm diện tích lắp đặt.

\begin{figure}[H]
	\centering
	\includegraphics[width=0.9\linewidth]{multi_split.png}
	\caption{Điều hoà loại Multi - Split}
\end{figure}		
		
		\item \textbf{Điều hoà không khí dạng tủ đứng}
		
		Máy lạnh tủ đứng là máy có công suất trung bình. Đây là dạng máy rất hay được lắp đặt ở các nhà hàng và các sảnh của các cơ quan. Công suất của máy từ 36.000 $ BTU/h $ đến 100.000 $ BTU/h $. 
		
		Về nguyên lý lắp đặt cũng giống như máy lạnh treo tường 2 khối gồm dàn nóng, dàn lạnh và hệ thống ống đồng, dây điện nối giữa chúng. 
		
		Ưu điểm của máy là gió lạnh được tuần hoàn và thổi trực tiếp vào không gian điều hoà nên tổn thất nhiệt rất ít.
\begin{figure}[H]
	\centering
	\includegraphics[width=0.3\linewidth]{image_10.png}
	\caption{Điều hoà loại tủ đứng}
\end{figure}		
	\end{enumerate}
	
	\section{HỆ THỐNG CẤP THOÁT NƯỚC VÀ CHỮA CHÁY}
	\subsection{GIỚI THIỆU VỀ HỆ THỐNG CẤP THOÁT NƯỚC}
	Nước thì không thể thiếu với cuộc sống mọi người. Việc xây dựng hệ thống cáp thoát nước đòi hỏi tính kỹ thuật cao và luôn đảm bảo mọi người có đủ nước sinh hoạt tại mọi thời điểm.
	
	\subsubsection{\emph{Hệ thống cấp nước:}}
	Hầu hết hệ thống cấp nước của các tòa nhà chung cư sử dụng tích hợp của ba loại hệ thống: hệ thống cấp nước trực tiếp, hệ thống cấp nước gián tiếp và hệ thống bơm nước thải.
		
	+ Đối với hệ thống cấp nước trực tiếp, nước sạch được cấp trực tiếp từ đường ống nước công cộng đến các hộ gia đình ở các tầng thấp bằng áp suất thủy lực bên trong đường ống chính;
	
	+ Đối với hệ thống cấp nước gián tiếp, sử dụng máy bơm nước để lấy nước từ các bể chứa ở tầng trệt của tòa nhà, và hút nước sạch vào bể trên mái nhà, sau đó dẫn nước đến từng hộ gia đình thông qua mạng lưới đường ống phụ;
		
	* Đối với hệ thống bơm nước thải, nước được truyền kết thúc nhận được bằng cách lắp máy bơm áp lực để cấp nước: đường ống cứu hỏa cũng có chức năng tương tự;
	
	Hệ thống cấp nước bao gồm: \emph{máy bơm nước, đường ống đứng, bể chứa, thiết bị phao tự ngắt} và \emph{các đường ống phụ}. Tất cả các phần cố định của hệ thống cấp nước phải được thường xuyên kiểm tra và duy trì hoạt động đúng cách và tất cả các bể nước phải được làm sạch theo định kỳ để kiểm soát chất lượng tốt nhất.
	\begin{figure}[H]
		\centering
		\includegraphics[width=0.8\linewidth]{so-luoc-he-thong-cap-thoat-nuoc-nha-cao-tang_2.jpg}	
		\caption{Sơ đồ hệ thống cấp nước toà nhà}
	\end{figure}
	
	\subsubsection{\emph{Hệ thống thoát nước:}}
	Hệ thống thoát nước có thể được chia thành \emph{hệ thống đường ống thoát nước mưa} và \emph{hệ thống đường ống nước thải}. Các phần cố định của hệ thống thoát nước bao gồm các đường ống nước thải, xi phông, hố ga. Các đường ống nước thải phải nối sao cho phù hợp nhất, chẳng hạn như nước thải từ bồn rửa không được xả ra theo đường ống nước mưa. Ngoài ra, phải đảm bảo đầu thoát nước thải không bị rác chặn hoặc phải có lưới để ngăn  rác khỏi tắc đường ống.
	
	Tất cả các đường ống nước thải bao gồm đường ống chôn dưới đất, ống dẫn chất thải, ống thông gió và ống cống ngầm phải luôn ở trong tình trạng hoạt động tốt. Cần phải kiểm tra định kỳ tất cả các đường ống trên; nếu phát hiện rò rỉ, tắc nghẽn hoặc hư hỏng, cần tiến hành sửa chữa ngay.
	
	Để ngăn chặn khí thải và côn trùng trong đường ống xâm nhập vào khu dân cư, các thiết bị vệ sinh bao gồm bồn rửa tay, chậu rửa, bồn tắm và vòi sen, nhà vệ sinh và nắp thoát nước ở sàn phải được gắn với ống xi phông (ống xi phông hình chữ U, ống xi phông hình chai hoặc loại chống chảy ngược).
	
	Cần kiểm tra các cửa cống  thường xuyên, nếu phát hiện tắc nghẽn thì phải xử lý ngay. Các cửa cống phải được bố trí sao cho việc bảo trì  được thực hiện dễ dàng và thường xuyên. Không nên để các vật cản như đồ đạc hay cây cảnh ở khu vực này. Có thể ngăn chặn khí thải do rò rỉ từ các hố ga bằng cách sử dụng loại nắp cống hai lớp, hoặc sửa chữa ở các cạnh của lỗ cống hoặc các vết nứt ở các miệng cống.
	
	Trách nhiệm sửa chữa và bảo trì hệ thống thoát nước được xác định dựa trên hư hỏng của đường ống công cộng hoặc đường ống của từng căn hộ. Ví dụ, nếu như xảy ra nổ đường ống thoát nước mưa, hoặc tất cả các chủ sở hữu phải chịu trách nhiệm sửa chữa. Tuy nhiên, một nhánh của đường ống được nối đến một căn hộ bị hư hỏng, chủ sở hữu hoặc người cư trú trong đó căn hộ phải có trách nhiệm sửa chữa.
	
	\subsection{TỔNG QUAN VỀ HỆ THỐNG CẤP THOÁT NƯỚC}
	-- Toà nhà được thiết kế với 27 tầng. Số lượng nhà vệ sinh được thống kê trong bảng bên dưới sau đây:
	
\begin{table}[H]
		\vspace{-0.5cm}
		\centering
		\textbf{\caption{Số lượng nhà vệ sinh theo từng tầng}}
		\begin{tabular}{|c|c|c|c|c|c|c|}
			\hline
			Tầng  & 1     & 2     & 3     & 4     & 5     & 6-27 \\
			\hline
			Số lượng nhà vệ sinh & 2     & 2     & 2     & 2     & 2     & 44 \\
			\hline
		\end{tabular}
		\label{tab:soluongnvs}
\end{table}
	
	$\Rightarrow$ Như vậy thì theo bảng thống kê, có tổng cộng \textbf{54 nhà vệ sinh}.  
	
	-- Trong đó, mỗi nhà vệ sinh có đúng số lượng thiết bị sau:
	
\begin{table}[H]
	\vspace{-0.25cm}
	\centering
	\begin{minipage}[t]{.5\textwidth}
		\centering
		\caption{Nhà vệ sinh nam} 
		\begin{tabular}{|c|c|c|}
    	\hline
    	\multicolumn{1}{|c|}{\textbf{STT}} & \multicolumn{1}{c|}{\textbf{Tên thiết bị}} & \multicolumn{1}{c|}{\textbf{Số lượng}} \\
    	\hline
    	1     & Bồn tiểu nam & 4 \\
    	\hline
    	2     & Chậu rửa mặt & 2 \\
    	\hline
    	3     & Bồn cầu & 4 \\
    	\hline
    	\end{tabular} 	
  		\label{tab:tb_nvs_nam}
	\end{minipage}
	\hspace{-0.5cm}
	\begin{minipage}[t]{.5\textwidth}
		\centering
		\caption{Nhà vệ sinh nữ}
		\begin{tabular}{|c|c|c|}
    	\hline
    	\multicolumn{1}{|c|}{\textbf{STT}} & \multicolumn{1}{c|}{\textbf{Tên thiết bị}} & \multicolumn{1}{c|}{\textbf{Số lượng}} \\
    	\hline
    	1     & Chậu rửa mặt & 3 \\
    	\hline
    	2     & Bồn cầu & 4 \\
    	\hline
    	\end{tabular}
  		\label{tab:tb_nvs_nữ}	
	\end{minipage}		
\end{table}
	
\begin{table}[H]
		\vspace{-0.5cm}
  		\centering
  		\caption{Tổng thiết bị có trong \textbf{54 nhà vệ sinh}}
    	\begin{tabular}{|c|c|c|}
    	\hline
    	\multicolumn{1}{|c|}{\textbf{STT}} & \multicolumn{1}{c|}{\textbf{Tên thiết bị}} & \multicolumn{1}{c|}{\textbf{Số lượng}} \\
    	\hline
    	1     & Bồn tiểu nam & 4 \\
    	\hline
    	2     & Chậu rửa mặt & 5 \\
    	\hline
    	3     & Bồn cầu & 8 \\
   	 	\hline
    	\end{tabular}%
  		\label{tab:total_equipment_toilet}
\end{table}
		
		-- Đối với công trình, hệ thống cấp nước sẽ được cung cấp gián tiếp từ nhà máy vì dù đây là building lớn, nằm ngay trung tâm thành phố Hà Nội - Rất ít xảy ra hiện tượng cúp nước. Nhưng không thể sử dụng áp lực từ chính đường ống nước để đẩy nước lên các tầng cao hơn được. Đồng thời, nếu xảy ra tình huống cúp nước thì toà nhà vẫn đảm bảo được một lượng nước để vận hành trong khoảng thời gian chờ khắc phục sự cố.
		
	\subsection{TỔNG QUAN VỀ HỆ THỐNG THOÁT NƯỚC}
	-- Đối với hệ thống thoát nước, ngoại trừ việc phải tính toán thoát nước cho các nhà vệ sinh thì còn phải tính tới cả thoát nước mưa, nước ngưng.
	
	-- Hệ thống thoát nước phải đảm bảo được các nguyên tắc sau đây:
	\begin{itemize}
		\setlength\itemsep{1mm}
		\item Hệ thống thoát nước phân và hệ thống thoát nước sàn riêng biệt. 
		\item Hệ thống ống đứng thoát nước mưa được bố trí trong các hộp gen thông tầng.
		\item Hệ thống đảm bảo thoát nước tốt. 
		\item Có tổng chiều dài ngắn nhất. 
		\item Dể dàng kiểm tra sữa chửa thay thế. 
		\item Tránh đi qua phòng khách, phòng ngủ. 
		\item Dễ phân biệt khi sửa chữa. 
		\item Thuận tiện trong quá trình thi công. 
	\end{itemize}
	\subsubsection{Hệ thống thoát nước nhà vệ sinh:}	
	-- Hệ thống thoát nước đầu tiên và dễ thấy nhất trong toà nhà đó chính là hệ thống thoát nước cho các nhà vệ sinh, sinh hoạt chung khác.
	
\begin{table}[H]
		\vspace{-0.5cm}
  		\centering
  		\caption{Tổng thiết bị có trong \textbf{54 nhà vệ sinh}}
    	\begin{tabular}{|c|c|c|}
    	\hline
    	\multicolumn{1}{|c|}{\textbf{STT}} & \multicolumn{1}{c|}{\textbf{Tên thiết bị}} & \multicolumn{1}{c|}{\textbf{Số lượng}} \\
    	\hline
    	1     & Bồn tiểu nam & 4 \\
    	\hline
    	2     & Chậu rửa mặt & 5 \\
    	\hline
    	3     & Bồn cầu & 8 \\
   	 	\hline
    	\end{tabular}%
  		\label{tab:tong_thoat_nvs}
\end{table}
	 
	\subsubsection{Vai trò của hệ thống thoát nước mưa:}
	\begin{itemize}
		\vspace{-2mm}
		\setlength\itemsep{1mm}
		\item Hệ thống thoát nước mưa trên mái sẽ làm cho nước mưa không tồn đọng lại trên mái và không bị thấm ngược vào trong nhà.
		\item Nếu thoát nước không tốt thì nước mưa sẽ thấm vào mái sẽ gây ra ẩm mốc làm ảnh hưởng kết cấu công trình cũng như thẩm mỹ công trình.
		\item Trong trường hợp không có hệ thống thoát nước mưa thì rác sẽ bị ùn ứ, đọng lâu ngày sinh ra mất vệ sinh làm cho các sinh vật sinh sôi như muỗi, ruồi, các loại vi khuẩn, nấm mốc gây bệnh.
	\end{itemize}
	
	$\ast$ Tuỳ từng loại mái của toà nhà mà sẽ có các hệ thống thoát nước mưa khác nhau, nhưng 2 loại mái thường gặp nhất là \emph{mái dốc} và \emph{mái bằng}.
	
	\subsubsection{Tại sao phải có hệ thống thoát nước ngưng:}
	-- Trong quá trình làm lạnh khí (ở dàn lạnh của điều hòa), hơi nước ngưng tụ và hoá lỏng, vì vậy phải có đường thoát nước từ dàn lạnh ra. Nhiều người không chú ý, thậm chí không biết đến vấn đề này nên khi lắp đặt điều hoà, không biết thoát nước đi đâu – nhất là khi dàn lạnh ở phía tường trong, không tiếp cận với hệ thống thoát nước nào hoặc đi ra ngoài mặt thoáng rất xa. Vì vậy các gia đình nên chú ý khi lắp điều hòa phải lắp đặt thêm ống thoát nước điều hòa.

-- Ống thoát nước điều hòa là một phần quan trọng nằm trong các loại vật tư khi lắp điều hòa, khi điều hòa hoạt động đặc biệt là mùa đông chạy chiều nóng, lượng nước thải trong một ngày đêm trung bình từ 6-12 lít tùy thuộc từng dòng và công suất điều hòa. Nước chảy qua ống thoát có độ lạnh khoảng 10-15 độ C do đó cần có các bộ phận dẫn hướng nước tránh rò rỉ, thoát ra sàn nhà, trần nhà.
	
	\vspace{0.5cm}$\pmb{\Rightarrow}$ Vì vậy hệ thống thoát nước toà nhà cũng là một yếu tố vô cùng quan trọng trong tính toán cấp thoát nước.
	
	\subsubsection{Tiêu chuẩn và quy chuẩn sẽ áp dụng cho hệ thống:}
	-- Hệ thống sẽ được tính toán dựa theo những \emph{tiêu chuẩn} và \emph{quy chuẩn} sau:
	\begin{itemize}
		\item[\textbf{1.}]Quy chuẩn QCVN 07-2:2016/BXD
		\item[\textbf{2.}]Tiêu chuẩn TCVN 4519 : 1988
		\item[\textbf{3.}]Tiêu chuẩn TCVN 5576 : 1991
	\end{itemize}
	\subsection{TỔNG QUAN VỀ HỆ THỐNG CHỮA CHÁY}	
	-- Hệ thống chữa cháy là tổng hợp các thiết bị kỹ thuật chuyên dùng, đường ống dẫn và các chất chữa cháy dùng để dập tắt đám cháy.
	
	-- Hệ thống chữa cháy vách tường là hệ thống chữa cháy được lắp đặt ở trên tường bên trong các công trình.
	
	-- Thiết bị chủ yếu trong hệ thống chữa cháy vách tường gồm: \emph{máy bơm nước chữa cháy}, \emph{đường ống cấp nước chữa cháy} và các phương tiện khác như \emph{van}, \emph{lăng phun nước}, \emph{cuộn vòi dẫn nước}…
	
	\subsubsection{Hệ thống chữa cháy tự động Sprinkler đường ống ướt:}
	-- Toà nhà sẽ sử dụng hệ thống chữa cháy tự động Sprinkler đường ống ướt. Hệ thống đường ống ướt là hệ thống sprinkler tiêu chuẩn thường xuyên nạp đầy nước có áp lực ở cả phía trên và phía dưới van báo động đường ống ướt.
	\subsubsection{Tiêu chuẩn và quy chuẩn sẽ áp dụng cho hệ thống:}
	-- Hệ thống sẽ được tính toán dựa theo những \emph{tiêu chuẩn} và \emph{quy chuẩn} sau:
	\begin{itemize}
		\item[\textbf{1.}]Tiêu chuẩn TCVN 6305 
		\item[\textbf{2.}]Tiêu chuẩn TCVN 7735 – 2003
		\item[\textbf{3.}]Tiêu chuẩn TCVN 5040 – 1990
		\item[\textbf{4.}]Tiêu chuẩn TCVN 5760 – 1993
	\end{itemize}
	
	\section{HỆ THỐNG ĐIỆN TRONG CÔNG TRÌNH}
	\subsection{GIỚI THIỆU VỀ HỆ THỐNG CHIẾU SÁNG}	
	-- Thiết kế hệ thống chiếu sáng trong toà nhà đòi hỏi sự hiểu biết về kỹ thuật điện, nguồn sáng và tầm nhìn, đồng thời cũng nhạy cảm với các vấn đề về kiến trúc và thẩm mỹ. Thiết kế cuối cùng cần đáp ứng nhu cầu trực quan cho mắt người thực hiện vô số công việc trong khi vẫn đáp ứng đc các hình dạng kiến trúc và môi trường ngay lập tức.
	
	-- Hệ thống chiếu sáng do đó cũng có thể xem là một trong những hệ thống thuộc diện ``sống còn'' của toà nhà mà đòi hỏi rất gắt gao ở việc thiết kế cũng như tính thẩm mỹ.
	
	-- Các nhà thiết kế về ánh sáng hiểu rằng hầu hết những người ở trong tòa nhà không nhất thiết muốn có đèn LED hoặc một chủng loại đèn bất kỳ nào - họ muốn thoải mái nhìn thấy những gì họ đang làm. Làm thế nào để cung cấp cho họ tầm nhìn này là vai trò của nhà thiết kế ánh sáng. Làm thế nào để cung cấp cho điều này trong khi vẫn thiết kế hài hòa với kiến trúc, tích hợp với ánh sáng ban ngày sẵn có, giảm thiểu việc sử dụng năng lượng xây dựng, phù hợp với quá trình xây dựng tổng thể và ngân sách là tất cả những việc phải làm của nhà thiết kế ánh sáng trong toàn bộ quá trình thiết kế tòa nhà.
	
	-- Một hệ thống chiếu sáng đủ tiêu chuẩn phải đáp ứng đc các tiêu chí sau:
	\begin{itemize}[leftmargin=2.5cm]
		\item Độ rọi chiếu sáng.
		\item Thiết kế không gian.
		\item Nhiệt độ màu.
		\item Điều kiện tiện nghi.
		\item Hệ thống điều khiển hợp lý có thể sử dụng điều khiển từ xa để tránh gây lãng phí điện năng và dể quản lý khâu bật tắt.
		\item Tính thẩm mỹ cao tôn lên vẻ đẹp không gian, sang trọng, hiện đại, phù hợp thời đại công nghệ mới.
	\end{itemize}
	\subsubsection{Tiêu chuẩn và quy chuẩn sẽ áp dụng cho hệ thống:}		
	-- Hệ thống sẽ được tính toán dựa theo những \emph{tiêu chuẩn} và \emph{quy chuẩn} sau:
	\begin{itemize}
		\item[\textbf{1.}]Tiêu chuẩn TCVN 7114 : 2008 
		\item[\textbf{2.}]Quy chuẩn QCXDVN 09 : 2005
	\end{itemize}	
	
	\section{HỆ THỐNG TỰ ĐỘNG HOÁ TRONG CÔNG TRÌNH}
	-- Sự phát triển bền vững của kinh tế, chính trị ở mỗi quốc gia trên thế giới làm cho nhu cầu đòi hỏi về vật chất, sự sang trọng tiện nghi và đảm bảo an ninh,an toàn trong cả nơi làm việc cũng như nhà ở ngày càng có nhu cầu cao hơn.

	-- Sự ra đời của các toà nhà, khách sạn, các trung tâm thương mại, các cao ốc văn phòng… với mức độ tự động hóa và bảo mật cao ngày càng nhiều hơn. Nhu cầu về nhân lực cũng như thiết bị vật tư, các giải pháp thiết kế và thi công cao. Đó là lĩnh vực có thể nghiên cứu đầu tư kinh doanh khả thi trong tương lai không xa.
	
	-- Trong toà nhà thông minh, đồ dùng trong nhà từ các phòng chức năng, các căn phòng làm việc, phòng ngủ, phòng khách đến toilet đều gắn các bộ điều khiển điện tử có thể kết nối với mạng Internet và điện thoại di động, cho phép chủ nhân có thể điều khiển tại chỗ, điều khiển vật dụng từ xa hoặc lập trình cho thiết bị ở nhà hoạt động tự động theo lịch với chương trình có sẵn.
	
	{\large $\pmb{\Rightarrow}$} Như vậy, toà nhà thông minh là một toà nhà có một hệ thống kỹ thuật hoàn hảo, được lập trình tối ưu hóa cho việc điều khiển, giám sát, vận hành thiết bị,vật dụng trong toà nhà.	
	\subsection{GIỚI THIỆU VỀ HỆ THỐNG BMS CỦA TOÀ NHÀ}
	-- Hệ thống BMS (Building Management System) là hệ thống đồng bộ cho phép điều khiển và quản lý mọi hệ thống kỹ thuật trong tòa nhà như \emph{hệ thống điện, hệ thống cung cấp nước sinh hoạt, điều hòa thông gió, cảnh báo môi trường, an ninh, báo cháy – chữa cháy,}… đảm bảo cho việc vận hành các thiết bị trong tòa nhà được chính xác, kịp thời, hiệu quả, tiết kiệm năng lượng và tiết kiệm chi phí vận hành. Hệ thống BMS là hệ thống đồng bộ mang tính thời gian thực, trực tuyến, đa phương tiện, nhiều người dung, hệ thống vi xử lý bao gồm các bộ vi xử lý trung tâm với tất cả các phần mềm và phần cứng máy tính, các thiết bị vào/ra, các bộ vi xử lý khu vực, các bộ cảm biến và điều khiển qua các ma trận điểm.
	
	\break
	\textbf{Hệ thống quản lý tòa nhà (hệ thống BMS) điều khiển và giám sát các hệ thống sau:}
\begin{multicols}{2}
	\begin{itemize}
		\item Trạm phân phối điện
		\item Máy phát điện dự phòng
		\item Hệ thống chiếu sáng
		\item Hệ thống điều hoà \& thông gió
		\item Hệ thống cấp nước sinh hoạt
		\item Hệ thống báo cháy
		\item Hệ thống chữa cháy
		\item Hệ thống thang máy
		\item Hệ thống âm thanh công cộng
		\item Hệ thống thẻ kiểm soát ra vào
		\item Hệ thống an ninh
		\item[\vspace{\fill}] %thêm vào để cho đủ item/2 đặng các cột có khoảng cách = nhau
	\end{itemize}
\end{multicols}
	
	\begin{figure}[H]
		\centering
		\includegraphics[width=0.9\textwidth]{hao_phuong_so_do_he_thong_bms.jpg}	
		\caption{Sơ đồ hệ thống BMS}
	\end{figure}

	\textbf{Tính năng của BMS}
\begin{itemize}
	\item Cho phép các tiện ích (thiết bị thông minh) trong tòa nhà hoạt động một cách đồng bộ, chính xác theo đúng yêu cầu của người điều hành
	
	\item Cho phép điều khiển các ứng dụng trong tòa nhà thông qua cáp điều khiển và giao thức mạng
	
	\item Kết nối các hệ thống kỹ thuật như an ninh, báo cháy… qua cổng giao diện mở của hệ thống với các ngôn ngữ giao diện theo tiêu chuẩn quốc tế
	
	\item Giám sát được môi trường không khí, môi trường làm việc của con người
	
	\item Tổng hợp, báo cáo thông tin
	
	\item Cảnh báo sự cố, đưa ra những tín hiệu cảnh báo kịp thời trước khi có những sự cố
	
	\item Quản lý dữ liệu gồm soạn thảo chương trình, quản lý cơ sở dữ liệu, chương trình soạn thảo đồ hoạ, lưu trữ và sao lưu dữ liệu
	
	\item Hệ thống BMS linh hoạt, có khả năng mở rộng với các giải pháp sẵn sàng đáp ứng với mọi yêu cầu
\end{itemize}

\begin{figure}[H]
	\centering
	\includegraphics[width=0.95\linewidth]{he_thong_bms_hao_phuong.png}	
	\caption{BMS tính năng}
\end{figure}
	
	\break
	\textbf{Lợi ích mang lại từ BMS}
	\begin{itemize}
		\item Đơn giản hóa và tự động hóa vận hành các thủ tục, chức năng có tính lặp đi lặp lại
		\item Quản lý tốt hơn các thiết bị trong tòa nhà nhờ hệ thống lưu trữ dữ liệu, chương trình bảo trì bảo dưỡng và hệ thống tự động báo cáo cảnh báo
		\item Giảm sự cố và phản ứng nhanh đối với các yêu cầu của khách hàng hay khi xảy ra sự cố
		\item Giảm chi phí năng lượng nhờ tính năng quản lý tập trung điều khiển và quản lý năng lượng
		\item Giảm chi phí nhân công và thời gian đào tạo nhân viên vận hành – cách sử dụng dễ hiểu, mô hình quản lý được thể hiện trực quan trên máy tính cho phép giảm tối đa chi phí dành cho nhân sự và đào tạo
		\item Dễ dàng nâng cấp, linh hoạt trong việc lập trình theo nhu cầu, kích thước, tổ chức và các yêu cầu mở rộng khác nhau
	\end{itemize}
	
	\textbf{Cấu trúc của hệ thống BMS gồm 4 phần:}
	\begin{enumerate}
		\item Phần mềm điều khiển trung tâm
		\item Thiết bị cấp quản lý
		\item Bộ điều khiển cấp trường
		\item Cảm biến và các thiết bị chấp hành
	\end{enumerate}

	\subsection{IoT VÀ TÁC ĐỘNG CỦA NÓ LÊN BMS}
	\subsubsection{IoT là gì?}
	-- Internet of Things, hay IoT, internet vạn vật là đề cập đến hàng tỷ thiết bị vật lý trên khắp thế giới hiện được kết nối với internet, thu thập và chia sẻ dữ liệu. Nhờ bộ xử lý giá rẻ và mạng không dây, có thể biến mọi thứ, từ viên thuốc sang máy bay, thành một phần của IoT. Điều này bổ sung sự “thông minh kỹ thuật số” cho các thiết bị, cho phép chúng giao tiếp mà không cần có con người tham gia và hợp nhất thế giới kỹ thuật số và vật lý.

	-- Một bóng đèn có thể được bật bằng ứng dụng điện thoại thông minh là một thiết bị IoT, như một cảm biến chuyển động hoặc một bộ điều chỉnh nhiệt thông minh trong văn phòng của bạn hoặc đèn đường được kết nối. Một thiết bị IoT có thể đơn giản như đồ chơi của trẻ em hoặc nghiêm trọng như một chiếc xe tải không người lái, hoặc phức tạp như một động cơ phản lực hiện chứa hàng ngàn cảm biến thu thập và truyền dữ liệu trở lại để đảm bảo nó hoạt động hiệu quả. Ở quy mô lớn hơn, các dự án thành phố thông minh đang được lấp đầy bằng các cảm biến để giúp chúng ta hiểu và kiểm soát môi trường.

	-- Thuật ngữ IoT chủ yếu được sử dụng cho các thiết bị thường không được mong đợi có kết nối internet và có thể giao tiếp với mạng độc lập với hành động của con người. Vì lý do này, PC thường không được coi là thiết bị IoT và cũng không phải là điện thoại thông minh – mặc dù thiết bị này được nhồi nhét bằng cảm biến. Tuy nhiên, một chiếc smartwatch hoặc một fitness band hoặc thiết bị đeo khác có thể được tính là một thiết bị IoT.
	
	\subsubsection{IoT thay đổi cách mà BMS vận hành thế nào?}
	-- Kết nối chính là điểm mạnh của IoT. Với IoT, các nhà sản xuất khác nhau đều có thể có mặt trong hệ thống BMS được nhờ vào các tiêu chuẩn giao thức như \emph{Ethernet/IP, XML, KNX, BACnet, Modbus và LonWorks}. 

	-- Khả năng kết nối được đó sẽ giúp hệ thống tiến đến thứ gọi là ``Đám mây''. Với các ``đám mây'' này thì việc điều khiển từ xa và khả năng tiên đoán lỗi hệ thống và giám sát có thể thực hiện từ khắp mọi nơi hoặc hơn nữa là xử lý và hiệu chỉnh tạm thời bằng các thuật toán trước khi nhân viên kỹ thuật xuống sửa chữa. Điều này làm tăng khả năng vận hành lâu dài của hệ thống.
	
	-- IoT cũng sẽ thay đổi mức giá của hệ thống BMS bởi các thành phần linh kiện của hệ thống giờ đây sẽ không còn độc quyền từ một hãng nữa mà sẽ là mức giá cạnh tranh tới từ nhiều hãng.
	\subsection{LÀM THẾ NÀO ĐỂ TIẾP CẬN IoT ĐÚNG CÁCH?}
	-- Internet Of Things viết tắt là IoT chính là internet trong mọi thứ. Và theo WikiPedia định nghĩa thì IoT chính là mạng lưới vạn vật kết nối Internet hoặc mạng lưới kết nối thiết bị Internet . Là một kịch bản của thế giới, khi mà mỗi đồ vật, con người được cung cấp một định danh riêng của nó và tất cả có khả năng truyền tải, trao đổi thông tin, dữ liệu qua một mạng duy nhất mà không cần đến sự tương tác trực tiếp giữa người với người, hay người với máy tính.
	
	-- Khi mà vạn vật đều có chung một mạng kết nối thì việc liên lạc và làm việc trở nên rất dễ dàng. Con người có thể hiện thực hóa mục đích của mình trong tương lai. Chúng ta hoàn toàn có thể kiểm soát mọi thứ. Giả sử 1 chiếc ví mà các bạn đang sử dụng có tích hợp công nghệ IoT. Chúng có nhiệm vụ kiểm tra số lượng tiền trong ví, kiểm tra ngày hết hạn của các giấy tờ mà các bạn để trong đó như: bảo hiểm y tế, hạn nộp học phí,.. và thông báo tình trạng của nó đến cho chúng ta biết thông qua các ứng dụng tin nhắn SMS, facebook, skype, zalo,…
	
	-- Hay như một hệ thống tưới nước tự động cây cối trong gia đình bạn được tích hợp công nghệ IoT. Giúp bạn điều khiển qui trình chăm sóc cây, tưới nước cây, thậm chí là bắt sâu bọ,… khi bạn có chuyến đi công tác xa vài ngày hay vài tháng mà không thể thực hiện được các chức năng đó. Điều đó sẽ trở nên rất đơn giải khi giả sử mà hệ thống tưới cây tự động và điện thoại hoặc laptop, PC,.. của bạn được kết nối và mạng lưới Internet và qua đó có thể trao đổi thông tin cũng như thực thi các câu lệnh mà bạn mong muốn.
	
	-- Điều đó thật mới mẻ và tiện dụng phải không nào? Chúng ta có thể tiết kiệm được rất nhiều thời gian cũng như tránh gặp phải những trường hợp khó khăn khi không làm chủ và quản lý được tất cả mọi vật xung quanh ta.
	\subsubsection{Hệ thống nhúng - Bước đầu tiên IoT}
	-- IoT là một hệ thống tự trị, để làm được điều này, nền tảng kiến thức của IoT bắt buộc phải xuất phát từ hệ thống nhúng. 
	
	-- Hệ thống nhúng thường có một số đặc điểm chung như sau:
\begin{itemize}
		\item Các hệ thống nhúng được thiết kế để thực hiện một số nhiệm vụ chuyên dụng chứ không phải đóng vai trò là các hệ thống máy tính đa chức năng. Một số hệ thống đòi hỏi ràng buộc về tính hoạt động thời gian thực để đảm bảo độ an toàn và tính ứng dụng; một số hệ thống không đòi hỏi hoặc ràng buộc chặt chẽ, cho phép đơn giản hóa hệ thống phần cứng để giảm thiểu chi phí sản xuất.
		
		\item Một hệ thống nhúng thường không phải là một khối riêng biệt mà là một hệ thống phức tạp nằm trong thiết bị mà nó điều khiển.
		
		\item Phần mềm được viết cho các hệ thống nhúng được gọi là firmware và được lưu trữ trong các chip bộ nhớ ROM hoặc bộ nhớ flash chứ không phải là trong một ổ đĩa. Phần mềm thường chạy với số tài nguyên phần cứng hạn chế: không có bàn phím, màn hình hoặc có nhưng với kích thước nhỏ, dung lượng bộ nhớ thấp.
\end{itemize}

\begin{figure}[H]
	\centering
	\includegraphics[width=0.6\linewidth]{embedded_system.jpg}
	\caption{Một board mạch nhúng}
\end{figure}

	-- Các hệ thống nhúng thường nằm trong các cỗ máy được kỳ vọng là sẽ chạy hàng năm trời liên tục mà không bị lỗi hoặc có thể khôi phục hệ thống khi gặp lỗi. Vì thế, các phần mềm hệ thống nhúng được phát triển và kiểm thử một cách cẩn thận hơn là phần mềm cho máy tính cá nhân. Ngoài ra, các thiết bị rời không đáng tin cậy như ổ đĩa, công tắc hoặc nút bấm thường bị hạn chế sử dụng. Việc khôi phục hệ thống khi gặp lỗi có thể được thực hiện bằng cách sử dụng các kỹ thuật như watchdog timer – nếu phần mềm không đều đặn nhận được các tín hiệu watchdog định kì thì hệ thống sẽ bị khởi động lại.
	
	-- Một số vấn đề cụ thể về độ tin cậy \& hướng giải quyết cho hệ thống nhúng như:
\begin{itemize}
		\item Hệ thống không thể ngừng để sửa chữa một cách an toàn, ví dụ như ở các hệ thống không gian, hệ thống dây cáp dưới đáy biển, các đèn hiệu dẫn đường,... Giải pháp đưa ra là chuyển sang sử dụng các hệ thống con dự trữ hoặc các phần mềm cung cấp một phần chức năng.
		
		\item Hệ thống phải được chạy liên tục vì tính an toàn, ví dụ như các thiết bị dẫn đường máy bay, thiết bị kiểm soát độ an toàn trong các nhà máy hóa chất,... Giải pháp đưa ra là lựa chọn backup hệ thống.
		
		\item Nếu hệ thống ngừng hoạt động sẽ gây tổn thất rất nhiều tiền của ví dụ như các dịch vụ buôn bán tự động, hệ thống chuyển tiền, hệ thống kiểm soát trong các nhà máy...
\end{itemize}

	-- Hệ thống nhúng tương tác với thế giới bên ngoài với nhiều cách:
\begin{itemize}
	\item Cảm nhận môi trường: cảm biến nhiệt độ, độ ẩm, ánh sáng, trọng lượng…, cảm nhận bằng tín hiệu điện (máy dò nhiễu điện từ)
	
	\item Tác động trở lại môi trường (hú còi báo động khi phát hiện khói trong tòa nhà…)
	
	\item Tốc độ tương tác phải đáp ứng thời gian thực (hệ thống còi báo hỏa, hệ thống chống trộm trên ô tô,…)
	
	\item Có thể có hoặc không có giao diện giao tiếp với người dùng như máy tính cá nhân. Với những hệ thống đơn giản, thiết bị nhúng sử dụng LCD nhỏ, Joystick, LED, nút bấm, chỉ thị chữ hoặc số và thường đi kèm với một menu đơn giản. Hiện nay chúng ta cũng có thể kết nối đến hệ thống nhúng thông qua giao diện Web, việc này cho phép giảm thiểu chi phí cho màn hình nhưng vẫn cung cấp khả năng hiển thị và nhập liệu thuận tiện thuận tiện thông qua mạng và máy tính khác.
\end{itemize}	
	
	\subsubsection{Hệ thống nhúng \& ngôn ngữ C/C++}
	-- Thiết kế hệ thống nhúng đòi hỏi 1 quá trình phức tạp rất nhiều khâu cũng như quá trình. Nếu phát triển một cách đầy đủ thì sẽ tốn rất nhiều thời gian, công sức cũng như đòi hỏi kiến thức chuyên môn sâu. Nên ở giới hạn trong tập luận văn này, chúng ta sẽ chỉ sử dụng những board mạch có sẵn trên thị trường \textit{(như STM32 hoặc Arduino hoặc ESP8266,...)} để sử dụng \& tạo model thử nghiệm, kiểm chứng.
	
	-- Ngôn ngữ mà chúng ta sẽ sử dụng chính là C/C++ vì đây là loại ngôn ngữ thường được sử dụng trong các hệ thống nhúng do tính chất vừa là ngôn ngữ lập trình bậc cao vừa có khả năng tiếp cận các thanh ghi. 
	
	-- Những khó khăn về việc triển khai hệ thống IoT ở ngoài thực tiễn sẽ được đề cập tại chương IoT cho HVAC \footnote{CHƯƠNG 10}.

\begin{figure}[H]
\begin{subfigure}{0.5\textwidth}
\begin{center}
	\includegraphics[width=0.75\linewidth]{esp8266.jpg} 
	\caption{Board ESP8266}
\end{center}
\end{subfigure}
\begin{subfigure}{0.5\textwidth}
\begin{center}
		\includegraphics[width=0.75\linewidth]{stm32f407_20discovery.jpg}
	\caption{Board STM32F407}
\end{center}
\end{subfigure}
\caption{Các loại kit có mặt trên thị trường}	
\end{figure}

	
	%Trang bìa
\chapmoi{TÍNH TẢI LẠNH BẰNG PHƯƠNG PHÁP \textit{CARRIER}}

%Phần nội dung
\section{THÔNG SỐ THIẾT KẾ}
\subsection{NHIỆT ĐỘ BÊN NGOÀI}
-- Các thông số bên ngoài toà nhà được thể hiện trong bảng dưới đây:
\begin{table}[H]
	\vspace{-0.3cm}
	\centering
	\caption{Thông số ngoài trời}
	\begin{tabular}{|l|r|}
		\hline
		Nhiệt độ bên ngoài -- t{\scriptsize N} & \textcolor[rgb]{ 1,  0,  0}{\textbf{32} $ ^{\circ} $C} \bigstrut\\
		\hline
		Độ ẩm tương đối -- $\varphi$ & \textcolor[rgb]{ 1,  0,  0}{\textbf{75}\%} \bigstrut\\
		\hline
		Nhiệt độ bầu ướt -- t{\scriptsize ư} & \textcolor[rgb]{ 1,  0,  0}{\textbf{18.76} $ ^{\circ} $C} \bigstrut\\
		\hline
	\end{tabular}
	\label{b:tsnt}
\end{table}

\subsection{NHIỆT ĐỘ BÊN TRONG}
-- Các thông số yêu cầu bên trong toà nhà được thể hiện trong bảng dưới đây:
\begin{table}[H]
	\vspace{-0.3cm}
	\centering
	\caption{Thông số trong nhà}
	\begin{tabular}{|l|r|}
		\hline
		Nhiệt độ trong nhà -- t{\scriptsize N} & \textcolor[rgb]{ 1,  0,  0}{\textbf{25} $ ^{\circ} $C} \bigstrut\\
		\hline
		Độ ẩm tương đối -- $\varphi$ & \textcolor[rgb]{ 1,  0,  0}{\textbf{55}\%} \bigstrut\\
		\hline
	\end{tabular}
	\label{b:tstn}
\end{table}

\section{TÍNH TOÁN TẢI LẠNH}
\subsection{NHIỆT HIỆN BỨC XẠ QUA KÍNH --- Q{\scriptsize 1}}
\subsubsection{Xác định Q{\scriptsize 1}}
Với điều kiện nhiệt độ và độ ẩm của không khí bên ngoài như đã chọn để thiết kế điều hoà không khí cho tòa nhà: t$_{T}$ = 32$^{\circ}$C; {\large $\varphi$}$_{N}$ = 75\%.

Tra đồ thị t – d ta được nhiệt độ đọng sương: {\large t}{\scriptsize đs} = 28.2619$^{\circ}$C.

Từ đó ta xác định được:
\begin{itemize}
	\item {\Large $\varepsilon$}{\scriptsize đs} $= 1 -\dfrac{t_{s} - 20}{10}\times0.13 = 1 -\dfrac{28.2619 - 20}{10}\times0.13 = 0.909$
	
	\item {\Large $\varepsilon$}{\scriptsize c} $= 1$ (do Tp.Hà Nội có độ cao gần mực nước biển).
	
	\item {\Large $\varepsilon$}{\scriptsize mm} $= 1$ (tính vào lúc bức xạ mặt trời lớn nhất).
	
	\item {\Large $\varepsilon$}{\scriptsize kh} $= 1.17$ (khung kính kim loại).
	
	\item {\Large $\varepsilon$}{\scriptsize m} $= 0.7$ (dùng kính Calorex màu xanh).
	
	\item {\Large $\varepsilon$}{\scriptsize r} $= 0.56$ (hệ số mặt trời kể đến ảnh hưởng của kính khác kính cơ bản khi có màn che bên trong).
\end{itemize}

Toàn bộ tòa nhà sử dụng loại kính Calorex, màu xanh, dày 6 mm và có sử dụng màn che màu sáng.
\begin{table}[H]
	\vspace{-0.3cm}
	\centering
	\caption{Các hệ số kính và màn che}
	\begin{adjustbox}{width=\textwidth}
		\begin{tabular}{|c|c|c|c|c|c|}
			\hline
			& \textbf{Hệ số hấp thụ} & \textbf{Hệ số phản xạ} & \textbf{Hệ số xuyên qua} & \textbf{Hệ số kính} & \textbf{Hệ số mặt trời} \bigstrut\\
			\hline
			\multirow{2}[2]{*}{Kính Carolex màu xanh} & \multirow{2}[2]{*}{{\large $\alpha$}$_{K}$ = 0.75} & \multirow{2}[2]{*}{{\large $\beta$}$_{K}$ = 0.05} & \multirow{2}[2]{*}{{\large $\tau$}$_{K}$ = 0.20} & \multirow{2}[2]{*}{{\large $\varepsilon$}$_{m}$ = 0.57} & \multirow{2}[2]{*}{} \bigstrut[t]\\
			&       &       &       &       &  \bigstrut[b]\\
			\hline
			\multirow{2}[2]{*}{Màn che màu sáng} & \multirow{2}[2]{*}{{\large $\alpha$}$_{m}$ = 0.37} & \multirow{2}[2]{*}{{\large $\beta$}$_{m}$ = 0.51} & \multirow{2}[2]{*}{{\large $\tau$}$_{m}$ = 0.12} & \multirow{2}[2]{*}{} & \multirow{2}[2]{*}{{\large $\varepsilon$}$_{r}$ = 0.56} \bigstrut[t]\\
			&       &       &       &       &  \bigstrut[b]\\
			\hline
		\end{tabular}%
		\label{tab:adadlabel}%
	\end{adjustbox}
\end{table}%

Nhiệt bức xạ mặt trời qua cửa kính khác kính cơ bản vào phòng theo các hướng của các phòng trong tòa nhà:
\begin{equation*}
	\begin{split}
		R_{K} &= [0.4\alpha_{K} + \tau_{K}(\alpha_{m} + \tau_{m} + \rho_{K}\rho_{m} + 0.4\alpha_{K}\alpha_{m})]R_{N} \\
		&= [0.4\alpha_{K} + \tau_{K}(\alpha_{m} + \tau_{m} + \rho_{K}\rho_{m} + 0.4\alpha_{K}\alpha_{m})]\dfrac{R}{0.88} \\
		&= [0.4\times0.75 + 0.2\times(0.37 + 0.12 + 0.05\times0.51 + 0.4\times0.75\times0.37)]\dfrac{R}{0.88} \\
		&= 0.4833\times R
	\end{split}
\end{equation*}
\begin{table}[H]
	\vspace{-0.3cm}
	\centering
	\caption{Bức xạ Mặt Trời qua kính lớn nhất theo hướng}
	\begin{tabular}{|c|c|c|}
		\hline
		\textbf{Hướng kính} & \textbf{RTmax ( W/m2 )} & \textbf{Rk ( W/m2 )} \bigstrut\\
		\hline
		Đông  & 520   & 251.32 \bigstrut\\
		\hline
		Tây   & 520   & 251.32 \bigstrut\\
		\hline
		Nam   & 470   & 227.15 \bigstrut\\
		\hline
		Bắc   & 82    & 39.63 \bigstrut\\
		\hline
	\end{tabular}
	\label{b:bxmtln}
\end{table}
Vậy: $Q_{1} = n_{t}\times F_{1}\times R_{K}\times $

\subsubsection{Hệ số tác dụng tức thời n{\scriptsize t}}
Để xác định hệ số tác dụng tức thời, ta phải xác định tổng khối lượng của các bề mặt tạo nên không gian điều hoà tính trên 1m$^2$:

\begin{center}
	G $= \rho \times \delta \times F$
\end{center}

Trong đó:
\begin{itemize}
	\item $F$ là diện tích tường, đơn vị m$^2$.
	\item $\rho$ là khối lượng riêng của vật liệu, đơn vị kg/m$^3$.
	\item $\delta$ là bề dày của vật liệu khảo sát, đơn vị m.
\end{itemize}

Cấu trúc kết cấu bao che như bảng dưới đây:

\begin{wraptable}[13]{l}{0.4\textwidth}
	\caption{Kết cấu bao che}
	\begin{tabular}{|r|r|r|r|r|}
		\hline
		\multicolumn{4}{|c|}{\textbf{Tường dày 200mm}} \bigstrut\\
		\hline
		\textbf{{\large $\delta_{g}$}} & 180 mm & {\large $\rho_{g}$} & 1800 kg/m$^3$ \bigstrut\\
		\hline
		{\large $\delta_{v}$} & 10 mm & \textbf{{\large $\rho_{v}$}} & 1800 kg/m$^3$ \bigstrut\\
		\hline
		\multicolumn{4}{|c|}{\textbf{Sàn}} \bigstrut\\
		\hline
		\textbf{{\large $\delta_{bt}$}} & 250 mm & {\large $\rho_{bt}$} & 2400 kg/m$^3$ \bigstrut\\
		\hline
		\textbf{{\large $\delta_{v}$}} & 50 mm & \textbf{{\large $\rho_{v}$}} & 1800 kg/m$^3$ \bigstrut\\
		\hline
		\multicolumn{4}{|c|}{\textbf{Trần}} \bigstrut\\
		\hline
		\textbf{{\large $\delta$}} & 50 mm & \textbf{{\large $\rho$}} & 1000 kg/m$^3$ \bigstrut\\
		\hline
	\end{tabular}
	\label{b:kcbc}%
\end{wraptable}%

\vspace{0.5cm}
Khối lượng bình quân của kết cấu bao che được xác định như sau:
\begin{equation*}
	g_{s} = \dfrac{G^{'}+0.5G^{''}}{F_{s}}, kg/m^2 
\end{equation*}

Trong đó:
\begin{itemize}
	\item $G^{'}$ -- khối lượng tường có mặt ngoài tiếp xúc với bức xạ mặt trời và của sàn nằm trên mặt đất, đơn vị là kg.
	\item $G^{''}$ -- khối lượng tường có mặt ngoài không tiếp xúc với bức xạ mặt trời và của sàn không nằm trên mặt đất, đơn vị là kg.
	\item $F_{s}$ -- diện tích sàn, đơn vị là m$^2$.
\end{itemize}

Bảng ở trang sau đây thể hiện khối lượng bình quân của kết cấu bao che:

\begin{landscape}
\begin{table}[H]
	\centering
	\begin{tabular}{|c|l|r|r|r|r|r|r|r|}
		\hline
		\multicolumn{1}{|l|}{\textbf{ TẦNG}} & \multicolumn{1}{c|}{\textbf{TÊN PHÒNG}} & \multicolumn{1}{p{4.93em}|}{\textbf{DIỆN TÍCH (m²) }} & \multicolumn{1}{p{4.855em}|}{\textbf{Fkinh}} & \multicolumn{1}{p{5.215em}|}{\textbf{Fng (m2)}} & \multicolumn{1}{p{5.07em}|}{\textbf{Ftr (m2)}} & \multicolumn{1}{c|}{\textbf{G'}} & \multicolumn{1}{c|}{\textbf{G''}} & \multicolumn{1}{p{4.07em}|}{\textbf{gs}} \bigstrut\\
		\hline
		&          &          &          &          &          &          &          & \multicolumn{1}{p{4.07em}|}{(kg/m2)} \bigstrut\\
		\hline
		&          &          &          & \multicolumn{1}{p{5.215em}|}{200 mm} & \multicolumn{1}{p{5.07em}|}{200 mm} &          &          &  \bigstrut\\
		\hline
		\multirow{4}[8]{*}{tầng 1} & sảnh văn phòng & 510      & 86       & 0        & 310      & 0        & 457680   & 449 \bigstrut\\
		\cline{2-9}         & coffe    & 718      & 123      & 0        & 139      & 0        & 542536   & 378 \bigstrut\\
		\cline{2-9}         & phòng máy & 58       & 0        & 38       & 100      & 12897    & 74165    & 862 \bigstrut\\
		\cline{2-9}         & phòng cấp cứu & 58       & 0        & 38       & 100      & 12897    & 74165    & 862 \bigstrut\\
		\hline
		\multirow{2}[4]{*}{tầng M} & phòng điều khiển 1 & 267      & 0        & 153      & 31       & 52220    & 194735   & 561 \bigstrut\\
		\cline{2-9}         & phòng điều khiển 2 & 663      & 0        & 256      & 97       & 87569    & 490558   & 502 \bigstrut\\
		\hline
		\multirow{6}[12]{*}{tầng 2 - 3} & cửa hàng 1 & 218      & 120      & 0        & 226      & 0        & 227930   & 522 \bigstrut\\
		\cline{2-9}         & cửa hàng 2 & 167      & 92       & 0        & 198      & 0        & 183349   & 547 \bigstrut\\
		\cline{2-9}         & cửa hàng 3 & 166      & 144      & 0        & 144      & 0        & 164074   & 493 \bigstrut\\
		\cline{2-9}         & cửa hàng 4 & 166      & 144      & 0        & 144      & 0        & 164074   & 493 \bigstrut\\
		\cline{2-9}         & cửa hàng 5 & 167      & 92       & 0        & 198      & 0        & 183349   & 547 \bigstrut\\
		\cline{2-9}         & cửa hàng 6 & 218      & 120      & 0        & 226      & 0        & 227930   & 522 \bigstrut\\
		\hline
		\multirow{2}[4]{*}{tầng 4} & khối văn phòng 1 & 725      & 416      & 0        & 375      & 0        & 628214   & 433 \bigstrut\\
		\cline{2-9}         & khối văn phòng 2 & 725      & 416      & 0        & 375      & 0        & 628214   & 433 \bigstrut\\
		\hline
		tầng 5 - 27 & khối văn phòng & 2041     & 783      & 0        & 271      & 0        & 1500889  & 368 \bigstrut\\
		\hline
	\end{tabular}%
	\caption{Khối lượng bình quân của kết cấu bao che}
	\label{b:klkcbc}%
\end{table}%
\end{landscape}

\begin{table}[H]
	\centering
	\caption{Nhiệt do bức xạ Mặt Trời}
	\begin{adjustbox}{width=\textwidth}
	\begin{tabular}{|c|r|r|l|r|r|r|r|}
		\hline
		\textbf{TẦNG} & \multicolumn{1}{c|}{\textbf{TÊN PHÒNG}} & \multicolumn{1}{c|}{\textbf{DIỆN TÍCH (m²) }} & \multicolumn{1}{c|}{\textbf{HƯỚNG}} & \multicolumn{1}{c|}{\textbf{gs}} & \multicolumn{1}{c|}{\textbf{Rk}} & \multicolumn{1}{c|}{\textbf{nt}} & \multicolumn{1}{c|}{\textbf{Q1}} \bigstrut\\
		\hline
		\multirow{4}[8]{*}{\textbf{Tầng 1}} & \multicolumn{1}{l|}{sảnh văn phòng} & 86    & B     & 449.0771 & 39.6306 & 0.91  & 1052.837 \bigstrut\\
		\cline{2-8}          & \multicolumn{1}{l|}{coffe} & 123   & Đ     & 378.047 & 251.316 & 0.8   & 8394.713 \bigstrut\\
		\cline{2-8}          & \multicolumn{1}{l|}{phòng máy} & 0     & N     & 862.491 & 227.151 & 0.67  & 0 \bigstrut\\
		\cline{2-8}          & \multicolumn{1}{l|}{phòng cấp cứu} & 0     & N     & 862.491 & 227.151 & 0.67  & 0 \bigstrut\\
		\hline
		\multirow{2}[4]{*}{\textbf{Tầng M}} & \multicolumn{1}{l|}{phòng điều khiển 1} & 0     & Đ     & 560.6694 & 251.316 & 0.65  & 0 \bigstrut\\
		\cline{2-8}          & \multicolumn{1}{l|}{phòng điều khiển 2} & 0     & N     & 502.1202 & 227.151 & 0.71  & 0 \bigstrut\\
		\hline
		\multirow{6}[12]{*}{\textbf{Tầng 2 - 3}} & \multicolumn{1}{l|}{cửa hàng 1} & 119.6478 & N     & 522.0778 & 227.151 & 0.71  & 6550.408 \bigstrut\\
		\cline{2-8}          & \multicolumn{1}{l|}{cửa hàng 2} & 91.8  & N     & 547.3625 & 227.151 & 0.71  & 5025.813 \bigstrut\\
		\cline{2-8}          & \multicolumn{1}{l|}{cửa hàng 3} & 144.3258 & N     & 493.4472 & 227.151 & 0.71  & 7901.464 \bigstrut\\
		\cline{2-8}          & \multicolumn{1}{l|}{cửa hàng 4} & 144.3258 & B     & 493.4472 & 39.6306 & 0.91  & 1766.878 \bigstrut\\
		\cline{2-8}          & \multicolumn{1}{l|}{cửa hàng 5} & 91.8  & B     & 547.3625 & 39.6306 & 0.91  & 1123.842 \bigstrut\\
		\cline{2-8}          & \multicolumn{1}{l|}{cửa hàng 6} & 119.6478 & B     & 522.0778 & 39.6306 & 0.91  & 1464.763 \bigstrut\\
		\hline
		\multirow{2}[4]{*}{\textbf{Tầng 4}} & \multicolumn{1}{l|}{khối văn phòng 1} & 416   & B     & 433.3636 & 39.6306 & 0.91  & 5092.793 \bigstrut\\
		\cline{2-8}          & \multicolumn{1}{l|}{khối văn phòng 2} & 416   & N     & 433.3636 & 227.151 & 0.71  & 22774.92 \bigstrut\\
		\hline
		\multirow{4}[8]{*}{\textbf{Tầng 5 - 27}} & \multicolumn{1}{c|}{\multirow{4}[8]{*}{khối văn phòng}} & 120.4 & Đ     & 367.7243 & 251.316 & 0.8   & 188997.1 \bigstrut\\
		\cline{3-8}          &       & 120.4 & T     & 367.7243 & 251.316 & 0.85  & 200809.4 \bigstrut\\
		\cline{3-8}          &       & 271.2 & N     & 367.7243 & 227.151 & 0.88  & 423258.3 \bigstrut\\
		\cline{3-8}          &       & 271.2 & B     & 367.7243 & 39.6306 & 0.99  & 83075.69 \bigstrut\\
		\hline
		\textbf{Mái} &       & 2040.78 & Ngang &       & 382.7736 & 1     & 265172.7 \bigstrut\\
		\hline
	\end{tabular}%
\end{adjustbox}
	\label{b:ndbxmt}%
\end{table}%

\textbf{Vậy tổng nhiệt do bức xạ:} $Q_{1} = 1222.462(KW)$.

\subsection{NHIỆT TRUYỀN KẾT CẤU BAO CHE -- Q{\scriptsize 2}}
Nhiệt truyền qua vách Q2 gồm hai thành phần:

-- Nhiệt do bức xạ vào tường được bỏ qua trong quá trình tính toán.

-- Nhiệt do chênh nhiệt độ giữa không khí trong phòng và ngoài nhà.
\begin{equation*}
	\Delta t = t_{n} - t_{T}
\end{equation*}

Vậy: $Q_{2} = Q_{21} + Q_{22} + Q_{23} + Q_{24} + Q_{25}$, W 

Nhiệt truyền qua kết cấu bao che được xác định bằng công thức:
\begin{equation*}
	Q_{2X} = k_{2X}\times F_{2X}\times\Delta t
\end{equation*}

Trong đó:
\begin{itemize}
	\item $k_{2X}$ : Hệ số truyền nhiệt qua kế cấu bao che W/(m$^2$.K).
	\item $F_{2X}$ : Diện tích kết cấu bao che (m²).
	\item $\Delta t$ : Độ chênh lệch nhiệt độ với không gian không diều hòa ($^{\circ}C$).
\end{itemize}

\subsubsection{Nhiệt truyền qua tường -- Q{\scriptsize 21}}

Xác định hệ số tryền nhiệt qua vách tường:

-- Đối với tường ngoài dày 200 mm:
\begin{table}[H]
	\centering
	\begin{tabular}{|r|r|r|r|r|}
		\hline
		\multicolumn{4}{|c|}{\textbf{Tường ngoài dày 200mm}} \bigstrut\\
		\hline
		{\large $\delta_{v}$} & 10 mm & \textbf{{\large $\lambda_{v}$}} & 0.93 W/mK \bigstrut\\
		\hline
		\textbf{{\large $\delta_{g}$}} & 180 mm & {\large $\lambda_{g}$} & 0.81 W/mK \bigstrut\\
		\hline
		{\large $\delta_{v}$} & 10 mm & \textbf{{\large $\lambda_{v}$}} & 0.93 W/mK \bigstrut\\
		\hline
	\end{tabular}
\end{table}
\vspace{-0.5cm}
{\Large \begin{equation*}
	\begin{split}
	k_{21} &= \dfrac{1}{\frac{1}{\alpha_{N}} + \frac{\delta_{g}}{\lambda_{g}} + 2\times\frac{\delta_{v}}{\lambda_{v}} + \frac{1}{\alpha_{T}}}\\
	&=\dfrac{1}{\frac{1}{20} + \frac{0.18}{0.81} + 2\times\frac{0.01}{0.93} + \frac{1}{10}}\\
	&={\scriptstyle 2.54, W/m^2 K}
	\end{split}
\end{equation*}}

-- Đối với tường trong dày 200 mm:
\begin{table}[H]
	\centering
	\begin{tabular}{|r|r|r|r|r|}
		\hline
		\multicolumn{4}{|c|}{\textbf{Tường trong dày 200mm}} \bigstrut\\
		\hline
		{\large $\delta_{v}$} & 10 mm & \textbf{{\large $\lambda_{v}$}} & 0.93 W/mK \bigstrut\\
		\hline
		\textbf{{\large $\delta_{g}$}} & 180 mm & {\large $\lambda_{g}$} & 0.81 W/mK \bigstrut\\
		\hline
		{\large $\delta_{v}$} & 10 mm & \textbf{{\large $\lambda_{v}$}} & 0.93 W/mK \bigstrut\\
		\hline
	\end{tabular}
\end{table}
\vspace{-0.5cm}
{\Large \begin{equation*}
	\begin{split}
		k_{21} &= \dfrac{1}{\frac{1}{\alpha_{T}} + \frac{\delta_{g}}{\lambda_{g}} + 2\times\frac{\delta_{v}}{\lambda_{v}} + \frac{1}{\alpha_{T}}}\\
		&=\dfrac{1}{\frac{1}{10} + \frac{0.18}{0.81} + 2\times\frac{0.01}{0.93} + \frac{1}{10}}\\
		&={\scriptstyle 2.25,  W/m^2 K}
	\end{split}
\end{equation*}}

\begin{table}[H]
	\vspace{-1cm}
	\centering
	\caption{Nhiệt truyền qua tường}
	\begin{adjustbox}{width=\textwidth}
	\begin{tabular}{|c|l|r|r|r|r|r|}
		\hline
		\textbf{ TẦNG} & \multicolumn{1}{c|}{\textbf{TÊN PHÒNG}} & \multicolumn{1}{c|}{\textbf{K kính}} & \multicolumn{1}{p{6.145em}|}{\textbf{DIỆN TÍCH (m²) }} & \multicolumn{1}{c|}{\textbf{t$_{N}$ ($^{\circ}C$)}} & \multicolumn{1}{c|}{\textbf{t$_{N}$ ($^{\circ}C$)}} & \multicolumn{1}{c|}{\textbf{Q$_{21}$(W)}} \bigstrut\\
		\hline
		\multirow{4}[8]{*}{\textbf{Tầng 1}} & sảnh văn phòng & 2.57  & 86    & 32    & 25    & 1547.14 \bigstrut\\
		\cline{2-7}          & coffe & 2.57  & 123   & 32    & 25    & 2212.77 \bigstrut\\
		\cline{2-7}          & phòng máy & 2.57  & 0     & 32    & 25    & 0 \bigstrut\\
		\cline{2-7}          & phòng cấp cứu & 2.57  & 0     & 32    & 25    & 0 \bigstrut\\
		\hline
		\multirow{2}[4]{*}{\textbf{Tầng M}} & phòng điều khiển 1 & 2.57  & 0     & 32    & 25    & 0 \bigstrut\\
		\cline{2-7}          & phòng điều khiển 2 & 2.57  & 0     & 32    & 25    & 0 \bigstrut\\
		\hline
		\multirow{6}[12]{*}{\textbf{Tầng 2 - 3}} & cửa hàng 1 & 2.57  & 120   & 32    & 25    & 2152.464 \bigstrut\\
		\cline{2-7}          & cửa hàng 2 & 2.57  & 92    & 32    & 25    & 1651.482 \bigstrut\\
		\cline{2-7}          & cửa hàng 3 & 2.57  & 144   & 32    & 25    & 2596.421 \bigstrut\\
		\cline{2-7}          & cửa hàng 4 & 2.57  & 144   & 32    & 25    & 2596.421 \bigstrut\\
		\cline{2-7}          & cửa hàng 5 & 2.57  & 92    & 32    & 25    & 1651.482 \bigstrut\\
		\cline{2-7}          & cửa hàng 6 & 2.57  & 120   & 32    & 25    & 2152.464 \bigstrut\\
		\hline
		\multirow{2}[4]{*}{\textbf{Tầng 4}} & khối văn phòng 1 & 2.57  & 416   & 32    & 25    & 7483.84 \bigstrut\\
		\cline{2-7}          & khối văn phòng 2 & 2.57  & 416   & 32    & 25    & 7483.84 \bigstrut\\
		\hline
		\textbf{Tầng 5 - 27} & khối văn phòng & 2.57  & 783   & 32    & 25    & 14089.77 \bigstrut\\
		\hline
	\end{tabular}%
	\end{adjustbox}
	\label{b:ntqvt}%
\end{table}%

\textbf{Vậy tổng nhiệt lượng truyền qua vách tường:} Q$_{21} = 45618.09(W)$

\subsubsection{Nhiệt truyền qua cửa ra vào -- Q{\scriptsize 22}}

Vì các phòng trong tòa nhà có sử dụng cửa gỗ (dày 40mm). Tra hệ số truyền nhiệt theo bảng 4.12\footnote{Theo sách Thiết kế hệ thống điều hoà không khí - Nguyễn Đức Lợi}.

Ta có được:
\begin{equation*}
	K_{22} = 2.23, W/m^2K
\end{equation*}
\newpage
\begin{table}[H]
	\centering
	\caption{Nhiệt lượng truyền qua cửa ra vào}
	\begin{adjustbox}{width=\textwidth}
	\begin{tabular}{|c|l|r|r|r|r|r|}
		\hline
		\textbf{ TẦNG} & \multicolumn{1}{c|}{\textbf{TÊN PHÒNG}} & \multicolumn{1}{c|}{\textbf{K kính}} & \multicolumn{1}{p{6.145em}|}{\textbf{DIỆN TÍCH (m²) }} & \multicolumn{1}{c|}{\textbf{t$_{N}$ ($^{\circ}C$)}} & \multicolumn{1}{c|}{\textbf{t$_{N}$ ($^{\circ}C$)}} & \multicolumn{1}{c|}{\textbf{Q$_{22}$(W)}} \bigstrut\\
		\hline
		\multirow{4}[8]{*}{\textbf{Tầng 1}} & sảnh văn phòng & 2.57  & 7.92  & 32    & 25    & 142.4808 \bigstrut\\
		\cline{2-7}          & coffe & 2.57  & 11.88 & 32    & 25    & 213.7212 \bigstrut\\
		\cline{2-7}          & phòng máy & 2.57  & 3.96  & 32    & 25    & 71.2404 \bigstrut\\
		\cline{2-7}          & phòng cấp cứu & 2.57  & 3.96  & 32    & 25    & 71.2404 \bigstrut\\
		\hline
		\multirow{2}[4]{*}{\textbf{Tầng M}} & phòng điều khiển 1 & 2.57  & 3.96  & 32    & 25    & 71.2404 \bigstrut\\
		\cline{2-7}          & phòng điều khiển 2 & 2.57  & 3.96  & 32    & 25    & 71.2404 \bigstrut\\
		\hline
		\multirow{6}[12]{*}{\textbf{Tầng 2 - 3}} & cửa hàng 1 & 2.57  & 7.92  & 32    & 25    & 142.4808 \bigstrut\\
		\cline{2-7}          & cửa hàng 2 & 2.57  & 7.92  & 32    & 25    & 142.4808 \bigstrut\\
		\cline{2-7}          & cửa hàng 3 & 2.57  & 7.92  & 32    & 25    & 142.4808 \bigstrut\\
		\cline{2-7}          & cửa hàng 4 & 2.57  & 7.92  & 32    & 25    & 142.4808 \bigstrut\\
		\cline{2-7}          & cửa hàng 5 & 2.57  & 7.92  & 32    & 25    & 142.4808 \bigstrut\\
		\cline{2-7}          & cửa hàng 6 & 2.57  & 7.92  & 32    & 25    & 142.4808 \bigstrut\\
		\hline
		\multirow{2}[4]{*}{\textbf{Tầng 4}} & khối văn phòng 1 & 2.57  & 15.84 & 32    & 25    & 284.9616 \bigstrut\\
		\cline{2-7}          & khối văn phòng 2 & 2.57  & 15.84 & 32    & 25    & 284.9616 \bigstrut\\
		\hline
		\textbf{Tầng 5 - 27} & khối văn phòng & 2.57  & 31.68 & 32    & 25    & 13108.23 \bigstrut\\
		\hline
	\end{tabular}%
	\end{adjustbox}
	\label{b:ntqcrv}%
\end{table}%

\textbf{Vậy tổng nhiệt lượng truyền qua vách tường:} Q$_{22} = 15174.21(W)$

\subsubsection{Nhiệt truyền qua cửa chiếu sáng -- Q{\scriptsize 23}}

Vì công trình không có cửa chiếu sáng nên lượng nhiệt truyền qua cửa chiếu sáng có thể xem như bằng 0.

\subsubsection{Nhiệt truyền qua nền sàn -- Q{\scriptsize 24}}

Với kết cấu sàn tầng 1 (tiếp xúc với không gian không điều hòa) như sau:

\begin{table}[H]
	\centering
	\begin{tabular}{|r|r|r|r|r|}
		\hline
		\multicolumn{4}{|c|}{\textbf{Sàn}} \bigstrut\\
		\hline
		\textbf{{\large $\delta_{g}$}} & 250 mm & {\large $\lambda_{g}$} & 0.81 kg/mK \bigstrut\\
		\hline
		\textbf{{\large $\delta_{v}$}} & 50 mm & \textbf{{\large $\lambda_{v}$}} & 0.93 W/mK \bigstrut\\	
		\hline
	\end{tabular}
	\label{b:kcs}
\end{table}

{\Large \begin{equation*}
	\begin{split}
		k_{24} &= \dfrac{1}{\frac{1}{\alpha_{N}} + \frac{\delta_{g}}{\lambda_{g}} + 2\times\frac{\delta_{v}}{\lambda_{v}} + \frac{1}{\alpha_{T}}}\\
		&= \dfrac{1}{\frac{1}{20} + \frac{0.18}{0.81} + 2\times\frac{0.01}{0.93} + \frac{1}{10}}\\
		&={\scriptstyle 2.54, W/m^2 K}
	\end{split}
\end{equation*}}

Do các phòng ở tầng hầm 1 và tầng 1 có sàn tiếp xúc với tầng (Có không gian không điều hòa) nên ta phải tính toán giá trị Q$ _{24} $, còn các phòng từ tầng 2 đến tầng 27 có sàn tiếp xúc với phòng có điều hòa nên giá trị Q$ _{24} $ = 0.

\begin{table}[H]
	\centering
	\caption{Nhiệt truyền qua sàn}
	\begin{tabular}{|c|l|r|r|r|r|r|}
		\hline
		\textbf{ TẦNG} & \multicolumn{1}{c|}{\textbf{TÊN PHÒNG}} & \multicolumn{1}{c|}{\textbf{K$ _{24} $}} & \multicolumn{1}{p{6.145em}|}{\textbf{DIỆN TÍCH (m²) }} & \multicolumn{1}{c|}{\textbf{t$_{N}$ ($^{\circ}C$)}} & \multicolumn{1}{c|}{\textbf{t$_{T}$ ($^{\circ}C$)}} & \multicolumn{1}{c|}{\textbf{Q$ _{24} $(W)}} \bigstrut\\
		\hline
		\multirow{4}[8]{*}{\textbf{Tầng 1}} & sảnh văn phòng & 2.54  & 510   & 32    & 25    & 9059.698 \bigstrut\\
		\cline{2-7}          & coffe & 2.54  & 718   & 32    & 25    & 12757.18 \bigstrut\\
		\cline{2-7}          & phòng máy & 2.54  & 58    & 32    & 25    & 1030.24 \bigstrut\\
		\cline{2-7}          & phòng cấp cứu & 2.54  & 58    & 32    & 25    & 1030.24 \bigstrut\\
		\hline
	\end{tabular}%
	\label{tab:addlabel}%
\end{table}%
\textbf{Vậy tổng nhiệt lượng truyền qua vách tường:} Q$_{24} = 23877.36(W)$

\subsubsection{Nhiệt truyền qua mái -- Q{\scriptsize 25}:}

Các phòng ở tầng 27 có trần tiếp xúc phòng không điều hòa.
\begin{equation*}
	Q_{25} = k_{25}\times F\times \Delta t
\end{equation*}
\begin{table}[H]
	\centering
	\begin{tabular}{|r|r|r|r|r|}
		\hline
		\multicolumn{4}{|c|}{\textbf{Trần}} \bigstrut\\
		\hline
		\textbf{{\large $\delta_{g}$}} & 250 mm & {\large $\lambda_{g}$} & 0.81 kg/mK \bigstrut\\
		\hline
		\textbf{{\large $\delta_{v}$}} & 50 mm & \textbf{{\large $\lambda_{v}$}} & 0.93 W/mK \bigstrut\\	
		\hline
	\end{tabular}
	\label{b:kct}
\end{table}
\begin{table}[H]
	\centering
	\caption{Nhiệt truyền qua mái}
	\begin{tabular}{|c|l|r|r|r|r|r|}
		\hline
		\textbf{ TẦNG} & \multicolumn{1}{c|}{\textbf{TÊN PHÒNG}} & \multicolumn{1}{c|}{\textbf{K$ _{25} $}} & \multicolumn{1}{p{6.145em}|}{\textbf{DIỆN TÍCH (m²) }} & \multicolumn{1}{c|}{\textbf{t$_{N}$ ($^{\circ}C$)}} & \multicolumn{1}{c|}{\textbf{t$_{T}$ ($^{\circ}C$)}} & \multicolumn{1}{c|}{\textbf{Q$ _{25} $(W)}} \bigstrut\\
		\hline
		\textbf{Tầng 27} & khối văn phòng & 2.54     & 2041     & 32       & 25       & 36282.6 \bigstrut\\
		\hline
	\end{tabular}%
	\label{b:ntqm}%
\end{table}%
\textbf{Vậy tổng nhiệt lượng truyền qua vách tường:} Q$_{25} = 36282.6(W)$

\subsubsection{NHIỆT TOẢ DO ĐÈN \& CÁC THIẾT BỊ KHÁC -- Q{\scriptsize 3}}

Do chưa biết chính xác có bao nhiêu bòng đèn chiếu sáng thì ta lấy mật độ tải chiếu sáng theo tài liệu.

-- Tải do chiếu sáng lấy 11 W/m$^2$.

-- Tải do thiết bị điện lấy 25 W/m$^2$.

Ta có công thức:
\begin{equation*}
	Q = A\times F
\end{equation*}

Trong đó:
\begin{itemize}
	\item F là diện tích sàn, đơn vị m$^2$.
	\item A là mật độ tải do chiếu sáng /mật độ tải thiết bị.
\end{itemize}

\begin{table}[H]
	\centering
	\caption{Tải do chiếu sáng}
	\begin{adjustbox}{width=\textwidth, height=.2\textheight}
	\begin{tabular}{|c|l|r|r|r|r|r|}
		\hline
		\textbf{SỐ TẦNG} & \multicolumn{1}{c|}{\textbf{TÊN PHÒNG}} & \multicolumn{1}{c|}{\textbf{DIỆN TÍCH (m²) }} & \multicolumn{1}{c|}{\textbf{A (W/m²)}} & \multicolumn{1}{c|}{\textbf{n$ _{t} $}} & \multicolumn{1}{c|}{\textbf{n{\scriptsize đ}}} & \multicolumn{1}{c|}{\textbf{Q$_{3}$ (W)}} \bigstrut\\
		\hline
		\multirow{4}[8]{*}{\textbf{Tầng 1}} & sảnh văn phòng & 509.579  & 1        & 0.87     & 0.85     & 376.8337 \bigstrut\\
		\cline{2-7}             & coffe    & 717.5506 & 1        & 0.84     & 0.85     & 512.3311 \bigstrut\\
		\cline{2-7}             & phòng máy & 57.9477  & 1        & 0.87     & 0.85     & 42.85232 \bigstrut\\
		\cline{2-7}             & phòng cấp cứu & 57.9477  & 1        & 0.87     & 0.85     & 42.85232 \bigstrut\\
		\hline
		\multirow{2}[4]{*}{\textbf{Tầng M}} & phòng điều khiển 1 & 266.8014 & 1        & 0.87     & 0.85     & 197.2996 \bigstrut\\
		\cline{2-7}             & phòng điều khiển 2 & 662.885  & 1        & 0.87     & 0.85     & 490.2035 \bigstrut\\
		\hline
		\multirow{6}[12]{*}{\textbf{Tầng 2 - 3}} & cửa hàng 1 & 218.2908 & 1        & 0.87     & 0.85     & 161.426 \bigstrut\\
		\cline{2-7}             & cửa hàng 2 & 167.484  & 1        & 0.87     & 0.85     & 123.8544 \bigstrut\\
		\cline{2-7}             & cửa hàng 3 & 166.2525 & 1        & 0.87     & 0.85     & 122.9437 \bigstrut\\
		\cline{2-7}             & cửa hàng 4 & 166.2525 & 1        & 0.87     & 0.85     & 122.9437 \bigstrut\\
		\cline{2-7}             & cửa hàng 5 & 167.484  & 1        & 0.87     & 0.85     & 123.8544 \bigstrut\\
		\cline{2-7}             & cửa hàng 6 & 218.2908 & 1        & 0.87     & 0.85     & 161.426 \bigstrut\\
		\hline
		\multirow{2}[4]{*}{\textbf{Tầng 4}} & khối văn phòng 1 & 724.812  & 1        & 0.84     & 0.85     & 517.5158 \bigstrut\\
		\cline{2-7}             & khối văn phòng 2 & 724.812  & 1        & 0.84     & 0.85     & 517.5158 \bigstrut\\
		\hline
		\textbf{Tầng 5 - 27} & khối văn phòng & 2040.78  & 1        & 0.84     & 0.85     & 33513.69 \bigstrut\\
		\hline
	\end{tabular}%
	\end{adjustbox}
	\label{b:tcs}%
\end{table}%
\textbf{Vậy tổng nhiệt lượng do chiếu sáng:} Q$_{3} = 37028(W)$

\begin{table}[H]
	\centering
	\caption{Tải do thiết bị}
	\begin{tabular}{|c|l|r|r|r|}
		\hline
		\textbf{SỐ TẦNG} & \multicolumn{1}{c|}{\textbf{TÊN PHÒNG}} & \multicolumn{1}{c|}{\textbf{DIỆN TÍCH (m²) }} & \multicolumn{1}{c|}{\textbf{A (W/m²)}} & \multicolumn{1}{c|}{\textbf{Q3 (w)}} \bigstrut\\
		\hline
		\multirow{4}[8]{*}{\textbf{Tầng 1}} & sảnh văn phòng & 509.579  & 25       & 12739.47 \bigstrut\\
		\cline{2-5}             & coffe    & 717.5506 & 25       & 17938.76 \bigstrut\\
		\cline{2-5}             & phòng máy & 57.9477  & 25       & 1448.693 \bigstrut\\
		\cline{2-5}             & phòng cấp cứu & 57.9477  & 25       & 1448.693 \bigstrut\\
		\hline
		\multirow{2}[4]{*}{\textbf{Tầng M}} & phòng điều khiển 1 & 266.8014 & 25       & 6670.035 \bigstrut\\
		\cline{2-5}             & phòng điều khiển 2 & 662.885  & 25       & 16572.13 \bigstrut\\
		\hline
		\multirow{6}[12]{*}{\textbf{Tầng 2 - 3}} & cửa hàng 1 & 218.2908 & 25       & 5457.269 \bigstrut\\
		\cline{2-5}             & cửa hàng 2 & 167.484  & 25       & 4187.1 \bigstrut\\
		\cline{2-5}             & cửa hàng 3 & 166.2525 & 25       & 4156.313 \bigstrut\\
		\cline{2-5}             & cửa hàng 4 & 166.2525 & 25       & 4156.313 \bigstrut\\
		\cline{2-5}             & cửa hàng 5 & 167.484  & 25       & 4187.1 \bigstrut\\
		\cline{2-5}             & cửa hàng 6 & 218.2908 & 25       & 5457.269 \bigstrut\\
		\hline
		\multirow{2}[4]{*}{\textbf{Tầng 4}} & khối văn phòng 1 & 724.812  & 25       & 18120.3 \bigstrut\\
		\cline{2-5}             & khối văn phòng 2 & 724.812  & 25       & 18120.3 \bigstrut\\
		\hline
		\textbf{Tầng 5 - 27} & khối văn phòng & 2040.78  & 25       & 1173449 \bigstrut\\
		\hline
	\end{tabular}%
	\label{b:ttb}%
\end{table}%
\textbf{Vậy tổng nhiệt lượng do thiết bị:} Q$_{3} = 1294.108(KW)$

\subsection{NHIỆT TOẢ DO NGƯỜI -- Q{\scriptsize 4} }
\subsubsection{Nhiệt hiện do người toả ra Q{\scriptsize 4h}}
-- Lượng nhiệt hiện do người toả ra:
\begin{equation*}
	Q_{4h} = n\times q_{h}, W
\end{equation*}

Trong đó:
\begin{itemize}
	\item n - là số người trong không gian cần điều hòa.
	\item $q_{h}$ - là nhiệt hiện tỏa ra từ một người, W/người.
\end{itemize}

Trong trường hợp số lượng người quá đông như : hội trường, rạp hát, vũ trường, sân khấu, phòng thi đấu thể thao …cần kể đến sự hấp thụ của kết cấu bao che . Do đó, ta cần kể đến hệ số tác động tức thời nt (hệ số tác động tức thời do chiếu sáng và nhiệt hiện của người), tra bảng 4-8.

Đối với các tòa nhà lớn ta cần tính thêm hệ số tác dụng không đồng thời n{\scriptsize đ}:
\begin{itemize}[label={-}]
	\item Cửa hàng bách hóa: n{\scriptsize đ} = 0,75 $ \div $ 0,9.
	\item Nhà cao tầng khách sạn: n{\scriptsize đ} = 0,8 $ \div $ 0,9.
	\item Nhà cao tầng công sở: n{\scriptsize đ} = 0,8 $ \div $ 0,9.
\end{itemize}

Vậy: $Q_{4h} = n_{t}\times n_{d}\times n\times q_{h}$

\subsubsection{Nhiệt ẩn do người toả ra -- Q{\scriptsize 4a}}
Trong không gian cần điều hoà ngoài sự hiện diện của thành phần nhiệt hiện còn có thành phần khác cũng phải kể đến là nhiệt ẩn. Nhiệt ẩn trong không gian điều hoà có thể do ngưới toả ra (như mồ hôi, do thở), do thức ăn toả ra (nơi ăn uống).

Nhiệt ẩn của phòng điều hoà được xác định như sau:
\begin{equation*}
	Q_{3} = n\times q_{a}, W
\end{equation*}

Trong đó:
\begin{itemize}
	\item n - số người trong phòng cần điều hoà.
	\item $q_{a}$ -  nhiệt ẩn do một người toả ra , W/người.
\end{itemize}

Đối với khu ăn uống cộng thêm 10 W/người do thức ăn toả ra .

Chọn hệ số tác dụng đồng thời n{\scriptsize đ} = 0,9.

Do số người hiện diện trong không gian cần điều hoà không cố định nên xem sự hiện diện của nam và nữ trong không gian điều hoà là như nhau để thuận lợi cho việc tính toán.

Ta lấy:
\begin{itemize}[label={-}]
	\item Nhiệt ẩn do một người tỏa ra: $q_{a}$ = 60 W.
	\item Nhiệt hiện do một người tỏa ra: $q_{h}$ = 70 W.
\end{itemize}

\begin{table}[H]
	\centering
	\caption{Mật độ người trong các loại phòng}
	\begin{adjustbox}{width=\textwidth}
	\begin{tabular}{|p{10.785em}|c|c|c|c|c|}
		\hline
		\textbf{Loại không  gian} & \multicolumn{1}{p{7.285em}|}{\textbf{Văn phòng làm việc}} & \multicolumn{1}{p{4.07em}|}{\textbf{Cà phê}} & \multicolumn{1}{p{4.07em}|}{\textbf{Sảnh}} & \multicolumn{1}{p{4.07em}|}{\textbf{Hội nghị}} & \multicolumn{1}{p{6.785em}|}{\textbf{Phòng phụ trợ}} \bigstrut\\
		\hline
		Mật độ , m2/người & 8        & 3        & 10       & 2        & 5 \bigstrut\\
		\hline
	\end{tabular}%
	\end{adjustbox}
	\label{b:mdn}%
\end{table}%

\begin{landscape}
\begin{table}[H]
	\vspace{1cm}
	\centering
	\begin{adjustbox}{width=1.5\textheight, height=0.4\textwidth}
	\begin{tabular}{|c|l|r|r|r|r|r|r|r|r|r|}
		\hline
		\textbf{TẦNG} & \multicolumn{1}{c|}{\textbf{TÊN PHÒNG}} & \multicolumn{1}{c|}{\textbf{DIỆN TÍCH (m²) }} & \multicolumn{1}{c|}{\textbf{MẬT ĐỘ NGƯỜI m²/người}} & \multicolumn{1}{c|}{\textbf{n}} & \multicolumn{1}{c|}{\textbf{qh (w)}} & \multicolumn{1}{c|}{\textbf{qn (w)}} & \multicolumn{1}{c|}{\textbf{nđ}} & \multicolumn{1}{c|}{\textbf{Qh (w)}} & \multicolumn{1}{c|}{\textbf{Qa (w)}} & \multicolumn{1}{c|}{\textbf{Q4 (w)}} \bigstrut\\
		\hline
		\multirow{4}[8]{*}{\textbf{tầng 1}} & sảnh văn phòng & 509.579  & 10       & 50.9579  & 70       & 60       & 0.9      & 3210.348 & 3057.474 & 6267.822 \bigstrut\\
		\cline{2-11}             & coffe    & 717.5506 & 3        & 239.1835 & 70       & 60       & 0.9      & 15068.56 & 14351.01 & 29419.57 \bigstrut\\
		\cline{2-11}             & phòng máy & 57.9477  & 5        & 11.58954 & 70       & 60       & 0.9      & 730.141  & 695.3724 & 1425.513 \bigstrut\\
		\cline{2-11}             & phòng cấp cứu & 57.9477  & 5        & 11.58954 & 70       & 60       & 0.9      & 730.141  & 695.3724 & 1425.513 \bigstrut\\
		\hline
		\multirow{2}[4]{*}{\textbf{tầng M}} & phòng điều khiển 1 & 266.8014 & 5        & 53.36028 & 70       & 60       & 0.9      & 3361.697 & 3201.617 & 6563.314 \bigstrut\\
		\cline{2-11}             & phòng điều khiển 2 & 662.885  & 5        & 132.577  & 70       & 60       & 0.9      & 8352.351 & 7954.62  & 16306.97 \bigstrut\\
		\hline
		\multirow{6}[12]{*}{\textbf{tầng 2 - 3}} & cửa hàng 1 & 218.2908 & 8        & 27.28635 & 70       & 60       & 0.9      & 1719.04  & 1637.181 & 3356.22 \bigstrut\\
		\cline{2-11}             & cửa hàng 2 & 167.484  & 8        & 20.9355  & 70       & 60       & 0.9      & 1318.937 & 1256.13  & 2575.067 \bigstrut\\
		\cline{2-11}             & cửa hàng 3 & 166.2525 & 8        & 20.78156 & 70       & 60       & 0.9      & 1309.238 & 1246.894 & 2556.132 \bigstrut\\
		\cline{2-11}             & cửa hàng 4 & 166.2525 & 8        & 20.78156 & 70       & 60       & 0.9      & 1309.238 & 1246.894 & 2556.132 \bigstrut\\
		\cline{2-11}             & cửa hàng 5 & 167.484  & 8        & 20.9355  & 70       & 60       & 0.9      & 1318.937 & 1256.13  & 2575.067 \bigstrut\\
		\cline{2-11}             & cửa hàng 6 & 218.2908 & 8        & 27.28635 & 70       & 60       & 0.9      & 1719.04  & 1637.181 & 3356.22 \bigstrut\\
		\hline
		\multirow{2}[4]{*}{\textbf{tầng 4}} & khối văn phòng 1 & 724.812  & 8        & 90.6015  & 70       & 60       & 0.9      & 5707.895 & 5436.09  & 11143.98 \bigstrut\\
		\cline{2-11}             & khối văn phòng 2 & 724.812  & 8        & 90.6015  & 70       & 60       & 0.9      & 5707.895 & 5436.09  & 11143.98 \bigstrut\\
		\hline
		\textbf{tầng 5 - 27} & khối văn phòng & 2040.78  & 8        & 255.0975 & 70       & 60       & 0.9      & 16071.14 & 15305.85 & 721670.8 \bigstrut\\
		\hline
	\end{tabular}%
	\end{adjustbox}
	\caption{Nhiệt lượng do người toả ra}
	\label{b:ndn}%
\end{table}%
\end{landscape}
\textbf{Vậy tổng nhiệt lượng do người toả ra:} Q$_{4} = 822342(W)$

\subsection{NHỆT HIỆN \& ẨN DO GIÓ TƯƠI MANG VÀO -- Q{\scriptsize 5}}
\subsubsection{Các khu vực sử dụng FCU:}
Lượng nhiệt do gió tươi mang vào được xác định bằng biểu thức sau:
\begin{equation*}
	\begin{split}
	Q_{5} &= Q_{5h} + Q_{5a} \\
	Q_{5h} &= 1.2 \times n \times l \times (t_{N} - t_{T}), W \\
	Q_{5a} &= 3 \times n \times l \times (d_{N} - d_{T}), W
	\end{split}
\end{equation*}

Trong đó:
\begin{itemize}
	\item $t_{N}$, $t_{T}$ - nhiệt độ của không khí tươi bên ngoài và không khí trong không gian điều hoà, $^{\circ}C$.
	\item $d_{N}$, $d_{T}$ - độ chứa hơi của không khí tươi bên ngoài và không khí trong không gian điều hoà, g /kg.
	\item $ n $ - số người trong không gian điều hoà.
	\item $l$ - lượng không khí tươi từ ngoài trời cần đưa vào phòng cho một người trong một giây, l/s.
\end{itemize}

Yêu cầu gió tươi:
\begin{itemize}[label={-}]
	\item Phòng làm việc: 25 m$^3$/h.người
	\item Hành lang: 25 m$^3$/h.m$^2$sàn
	\item Bếp: 30 m$^3$/h.người
	\item Cà phê: 30 m$^3$/h.người
	\item Phòng hội nghị: 30 m$^3$/h.người
\end{itemize}

-- Tuỳ thuộc vào không gian mà nhiệt độ và độ chứa hơi khác nhau.

-- Độ chứa hơi của không khí được xác định theo phần mềm DAIKIN.

\begin{table}[H]
	\centering
	\caption{Độ chứa hơi của không khí}
	\begin{tabular}{|p{7.645em}|c|c|c|}
		\hline
		\textbf{Thông số} & \multicolumn{1}{p{5.43em}|}{\textbf{t ($^{\circ}C$)}} & \multicolumn{1}{p{4.645em}|}{\textbf{j (\%)}} & \multicolumn{1}{p{5.785em}|}{\textbf{d (g/kg kk)}} \bigstrut\\
		\hline
		Bên trong & 25       & 55       & 10.94 \bigstrut\\
		\hline
		Bên ngoài & 32       & 75       & 22.84 \bigstrut\\
		\hline
	\end{tabular}%
	\label{b:dch}%
\end{table}%

\begin{landscape}
	\begin{table}[H]
		\vspace{1.3cm}
		\centering
		\begin{adjustbox}{width=1.5\textheight, height=0.4\textwidth}
		\begin{tabular}{|c|l|r|r|r|r|r|r|r|r|r|r|r|}
			\hline
			\textbf{TẦNG} & \multicolumn{1}{c|}{\textbf{TÊN PHÒNG}} & \multicolumn{1}{p{5.285em}|}{\textbf{THỂ TÍCH (m3) }} & \multicolumn{1}{p{4.07em}|}{\textbf{n}} & \multicolumn{1}{p{5.5em}|}{\textbf{l (l/s/người)}} & \multicolumn{1}{p{6.645em}|}{\textbf{LN (l/s)}} & \multicolumn{1}{p{4.07em}|}{\textbf{tN}} & \multicolumn{1}{p{4.07em}|}{\textbf{tT}} & \multicolumn{1}{p{4.07em}|}{\textbf{dT}} & \multicolumn{1}{p{4.07em}|}{\textbf{dN}} & \multicolumn{1}{p{5.855em}|}{\textbf{Q5h (W)}} & \multicolumn{1}{p{6em}|}{\textbf{Q5a (W)}} & \multicolumn{1}{p{6.715em}|}{\textbf{Q5 (W)}} \bigstrut\\
			\hline
			\multirow{4}[8]{*}{\textbf{Tầng 1}} & sảnh văn phòng & 2293.11  & 51       & 5.5      & 280.27   & 32       & 25       & 10.94    & 22.84    & 2354.25  & 10003.19 & 12357.45 \bigstrut\\
			\cline{2-13}             & coffe    & 3228.98  & 239      & 4.7      & 1124.16  & 32       & 25       & 10.94    & 22.84    & 9442.97  & 40123.02 & 49565.98 \bigstrut\\
			\cline{2-13}             & phòng máy & 260.76   & 12       & 4.5      & 52.15    & 32       & 25       & 10.94    & 22.84    & 438.08   & 1861.41  & 2299.50 \bigstrut\\
			\cline{2-13}             & phòng cấp cứu & 260.76   & 12       & 4.5      & 52.15    & 32       & 25       & 10.94    & 22.84    & 438.08   & 1861.41  & 2299.50 \bigstrut\\
			\hline
			\multirow{2}[4]{*}{\textbf{Tầng M}} & phòng điều khiển 1 & 1200.61  & 53       & 4.5      & 240.12   & 32       & 25       & 10.94    & 22.84    & 2017.02  & 8570.28  & 10587.30 \bigstrut\\
			\cline{2-13}             & phòng điều khiển 2 & 2982.98  & 133      & 4.5      & 596.60   & 32       & 25       & 10.94    & 22.84    & 5011.41  & 21293.41 & 26304.82 \bigstrut\\
			\hline
			\multirow{6}[12]{*}{\textbf{Tầng 2 - 3}} & cửa hàng 1 & 1178.77  & 27       & 5        & 136.43   & 32       & 25       & 10.94    & 22.84    & 1146.03  & 4869.45  & 6015.48 \bigstrut\\
			\cline{2-13}             & cửa hàng 2 & 904.41   & 21       & 5        & 104.68   & 32       & 25       & 10.94    & 22.84    & 879.29   & 3736.09  & 4615.39 \bigstrut\\
			\cline{2-13}             & cửa hàng 3 & 897.76   & 21       & 5        & 103.91   & 32       & 25       & 10.94    & 22.84    & 872.83   & 3708.62  & 4581.45 \bigstrut\\
			\cline{2-13}             & cửa hàng 4 & 897.76   & 21       & 5        & 103.91   & 32       & 25       & 10.94    & 22.84    & 872.83   & 3708.62  & 4581.45 \bigstrut\\
			\cline{2-13}             & cửa hàng 5 & 904.41   & 21       & 5        & 104.68   & 32       & 25       & 10.94    & 22.84    & 879.29   & 3736.09  & 4615.39 \bigstrut\\
			\cline{2-13}             & cửa hàng 6 & 1178.77  & 27       & 5        & 136.43   & 32       & 25       & 10.94    & 22.84    & 1146.03  & 4869.45  & 6015.48 \bigstrut\\
			\hline
			\multirow{2}[4]{*}{\textbf{Tầng 4}} & khối văn phòng 1 & 3913.98  & 91       & 8.5      & 770.11   & 32       & 25       & 10.94    & 22.84    & 6468.95  & 27486.46 & 33955.41 \bigstrut\\
			\cline{2-13}             & khối văn phòng 2 & 3913.98  & 91       & 8.5      & 770.11   & 32       & 25       & 10.94    & 22.84    & 6468.95  & 27486.46 & 33955.41 \bigstrut\\
			\hline
			\textbf{Tầng 5 - 27} & khối văn phòng & 8163.12  & 255      & 8.5      & 2168.33  & 32       & 25       & 10.94    & 22.84    & 18213.96 & 77390.85 & 2198910.68 \bigstrut\\
			\hline
		\end{tabular}%
		\end{adjustbox}
		\caption{Nhiệt do gió tươi mang vào}
		\label{b:ndgt}%
	\end{table}%
\end{landscape}
\textbf{Vậy tổng nhiệt lượng do gió tươi mang vào:} Q$_{5} = 2 400 660.6(W)$

\subsection{NHIỆT HIỆN \& ẨN DO LỌT GIÓ -- Q{\scriptsize 6}}
Để tiết kiệm năng lượng, phòng cần điều hoà phải được làm kín để ta chủ động cấp lượng không khí tươi cho phòng. Tuy nhiên vẫn có hiện tượng không khí tươi lọt vào phòng qua cửa ra vào, qua khe cửa sổ,…Mức độ rò rỉ phụ thuộc vào nhiều yếu tố: độ chênh áp bên trong và bên ngoài, tốc độ gió, số lần đóng mở cửa,..Lượng nhiệt đó được xác định như sau :
\begin{equation*}
	\begin{split}
	Q_{6} &= Q_{6h} + Q_{6a} \\
	Q_{6h} &= 0.39 \times \xi \times V \times (t_{N} - t_{T}), W \\
	Q_{6a} &= 0.84 \times \xi \times V \times (d_{N} - d_{T}), W		
	\end{split}
\end{equation*}

Trong đó:
\begin{itemize}
	\item $\xi$ - hệ số kinh nghiệm.
	\item $V$ - thể tích phòng, m$^3$.
\end{itemize}

Nếu ở không gian cần điều hoà có số người ra vào nhiều, cửa đóng mở nhiều lần thì cần phải bổ sung thêm vào lượng nhiệt hiện và ẩn sau:

\begin{equation*}
	\begin{split}
		Q_{bsh} &= 1.23\times L_{bs}\times (t_{N} - t_{T}), W\\
		Q_{bsa} &= 3\times L_{bs}\times (d_{N} - d_{T}), W\\
	\end{split}
\end{equation*}

Trong đó:
\begin{itemize}
	\item $t_{N}$, $t_{T}$ - nhiệt độ của không khí tươi bên ngoài và không khí trong không gian điều hoà, $^{\circ}C$.
	\item $d_{N}$, $d_{T}$ - độ chứa hơi của không khí tươi bên ngoài và không khí trong không gian điều hoà, g /kg.
	\item $ n $ - số người qua cửa trong một giờ.
	\item $L_{bs}$ - $= 0.28\times n\times L_{c}$, l/s.
	\item $L_{c}$ - Lượng không khí lọt qua mỗi lần mở cửa, m$^3$/người. 
\end{itemize}

\begin{landscape}
	\begin{table}[H]
		\vspace{0.5cm}
		\centering
	\begin{adjustbox}{width=1.5\textheight}
		\begin{tabular}{|c|l|r|r|r|r|r|r|r|r|r|}
			\hline
			\textbf{TẦNG} & \multicolumn{1}{c|}{\textbf{TÊN PHÒNG}} & \multicolumn{1}{p{6.645em}|}{\textbf{THỂ TÍCH (m3) }} & \multicolumn{1}{p{6.855em}|}{\textbf{HỆ SỐ KINH NGHIỆM}} & \multicolumn{1}{c|}{\textbf{tN}} & \multicolumn{1}{c|}{\textbf{tT}} & \multicolumn{1}{c|}{\textbf{dT}} & \multicolumn{1}{c|}{\textbf{dN}} & \multicolumn{1}{c|}{\textbf{Q6h}} & \multicolumn{1}{c|}{\textbf{Q6a}} & \multicolumn{1}{c|}{\textbf{Q6}} \bigstrut\\
			\hline
			\multirow{4}[8]{*}{\textbf{Tầng 1}} & sảnh văn phòng & 2293.11  & 0.42     & 32       & 25       & 10.94    & 22.84    & 2629.3   & 9624.9   & 12254 \bigstrut\\
			\cline{2-11}             & coffe    & 3228.98  & 0.35     & 32       & 25       & 10.94    & 22.84    & 3085.3   & 11294    & 14379 \bigstrut\\
			\cline{2-11}             & phòng máy & 260.76   & 0.7      & 32       & 25       & 10.94    & 22.84    & 498.32   & 1824.2   & 2322.5 \bigstrut\\
			\cline{2-11}             & phòng cấp cứu & 260.76   & 0.7      & 32       & 25       & 10.94    & 22.84    & 498.32   & 1824.2   & 2322.5 \bigstrut\\
			\hline
			\multirow{2}[4]{*}{\textbf{Tầng M}} & phòng điều khiển 1 & 1200.61  & 0.55     & 32       & 25       & 10.94    & 22.84    & 1802.7   & 6599.1   & 8401.8 \bigstrut\\
			\cline{2-11}             & phòng điều khiển 2 & 2982.98  & 0.4      & 32       & 25       & 10.94    & 22.84    & 3257.4   & 11924    & 15182 \bigstrut\\
			\hline
			\multirow{6}[12]{*}{\textbf{Tầng 2 - 3}} & cửa hàng 1 & 1178.77  & 0.55     & 32       & 25       & 10.94    & 22.84    & 1769.9   & 6479.1   & 8249 \bigstrut\\
			\cline{2-11}             & cửa hàng 2 & 904.41   & 0.6      & 32       & 25       & 10.94    & 22.84    & 1481.4   & 5423     & 6904.4 \bigstrut\\
			\cline{2-11}             & cửa hàng 3 & 897.76   & 0.6      & 32       & 25       & 10.94    & 22.84    & 1470.5   & 5383.1   & 6853.7 \bigstrut\\
			\cline{2-11}             & cửa hàng 4 & 897.76   & 0.6      & 32       & 25       & 10.94    & 22.84    & 1470.5   & 5383.1   & 6853.7 \bigstrut\\
			\cline{2-11}             & cửa hàng 5 & 904.41   & 0.6      & 32       & 25       & 10.94    & 22.84    & 1481.4   & 5423     & 6904.4 \bigstrut\\
			\cline{2-11}             & cửa hàng 6 & 1178.77  & 0.55     & 32       & 25       & 10.94    & 22.84    & 1769.9   & 6479.1   & 8249 \bigstrut\\
			\hline
			\multirow{2}[4]{*}{\textbf{Tầng 4}} & khối văn phòng 1 & 3913.98  & 0.35     & 32       & 25       & 10.94    & 22.84    & 3739.8   & 13690    & 17430 \bigstrut\\
			\cline{2-11}             & khối văn phòng 2 & 3913.98  & 0.35     & 32       & 25       & 10.94    & 22.84    & 3739.8   & 13690    & 17430 \bigstrut\\
			\hline
			\textbf{Tầng 5 - 27} & khối văn phòng & 8163.12  & 0.35     & 32       & 25       & 10.94    & 22.84    & 7799.9   & 28553    & 836108 \bigstrut\\
			\hline
		\end{tabular}%
	\end{adjustbox}
		\caption{Nhiệt do lọt gió}
		\label{b:ndlg}%
	\end{table}%
\end{landscape}
\textbf{Vậy tổng nhiệt lượng do gió tươi mang vào:} Q$_{6} = 969845(W)$

$\Rightarrow$ \textbf{Vậy tổng tải lạnh Q{\scriptsize 0} công trình là}: \textbf{6821.779267, KW} 

\section{THÀNH LẬP \& TÍNH TOÁN SƠ ĐỒ ĐHKK}
\subsection{SƠ ĐỒ THẲNG}
-- Sơ đồ thẳng là sơ đồ không có tái tuần hoàn không khí từ không gian điều hòa về thiết bị xử lý. Trong sơ đồ này toàn bộ không khí đưa vào thiết bị xử lý là không khí bên ngoài trời túc là khí tươi.

-- Sơ đồ thẳng được sử dụng trong các trường hợp sau:
\begin{itemize}
	\item Khi kênh gió hồi quá lớn, việc thực hiện hồi gió quá tốn kém hoặc không thực hiện được do không gian quá tốn kém.
	\item Khi phòng phát sinh nhiều chất độc hại, việc hồi gió không có lợi.
\end{itemize}
\begin{figure}[H]
	\centering
	\caption{Sơ đồ thẳng trên đồ thị l-d}
	\includegraphics[width=\textwidth]{sodothang}
\end{figure}
\subsection{SƠ ĐỒ TUẦN HOÀN KHÔNG KHÍ 1 CẤP}
Sơ đồ có thực hiện hồi một phần gió từ gian máy điều hòa trở lại thiết bị xử lý nhiệt ẩm. Do có tận dụng nhiệt của không khí tuần hoàn nên năng suất lạnh tăng so với sơ đồ thẳng nhưng chi phí đầu tư tăng do phải trang bị hệ thống kênh hồi gió, miệng hút….Sơ đồ được sử dụng rộng rãi nhất vì hệ thống tương đối đơn giản, đảm bảo được các yêu cầu vệ sinh, vận hành không phức tạp lại có tính kinh tế cao. Sơ đồ này được sử dụng cả ở lĩnh vực điều hòa tiện nghi và công nghệ yêu cầy xử lý không khí kiểu trung tâm như hội trường, rạp hát, nhà ăn, tiền sảnh, phòng họp, nhà hàng ăn uống,...
\begin{figure}[H]
	\centering
	\caption{Sơ đồ tuần hoàn không khí 1 cấp}
	\includegraphics[width=\textwidth]{kk1cap}
\end{figure}
\subsection{SƠ ĐỒ TUẦN HOÀN KHÔNG KHÍ 2 CẤP}
-- Nhiệt độ và độ ẩm không khí thổi vào phòng có thể điều chỉnh để thỏa mãn điều kiện vệ sinh hoặc thỏa mãn về ẩm độ. Hệ thống bắt buộc phải trang bị buồng hòa trộn thứ hai và hệ thống trích gió đến buồng hòa trộn nên chi phí đầu tư và vận hành tăng. Vậy nên sơ đồ này được sử dụng trong các xí nghiệp công nghiệp nhằm nâng cao hiệu quả kinh tế và tiết kiệm năng lượng.

-- Ta thấy sơ đồ tuần hoàn không khí 1 cấp là phù hợp với công trình nên \textbf{lựa chọn sơ đồ tuần hoàn không khí 1 cấp}.

\subsection{TÍNH TOÁN SƠ ĐỒ ĐHKK 1 CẤP}
\subsubsection{Xác định điểm gốc}
\subsubsection{Xác định hệ số nhiệt hiện của phòng (Room Sensible Heat Factor)}
\begin{equation*}
	RSHF = \dfrac{Q_{hf}}{Q_{hf} + Q_{af}}
\end{equation*}

Trong đó:
\begin{itemize}
	\item $Q_{hf}$ : Tổng nhiệt hiện do bức xạ, truyền nhiệt qua kết cấu bao che và nhiệt do các nguồn nhiệt do bên trong phòng tỏa ra.
	\item $Q_{af}$ : Tổng nhiệt ẩn tỏa ra từ phòng.
\end{itemize}
\subsubsection{Nhiệt hiện tổng (Grand Sensible Heat Factor)}
\begin{equation*}
	GSHF = \dfrac{Q_{h}}{Q_{h} + Q_{a}}
\end{equation*}

Trong đó:
\begin{itemize}
	\item $Q_{h}$ : Tổng nhiệt hiện của không gian điều hòa bao gồm nhiệt hiện do gió tươi mang vào.
	\item $Q_{a}$ : Tổng nhiệt hiện của không gian điều hòa bao gồm nhiệt hiện do gió tươi mang vào.
\end{itemize}
\subsubsection{Hệ số đi vòng (Bypass Factor)}
\begin{equation*}
	BF = \dfrac{Q_{h}}{Q_{h} + Q_{0}}
\end{equation*}

Trong đó:
\begin{itemize}
	\item $Q_{h}$ : Lưu lượng không khí đi qua dàn lạnh nhưng không trao đổi nhiệt ẩm.
	\item $Q_{0}$ : Lưu lượng không khí đi qua dàn lạnh có trao đổi nhiệt ẩm với dàn.
\end{itemize}
\subsubsection{Hệ số nhiệt hiện hiệu dụng ( Efective Sensible Heat Factor)}
\begin{equation*}
	ESHF = \dfrac{Q_{hf} + BF\times Q_{hN}}{(Q_{hf} + BF\times Q_{hN}) + (Q_{af} + BF\times Q_{aN})}
\end{equation*}

Trong đó:
\begin{itemize}
	\item $Q_{hN}$ : Nhiệt hiện của không gian do gió tươi mang vào.
	\item $Q_{aN}$ : Nhiệt ẩn của không gian điều hòa do gió tươi mang vào.
\end{itemize}
\subsubsection{Nhiệt độ đọng sương của thiết bị}
-- Tra bảng 4.24 [1] hay đồ thi t-d ta có nhiệt độ điểm đọng sương của thiết bị.
\subsubsection{Xác định nhiệt độ không khí sau dàn lạnh}
\begin{equation*}
	t_{0} = t_{S} + BF\times (t_{H} + t_{S}) = t_{V}
\end{equation*}

Với:
\begin{itemize}
	\item $t_{0} = t_{V}$: là nhiệt độ không khí sau dàn lạnh.
	\item $t_{S}$: là nhiệt độ đọng sương của thiết bị.
	\item $t_{H}$:  là nhiệt độ điểm hòa trộn.
	\item $BF = 0.1$: hệ số đi vòng của dàn lạnh.
\end{itemize}
\subsubsection{Xác định lưu lượng không khí của dàn lạnh}
\begin{equation*}
	L = \dfrac{Q_{hef}}{1.2\times (t_{T} - t_{S})\times (1-BF)}
\end{equation*}

\subsection{TÍNH CHỌN AHU}
\begin{itemize}
	\item \textbf{Tầng 1}\\
	 $\Rightarrow$ $Q = Q_{vp} + Q_{coffee} + Q_{pm} + Q_{cc} = 61.58 + 135.63 + 12.17 + 12.17$ = 221.55(KW)
	 
	 \item \textbf{Tầng M}\\
	 $\Rightarrow$ $Q = Q_{dk1} + Q_{dk2} = 37.34 + 87.23$ = 124.57(KW)
	 
	 \item \textbf{Tầng 2-3}\\
	 $\Rightarrow$ $Q = Q_{ch1} + Q_{ch2} + Q_{ch3} + Q_{ch4} + Q_{ch5} + Q_{ch6} = 31.49 + 25.11 + 24.95 + 19.07 + 19.20 + 24.45$ = 144.28(KW)
\end{itemize}

\begin{table}[H]
	\centering
	\caption {\textbf{Lưu lượng gió của từng khu vực}}
	% Table generated by Excel2LaTeX from sheet 'Sheet1'
	\begin{tabular}{|c|c|c|c|c|}
		\hline
		\textbf{TẦNG} & \textbf{TÊN PHÒNG} & \textbf{L } & \textbf{LN} & \textbf{LT} \bigstrut\\
		\hline
		\multirow{4}[8]{*}{\textbf{TẦNG 1}} & sảnh văn phòng & 4285.32 & 280.27 & 4005.05 \bigstrut\\
		\cline{2-5}      & coffe & 4533.46 & 1124.16 & 3409.30 \bigstrut\\
		\cline{2-5}      & phòng máy & 614.45 & 52.15 & 562.30 \bigstrut\\
		\cline{2-5}      & phòng cấp cứu & 614.45 & 52.15 & 562.30 \bigstrut\\
		\hline
		\multirow{2}[4]{*}{\textbf{Tầng M}} & phòng điều khiển 1 & 619.53 & 240.12 & 379.41 \bigstrut\\
		\cline{2-5}      & phòng điều khiển 2 & 1473.39 & 596.60 & 876.80 \bigstrut\\
		\hline
		\multirow{6}[12]{*}{\textbf{Tầng 2 - 3}} & cửa hàng 1 & 909.95 & 136.43 & 773.51 \bigstrut\\
		\cline{2-5}      & cửa hàng 2 & 337.39 & 104.68 & 232.72 \bigstrut\\
		\cline{2-5}      & cửa hàng 3 & 626.44 & 103.91 & 522.53 \bigstrut\\
		\cline{2-5}      & cửa hàng 4 & 626.44 & 103.91 & 522.53 \bigstrut\\
		\cline{2-5}      & cửa hàng 5 & 630.46 & 104.68 & 525.78 \bigstrut\\
		\cline{2-5}      & cửa hàng 6 & 909.95 & 136.43 & 773.51 \bigstrut\\
		\hline
	\end{tabular}%
\end{table}
\newpage
\begin{table}[H]
	\centering
	\caption{\textbf{Chọn FCU}}
	\begin{tabular}{|c|c|c|c|c|c|}
		\hline
		\textbf{Tầng} & \textbf{Ký hiệu} & \textbf{Khu vực} & \textbf{Loại} & \multicolumn{1}{|}{\textbf{Công suất \newline{}(kw)}} & \multicolumn{1}{|}{\textbf{Số lượng \newline{}(cái)}} \bigstrut\\
		\hline
		\multirow{3}[6]{*}{\textbf{Tầng 1}} & FCU-1F-01 & Sảnh văn phòng & \multicolumn{1}{c|}{\multirow{6}[12]{*}{Âm trần \newline{}nối ống gió}} & 9,94  & 10 \bigstrut\\
		\cline{2-3}\cline{5-6}      & FCU-1F-11 & Cấp cứu, phòng máy &       & 12.39 & 2 \bigstrut\\
		\cline{2-3}\cline{5-6}      & FCU-1F-13 & Coffee &       & 9,94  & 10 \bigstrut\\
		\cline{1-3}\cline{5-6}\multirow{2}[4]{*}{\textbf{Tầng M}} & FCU-M-01 & Phòng điều khiển 1 &       & 17.71 & 6 \bigstrut\\
		\cline{2-3}\cline{5-6}      & FCU-M-07 & Phòng điều khiển 2 &       & 9,94  & 2 \bigstrut\\
		\cline{1-3}\cline{5-6}\textbf{Tầng 2-3} & FCU-2F-01 & Cửa hàng &       & 14.1  & 12 \bigstrut\\
		\hline
	\end{tabular}%
\end{table}
	

\section{TÍNH TOÁN CHU TRÌNH LẠNH – CHỌN MÁY NÉN}
\subsection{CHỌN MÔI CHẤT LẠNH}
-- Tác nhân lạnh phải thỏa mãn các yêu cầu:
\begin{itemize}
	\item Yêu cầu nhiệt động.
	\item Yêu cầu về hóa lý.
	\item Yêu cầu về lý sinh.
	\item Yêu cầu kinh tế.
\end{itemize}

Có nhiều tác nhân lạnh, với những đặc điểm khác nhau do đó rất khó khăn khi tìm tác nhân lạnh thoả mãn các yêu cầu trên. Sau đây ta sẽ phân tích và lựa chọn tác nhân lạnh thích hợp.

\subsubsection{Amoniac}
Là chất khí không màu, mùi hôi khó thở, độc hại với cơ thể con người, hàm lượng cho phép trong không khí là 0,02mg/l. Hơi NH3 nhẹ hơn không khí.

Ở áp suất: 
\hspace{1cm}
p = 6.3 bar, có nhiệt độ sôi : t$_{S}$ = 10$^{\circ}$C

\hspace{2.95cm}
p = 1 bar, có nhiệt độ sôi : t$_{S}$ = –34$^{\circ}$C
\begin{figure}[H]
	\centering
	\includegraphics[width=0.6\textwidth]{amoniac}
	\caption{Bình ga Amoniac}
\end{figure}

Dễ hoà tan trong nước, cho phép chứa 0,2\% nước.

Không ăn mòn kim loại đen, nhôm.

Dễ kiếm, rẻ tiền.

Năng suất lạnh thể tích tương đối lớn.

Nếu có nước thì ăn mòn các kim loại màu như kẽm, đồng…

Ít hòa tan tốt dầu, ảnh hưởng đến bôi trơn, truyền nhiệt, không làm mát tốt.

Nhiệt độ cuối tầm nén cao, nếu không làm mát tốt máy nén dễ làm cháy dầu bôi trơn và làm hỏng lớp bề mặt kim loại.
\subsubsection{Freon(R12)}
Là chất khí không màu, có mùi thơm rất nhẹ đặc biệt không cảm thấy được ở nồng độ nhỏ hơn 20\%, ở thể hơi nặng hơn không khí 4,18 lần, ở thể lỏng nặng hơn nước khoảng 1.3 lần.

Ở áp suất: 
\hspace{1cm}
p = 4.5 bar, có nhiệt độ sôi : t$_{S}$ = 10$^{\circ}$C

\hspace{2.95cm}
p = 1.014 bar, có nhiệt độ sôi : t$_{S}$ = –30$^{\circ}$C
\begin{figure}[H]
	\centering
	\includegraphics[width=0.3\textwidth]{R12}
	\caption{Bình ga R12}
\end{figure}

Không gây cháy nổ, được xem là tác nhân lạnh an toàn.

Năng suất lạnh riêng khối lượng và thể tích đều nhỏ → kết cấu cồng kềnh thích hợp cho phụ tải lạnh nhỏ.

Nhiệt độ cuối tầm nén thấp.

Không độc hại cho cơ thể.

Không ăn mòn kim loại và không dẫn điện.

Hoà tan vô tận trong dầu.

Dễ kiếm và bảo quản dễ dàng.

Ở nhiệt độ > 400$^{\circ}$C rất dể cháy khi gặp lửa tạo thành các hổn hợp độc hại.

Tính lưu động kém, không hòa tan nước nên dễ bị tách ẩm.

Độ nhớt động học lớn nên tổn thất áp suất trên đường ống lớn.

Năng suất lạnh riêng thể tích bé. Do đó, hệ thống cồng kềnh.

Năng suất lạnh riêng khối lượng bé, vì thế lượng tác nhân nạp vào hệ thống nhiều, thích hợp cho hệ thống có năng suất lạnh nhỏ và rất nhỏ.

Khả năng rò rỉ lớn, những nơi rò rỉ có vết dầu do tính hoà tan dầu tốt.

\subsubsection{Freon R22}
Là chất khí không màu, có mùi thơm rất nhẹ.

Ở áp suất: 
\hspace{1cm}
p = 6.82 bar, có nhiệt độ sôi : t$_{S}$ = 10$^{\circ}$C

\hspace{2.95cm}
p = 1.05 bar, có nhiệt độ sôi : t$_{S}$ = –40$^{\circ}$C
\begin{figure}[H]
	\centering
	\includegraphics[width=0.6\textwidth]{R22}
	\caption{Ga R22}
\end{figure}

Năng suất lạnh riêng thể tích lớn hơn 1,6 lần so với R12.

Khả năng trao đổi nhiệt lớn hơn 1,3 lần so với R12.

Không gây nổ.

Hoà tan vô tận trong dầu ở nhiệt độ cao, còn ở nhiệt độ thấp thì ít hơn.

Khả năng hoà tan nước lớn hơn R12 khoảng 5 lần, do đó giảm bớt nguy cơ tắc ẩm.

Không ăn mòn kim loại, không độc hại.

Có tính chất nhiệt động tốt.

Trong khoảng nhiệt độ từ –20$^{\circ}$C đến –40$^{\circ}$C thì hầu như hoàn toàn không hoà tan trong dầu. Do đó, tránh hoạt động ở vùng nhiệt độ này.

Có khả năng dẫn điện ở thể lỏng, các tính chất về điện của R22 bị giảm nhanh khi có ẩm và bẩn.

Có khả năng làm trương phòng một số chất hữu cơ như : cao su… Do đó, cần sử dụng các loại chất dẻo và cao su đặc biệt để làm kín.

Giá thành tương đối đắt.

\subsubsection{Freon R134A}
Ở áp suất: 
\hspace{1cm}
p = 4.15 bar, có nhiệt độ sôi : t$_{S}$ = 10$^{\circ}$C

\hspace{2.95cm}
p = 1.07 bar, có nhiệt độ sôi : t$_{S}$ = –25$^{\circ}$C
\begin{figure}[H]
	\centering
	\includegraphics[width=0.6\textwidth]{R134A}
	\caption{Ga 134A}
\end{figure}

Không gây nổ.

Không ăn mòn kim loại, không độc hại.

Có tính chất nhiệt động tốt.

Hòa tan trong dầu.

Không gây ảnh hưởng đến môi trường.

Giá thành tương đối đắt.

Tính chất nhiệt động không tốt bằng R22.

\textbf{$\Rightarrow$ Từ những đặc điểm phân tích ta chọn tác nhân cho hệ thống máy lạnh là Freon R134A.} Vì với tác nhân này, không huỷ hoại tầng ozon như các loại chất Freon khác.

\subsection{TÍNH TOÁN CHU TRÌNH LẠNH}
\subsubsection{Các thông số ban đầu}
Ở đây ta dùng dàn ngưng giải nhiệt bằng nước.

Trong đó:
\begin{itemize}
	\item $t_{W1}, t_{W2}$ - Nhiệt độ vào và ra của nước giải nhiệt (dàn ngưng).
	\item $t_{W1}$ = 35$^{\circ}$C
	\item $t_{W2}$ = 40$^{\circ}$C
	\item $t_{K}$ = 45$^{\circ}$C
\end{itemize}

Chọn nhiệt độ bay hơi $t_{0}$ = 5$^{\circ}$C

Chọn độ quá nhiệt: 10$^{\circ}$C

Chọn độ quá lạnh: 5$^{\circ}$C

\subsubsection{Tính toán thông số điểm nút của chu trình}
\begin{figure}[H]
	\centering
	\includegraphics[width=0.47\textwidth]{SDNL1}
	\caption{Sơ đồ nguyên lý chu trình lạnh 1 cấp}
	\includegraphics[width=0.45\textwidth]{lgp-h}
	\caption{Đồ thị $\log$P - h chu trình lạnh 1 cấp}
\end{figure}

-- Từ các thông số đầu vào, ta tính được các thông số còn lại của môi chất lạnh:

\begin{table}[H]
	\vspace{0.5cm}
	\centering
	
	\begin{tabular}{|c|c|c|c|c|c|}
		\hline
		Điểm  & t     & p     & i or h & s     & v \bigstrut\\
		\hline
		& °C    & bar   & kJ/kg & kJ/kgK & m3/kg \bigstrut\\
		\hline
		1     & 5     & 3.496 & 400   & 1.721 & 0.0581 \bigstrut\\
		\hline
		1'    & 15    & 3.496 & 410.72 & 1.756 & 0.0612 \bigstrut\\
		\hline
		2     & 57    & 11.597 & 435.65 & 1.756 &  \bigstrut\\
		\hline
		3     & 45    & 11.597 & 264.16 & 1.215 &  \bigstrut\\
		\hline
		3'    & 40    & 11.597 & 255.95 & 1.189 &  \bigstrut\\
		\hline
		4     & 5     & 3.496 & 255.95 & 1.202 &  \bigstrut\\
		\hline
	\end{tabular}%
\end{table}

Công nén đơn vị: $w = i_{2} - i_{1'} = $ 435.65 - 410.72 = 24.93 (kJ/kg)

Năng suất lạnh đơn vị: $q_{o} = i_{1'} - i_{4} = $ 410.72 - 255.95 = 154.77 (kJ/kg)

Năng suất giải nhiệt đơn vị: $q_{k} = i_{2} - i_{3'} = $ 435.65 - 255.95 = 179.70 (kJ/kg)

Kiểm tra lại bằng định luật nhiệt động thứ 1:
\begin{equation*}
	\begin{split}
		q_{k} &= q_{o} + w = 154.77 + 24.93 = 179.70 (kJ/kg)\\	
		COP &= \dfrac{q_{o}}{w} = \dfrac{154.77}{24.93} = 6.21
	\end{split}
\end{equation*}

\subsection{TÍNH CHỌN MÁY NÉN}
Tổng tải lạnh là Q = 490.398 (kw)

Chọn hệ thống gồm 2 tổ máy, mỗi tổ máy gồm có máy nén, bình ngưng, bình bốc hơi, tháp giải nhiệt. Gồm 1 tổ máy hoạt động và 1 tổ máy dự phòng. Vì tính chất công trình quốc tế quan trọng và không thể dừng hệ thống, bên cạnh đó 2 tổ máy sẽ được chạy luân phiên để tăng tuổi thọ của máy.

-- Năng suất lạnh của chiller: $Q_{o} = $ 539.44(kw)

-- Lưu lượng tác nhân lạnh: 
\begin{equation*}
	G = \dfrac{Q_{o}}{q_{o}} = \dfrac{539.44}{154.77} = 3.49 (kg/s)
\end{equation*}

-- Hệ số lưu lượng của máy nén: 

Theo trang 116 TL Máy Lạnh với tỉ số áp suất $\dfrac{p_{k}}{p_{o}} = \dfrac{11.6}{3.5} = 3.32$
	
Theo đồ thị 5-41 TL 2 ta được $\lambda$ = 0,75 lấy theo R134a.


-- Thể tích hơi thực tế hút vào máy nén: 
\begin{equation*}
	V_{tt} = G \times v_{1'} = 3.49 \times 0.06 = 0.21(m^3/s) = 12.6(m^3/min)
\end{equation*}

-- Thể tích hơi lý thuyết hút vào máy nén:
\begin{equation*}
	V_{lt} = \dfrac{V_{tt}}{\lambda} = \dfrac{0.21}{0.75} = 0.28 (m^3/s)
\end{equation*}

-- Công nén đoạn nhiệt của máy nén:
\begin{equation*}
	N_{a} = G\times w = 3.49 \times 24.93 = 86.89 (kw)
\end{equation*}

-- Hiệu suất hiệu dụng của máy nén:

Theo đồ thị 5-42 trang 117 TL 2, với tỉ số nén $ \dfrac{p_{k}}{p_{o}} =  $3.32

Ta được $\eta_{e}$ = 0.85

-- Công suất hiệu dụng của máy nén:
\begin{equation*}
	N_{e} = \dfrac{N_{a}}{\eta_{e}} = \dfrac{86.89}{0.85} = 102.23(kw)
\end{equation*}

-- Công suất trên trục động cơ:
\begin{equation*}
	N = \dfrac{N_{e}}{\eta_{tr}} = \dfrac{102.23}{1} = 102.23(kw)
\end{equation*}

$ \eta_{tr} $ = 1 vì truyền động trực tiếp. 

-- Hệ số lạnh hiệu dụng:
\begin{equation*}
	K_{e} = \dfrac{Q_{o}}{N_{e}} = \dfrac{539.44}{102.23} = 5.28
\end{equation*}

-- Năng suất giải nhiệt của thiết bị ngưng tụ:
\begin{equation*}
	Q_{k} = G\times q_{k} = 3.49 \times 179.70 = 626.33(kw)
\end{equation*}

-- Lượng nước giải nhiệt cho bình ngưng tụ:
\begin{equation*}
	G_{k} = \dfrac{Q_{k}}{c_{p}\times \Delta t} = \dfrac{626.33}{4.183\times 5} = 29.95(kg/s)
\end{equation*}

-- Lưu lượng nước lạnh qua bình bay hơi:
\begin{equation*}
	G_{o} = \dfrac{Q_{o}}{c_{p}\times \Delta t} = \dfrac{539.44}{4.183 \times 5} = 25.79(kg/s)
\end{equation*}

-- Xác định lượng dầu bôi trơn máy nén

Lượng dầu có thể xác định theo Trang 84 TL[2]: $G_{d} = G\times g_{d}$.

Trong đó:
\begin{itemize}
	\item $G$ - Lượng tác nhân làm việc trong máy nén.
	\item $g_{d}$ - Lượng dầu tiêu hao trên 1 kg tác nhân lạnh.
\end{itemize}

Theo bảng 5-28 trang 90 TL Máy Lạnh, với pk/po =  ta tra được gd = 0.5

$\Rightarrow G_{d} = 3.49 \times 0.5 = 1.74(kg/s)$

Ngoài lượng dầu Gd đưa vào khoang làm việc của máy nén cón có một lượng dầu khác để bôi trơn ổ trục, chèn. Vậy tổng lượng dầu cần thiết cho máy nén là:

$G_{d} = \times 1.2 = 1.74 \times 1.2 = 2.09(kg/s)$

	%Trang bìa
\addtocontents{toc}{\protect\newpage}
\chapmoi{HỆ THỐNG PHÒNG CHÁY CHỮA CHÁY}
\section{KHÁI QUÁT}
-- Công trình Ha Noi Plaza là một công trình xây dựng có quy mô lớn, với mục đích sử dụng chủ yếu là nhà làm việc văn phòng, khách sạn 5 sao, trung tâm thương mại. 

-- Đặc biệt đây là công trình có kiến trúc nhiều tầng nên việc tác chiến chữa cháy có những khó khăn nhất định. Do mức độ quan trọng trên nên việc đầu tư trang thiết bị PCCC cho công trình là một mục tiêu rất quan trọng và thiết thực.

-- Thực hiện ý tưởng trên chúng em đã chọn phương án thiết kế hệ thống PCCC cho công trình. Căn cứ vào yêu cầu đặc tính của công trình, tiêu chuẩn Việt Nam về an toàn phòng cháy chữa cháy để thiết kế các hệ thống phòng cháy chữa cháy cho công trình gồm các hạng mục sau:
\begin{itemize}[label={-}]
	\item Hệ thống báo cháy tự động.
	\item Hệ thống phát hiện \& báo cháy.
	\item Hệ thống chữa cháy Sprinkler tự động.
	\item Hệ thống chữa cháy màn ngăn nước tự động.
	\item Hệ thống trụ chữa cháy và chữa cháy cuộn vòi.
	\item Bình chữa cháy xách tay.
\end{itemize}
\subsection{CÁC TIÊU CHUẨN ÁP DỤNG}
Các tiêu chuẩn áp dụng cho hệ thống được liệt kê theo danh sách dưới đây:
\begin{enumerate}
	\item QCVN 06-2010/BXD –- Quy chuẩn kỹ thuật quốc gia về an toàn cháy cho nhà và công trình.
	\item TCVN 2622-1995 –- Phòng cháy, chống cháy cho nhà và công trình –- Yêu cầu thiết kế.
	\item TCVN 3890-2009 –- Phương tiện phòng cháy và chữa cháy cho nhà và công trình –- Trang bị, bố trí, kiểm tra, bảo dưỡng.
	\item TCCN 5738-2001 -- Hệ thống báo cháy –- Yêu cầu kỹ thuật.
	\item TCVN 5760-1993 –- Hệ thống chữa cháy – Yêu cầu chung về thiết kế, lắp đặt và sử dụng.
	\item TCVN 6101-1996 –- Thiết bị chữa cháy –- Hệ thống chữa cháy Cacbon Dioxit –- Thiết kế và lắp đặt.
	\item TCVN 6379-1998 –- Thiết bị chữa cháy –- Trụ nước chữa cháy – Yêu câu kỹ thuật
	\item TCVN 7161-1-2009 –- Hệ thống chữa cháy bằng khí –- TÍnh chất vật lý và thiết kế -- Phần 1: Yêu cầu chung
	\item TCVN 7568-1 $\sim$ 6:2013 –- Hệ thống phát hiện và báo cháy – Thiết bị
	\item TCVN 5738:2001 -- Hệ thống báo cháy tự động - Yêu cầu kĩ thuật
	\item TCVN 7336:2003 -- Hệ thống chữa cháy tự động sprinker – Yêu cầu thiết kế và lắp đặt.
	\item TCVN 7435-1:2004 -– Phòng cháy, chữa cháy -– Bình chữa cháy xách tay và xe đẩy chữa cháy.
	\item Tham chiếu các tiêu chuẩn của NFPA (National Fire Protection Association -- Hiệp hội Phòng cháy Quốc gia của Hoa Kỳ):
	
	NFPA 13: Tiêu chuẩn thiết kế, lắp đặt hệ thống chữa cháy sprinkler.
\end{enumerate}

\subsection{YÊU CẦU ĐỐI VỚI CÁC HỆ THỐNG PCCC CHO CÔNG TRÌNH NHÀ CAO TẦNG}
Căn cứ vào tính chất nguy hiểm cháy nổ của công trình hệ thống PCCC cho công trình phải đảm bảo yêu cầu sau:
\subsubsection{Yêu cầu về phòng cháy:}
-- Phải áp dụng cac giải pháp phòng cháy đảm bảo hạn chế tối đa khả năng xảy ra hoả hoạn. Trong trường hợp xảy ra hoả hoạn thì phải phát hiện đám cháy nhanh để cứu chữa kịp thời không để đám cháy lan ra các khu vực khác sinh ra cháy lớn khó cứu chữa gây ra hậu quả nghiêm trọng.

-- Biện pháp phòng cháy phải đảm bảo sao cho khi có cháy thì người và tài sản trong toà nhà dễ dàng sơ tán sang khu vực an toàn một cách nhanh chóng nhất.

-- Trong bất cứ điều kiện nào khi xảy ra cháy ở những vị trí dễ xảy ra cháy như các khu vực kỹ thuật, sảnh trong trường phải phát hiện được ngay ở nơi phát sinh cháy để tổ chức cứu chữa kịp thời.
\subsubsection{Yêu cầu về chữa cháy:}
Trang thiết bị chữa cháy của công trình phải đảm bảo các yêu cầu sau:
\begin{itemize}
	\item Trang thiết bị chữa cháy phải sẵn sàng ở chế độ thường trực, khi xảy ra cháy phải sử dụng ngay.
	\item Thiết bị chữa cháy phải là loại phù hợp và chữa cháy có hiệu quả đối với các đám cháy xảy ra trong công trình.
	\item Thiết bị chữa cháy trang bị cho công trình phải là loại dễ sử dụng, phù hợp với công trình và điều kiện nước ta.
	\item Thiết bị chữa cháy phải là loại chữa cháy không làm hư hỏng các dụng cụ, thiết bị khác tại khu vực chữa cháy thiệt hại thứ cấp.
	\item Trang thiết bị hệ thống PCCC được trang bị phải đảm bảo điều kiện đầu tư tối thiểu nhưng đạt hiệu quả tối đa.
\end{itemize}

\section{THIẾT KẾ HỆ THỐNG BÁO CHÁY TỰ ĐỘNG}
\subsection{MÔ TẢ CHUNG}
Phương án thiết kế bao gồm:
\begin{itemize}[label={-}]
	\item Thiết kế lắp đặt hệ thống báo cháy tự động địa chỉ cho toàn bộ công trình.
	\item Thiết kế lắp đặt hệ thống chuông, nút ấn báo cháy cho toàn bộ công trình.
	\item Thiết kế lắp đặt đèn chỉ dẫn lối thoát nạn (EXIT) đèn sự cố khi có sự cố xảy ra.
\end{itemize}

\subsection{HỆ THỐNG BÁO CHÁY ĐỊA CHỈ TỰ ĐỘNG}
Hệ thống báo cháy tự động bao gồm các bộ phận cơ bản như:
\begin{enumerate}
	\item Các đầu cảm biến (Detector) phát hiện sự cháy.
	\item Nút ấn báo cháy tay.
	\item Các modul phân tích, xử lý tín hiệu.
	\item Trung tâm điều khiển xử lý các thông tin từ đầu cảm biến và nút ấn báo cháy tay đưa về.
	\item Bộ phận báo động cháy gồm: còi, chuông.
	\item Hệ thống dây dẫn: gồm hệ thống dây dẫn tín hiệu và dây cấp nguồn.
	\item Nguồn điện.
\end{enumerate}

Các thiết bị điều khiển ngoại vi như máy in dữ liệu báo cháy, tủ ghép nối điều khiển hệ thống trên máy tính, tủ ghép nối tín hiệu điều khiển hệ thống chữa cháy, hệ thống thang máy, cũng như đóng mở thiết bị thông gió, cửa thoát nạn.

\subsection{PHƯƠNG ÁN THIẾT KẾ}
Phương án thiết kế hệ thống báo cháy tự động cho toà nhà được chọn là hệ thống báo cháy địa chỉ thông minh.

Tổ hợp chuông, nút ấn báo cháy trên các tầng được bố trí tại các vị trí nhiều người đi lại như khu vực gần cầu thang máy và cầu thang bộ để thuận tiện cho việc quan sát xử lý sự cố khi có đám cháy xảy ra. Các nút ấn báo cháy trên mỗi tầng đều là nút ấn báo cháy thường kết hợp với module địa chỉ và được lắp trên 01 địa chỉ cho từng tầng.

Thiết bị báo động được chọn là chuông báo cháy. Trên các tầng chuông báo cháy được lắp đặt trong tổ hợp cùng nút ấn báo cháy. Các chuông được lắp đặt trên cùng 01 tuyến dây cấp nguồn và đưa về module địa chỉ cho chuông báo cháy để điều khiển hoạt
động của tất cả các chuông trên cùng một tầng.

Hệ thống dây dẫn tín hiệu cho đầu báo cháy địa chỉ là loại cáp chống nhiễu chuyên dụng và có tiết diện 1.2 mm$^2$.

\section{PHƯƠNG TIỆN CHỮA CHÁY BAN ĐẦU \& HỆ THỐNG CHỮA CHÁY BẰNG NƯỚC}
\subsection{KHÁI QUÁT}
\subsubsection{Căn cứ thiết kế}
-- Các tiêu chuẩn áp dụng, tiêu chuẩn và tài liệu tham khảo trong phần II - Các tiêu chuẩn áp dụng.

-- Bản vẽ kiến trúc của toà nhà.
\subsubsection{Mô tả chung hệ thống}
Sau khi nghiên cứu đặc điểm kiến trúc, quy mô, tầm quan trọng, tính chất sử dụng và mức độ nguy hiểm của công trình, giải pháp thiết kế hệ thống chữa cháy bằng nước cho công trình gồm các bộ phận cơ bản sau:
\begin{itemize}[label={-}]
	\item Bể chứa nước chữa cháy.
	\item Hệ thống bơm chữa cháy chính, máy bơm dự phòng, máy bơm bù áp lực.
	\item Tủ điều khiển bơm chữa cháy.
	\item Bình áp lực, đồng hồ đo áp lực, công tắc áp lực.
	\item Hệ thống Alarm van và công tắc dòng chảy.
	\item Hệ thống van chặn, van một chiều, van hút lọc.
	\item Hệ thống đường ống dẫn nước.
	\item Các đầu phun Sprinkler.
	\item Họng nước chữa cháy vách tường.
	\item Họng nước chữa cháy ngoài nhà.
	\item Phương tiện chữa cháy ban đầu sử dụng các bình chữa cháy xách tay ABC , bình khí CO2 cho các phòng cơ điện, phòng điều khiển, phòng máy tính.
\end{itemize}

\subsection{HỆ THỐNG CHỮA CHÁY BẰNG NƯỚC}
-- Hệ thống chữa cháy bằng nước được cấp tới các đầu phun và Sprinker và các họng nước chữa cháy vách tường, bằng mạng đường ống có đường kính từ D25-D150. Đường ống chính D150 chạy từ trạm bơm đến hộp kỹ thuật. Đường ống đứng trục chính đi trong hộp kỹ thuật sử dụng ống thép tráng kẽm D150 và D100 chạy dọc theo hộp kỹ thuật và được nối với bể nước tại tầng mái. Tại mỗi tầng có hai trục đường ống chính.

-- Việc cấp nước và tạo áp cho mạng đường ống chữa cháy được sử dụng chung giữa hệ thống Sprinkler và họng nước chữa cháy vách tường bằng bộ bơm chữa cháy trục ngang.

-- Tại các tầng thiết kế Sprinker bố trí các đồng hồ đo áp suất nhằm kiểm tra áp lực của mạng đường ống.

-- Việc khởi động và tắt máy bơm có thể hoàn toàn tự động hoặc bằng tay. Tủ điều khiển chữa cháy nhận tín hiệu từ công tắc áp lực, công tắc dòng chảy để đưa ra tín hiệu điều khiển chữa cháy đến tủ khởi động bơm chữa cháy. Tuỳ theo các trạng thái mà tủ điểu khiển chữa cháy quyết định đưa ra tín hiệu điều khiển khởi động bơm chữa cháy chính, bơm chữa cháy dự phòng hay bơm bù áp lực. Ngoài ra tủ điều khiển chữa cháy còn đưa ra tín hiệu báo động chữa cháy ra chuông, đèn báo chữa cháy cũng như trung tâm báo cháy tự động.

-- Khi áp suất trong hệ thống tụt xuống còn 90\% so với mức cài đặt trước thì công tắc áp suất sẽ khởi động bơm bù áp suất ( Jockey pump). Một Zole khống chế thời gian chạy tối thiểu được gắn vào hệ thống điều khiển để tránh trường hợp máy bơm bù bị khởi động liên tục.

-- Nếu như áp suất của hệ thống tụt xuống còn 80\% so với mức cài đặt từ trước thì bơm bù áp suất sẽ dừng và máy bơm chữa cháy chính sẽ được khởi động ( 01 máy bơm thường trực đã được lựa chọn).

-- Nguồn điện cấp cho máy bơm lấy từ nguồn ưu tiên, đồng thời được cấp bằng nguồn điện máy phát của toà nhà.

-- Bể nước phục vụ cho toàn hệ thống chữa cháy được dùng chung với bể nước sinh hoạt được đặt chìm tại tầng trệt. Lượng nước dùng cho chữa cháy được đảm bảo bởi các thiết bị kiểm tra mức nước và điều khiển bơm sinh hoạt.

\subsection{HỆ THỐNG CHỮA CHÁY SPRINKLER}
-- Hệ thống chữa cháy sử dụng các đầu Sprinkler hướng lên (Upright) được lắp đặt cho tầng hầm và các tầng không có trần giả, sử dụng Sprinkler quay xuống (Pendent) được bố trí cho các tầng có bố trí trần giả. Khoảng cách giữa các đầu phun là 2.6m đến 4m, khoảng cách đến tường 1.2m đến 2.0m (xem bản vẽ thiết kế).

-- Thông số kỹ thuật cơ bản để tính toán, thiết kế hệ thống “Theo TCVN 7336:2003” như bảng dưới đây:
\begin{table}[H]
	\centering
	\begin{tabular}{|l|l|}
		\hline
		\textbf{Diện tích bảo vệ tối đa cho một đầu phun: } & \textcolor[rgb]{ 1,  0,  0}{9-12m$^2$} \bigstrut\\
		\hline
		\textbf{Mức độ nguy hiểm:} & \textcolor[rgb]{ 1,  0,  0}{Thông thường} \bigstrut\\
		\hline
		\textbf{Áp suất tại đầu phun: } & \textcolor[rgb]{ 1,  0,  0}{1at (10m.c.n)} \bigstrut\\
		\hline
		\textbf{Cường độ phun :} & \textcolor[rgb]{ 1,  0,  0}{14,4 lit/phút m$^2$} \bigstrut\\
		\hline
		\textbf{Thời gian phun: } & \textcolor[rgb]{ 1,  0,  0}{30 phút} \bigstrut\\
		\hline
		\textbf{Diện tích được bảo vệ tính toán :} & \textcolor[rgb]{ 1,  0,  0}{200 m$^2$} \bigstrut\\
		\hline
	\end{tabular}%
	\label{b:pcccts}%
\end{table}%

-- Nguồn nước được cấp sẽ đủ cung cấp cho cả hệ thống Sprinkler và họng nước chữa cháy trong nhà hoạt động đồng thời.

\begin{table}[H]
	\centering
	\caption{Thống kê đầu Sprinkler}
	\begin{tabular}{|c|l|r|r|r|}
		\hline
		\multirow{2}[2]{*}{\textbf{ TẦNG}} & \multicolumn{1}{c|}{\multirow{2}[2]{*}{\textbf{TÊN PHÒNG}}} & \multicolumn{1}{c|}{\multirow{2}[2]{*}{\textbf{DIỆN TÍCH (m²) }}} & \multicolumn{1}{c|}{\multirow{2}[2]{*}{\textbf{UPRIGHT}}} & \multicolumn{1}{c|}{\multirow{2}[2]{*}{\textbf{PENDANT}}} \bigstrut[t]\\
		&          &          &          &  \bigstrut[b]\\
		\hline
		\multirow{7}[14]{*}{\textbf{Hầm B1}} & Kho M\&E & 215.2    & 18       & 0 \bigstrut\\
		\cline{2-5}             & Phòng điều khiển & 52       & 4        & 0 \bigstrut\\
		\cline{2-5}             & Phòng máy phát điện & 170      & 14       & 0 \bigstrut\\
		\cline{2-5}             & Phòng nồi hơi & 650      & 54       & 0 \bigstrut\\
		\cline{2-5}             & Phòng máy chiller & 910      & 76       & 0 \bigstrut\\
		\cline{2-5}             & Phòng ắc quy & 43       & 4        & 0 \bigstrut\\
		\cline{2-5}             & Phòng máy biến áp & 400      & 33       & 0 \bigstrut\\
		\hline
		\multirow{4}[8]{*}{\textbf{Tầng 1}} & sảnh văn phòng & 510      & 42       & 0 \bigstrut\\
		\cline{2-5}             & coffe    & 718      & 0        & 60 \bigstrut\\
		\cline{2-5}             & phòng máy & 58       & 0        & 5 \bigstrut\\
		\cline{2-5}             & phòng cấp cứu & 58       & 0        & 5 \bigstrut\\
		\hline
		\multirow{2}[4]{*}{\textbf{Tầng M}} & phòng điều khiển 1 & 267      & 0        & 22 \bigstrut\\
		\cline{2-5}             & phòng điều khiển 2 & 663      & 0        & 55 \bigstrut\\
		\hline
		\multirow{6}[12]{*}{\textbf{Tầng 2 - 3}} & cửa hàng 1 & 218      & 0        & 18 \bigstrut\\
		\cline{2-5}             & cửa hàng 2 & 167      & 0        & 14 \bigstrut\\
		\cline{2-5}             & cửa hàng 3 & 166      & 0        & 14 \bigstrut\\
		\cline{2-5}             & cửa hàng 4 & 166      & 0        & 14 \bigstrut\\
		\cline{2-5}             & cửa hàng 5 & 167      & 0        & 14 \bigstrut\\
		\cline{2-5}             & cửa hàng 6 & 218      & 0        & 18 \bigstrut\\
		\hline
		\multirow{2}[4]{*}{\textbf{Tầng 4}} & khối văn phòng 1 & 725      & 0        & 60 \bigstrut\\
		\cline{2-5}             & khối văn phòng 2 & 725      & 0        & 60 \bigstrut\\
		\hline
		\textbf{Tầng 5 - 27} & khối văn phòng & 2041     & 0        & 170 \bigstrut\\
		\hline
	\end{tabular}%
	\label{tab:addlabel}%
\end{table}%


\subsubsection{Tính toán thông số cần thiết cho hệ thống Sprikler}
Lưu lượng cần thiết lấy từ nguồn cấp nước cơ bản để hệ thống làm việc:
\begin{equation*}
	Q = I_{b} \times F
\end{equation*}

\break
Trong đó:
\begin{itemize}
	\item $I_{b}$ - Cường độ phun tiêu chuẩn 4,8 lít/m$^2$.phút.
	\item $F$ - Diện tích bảo vệ cùng một lúc khi hệ thống làm việc 12 m$^2$.
\end{itemize}
\begin{equation*}
	\Rightarrow Q = 4.8 \times 200 = 9.6 (l/s)
\end{equation*}

\subsection{HỆ THỐNG CHỮA CHÁY HỌNG NƯỚC VÁCH TƯỜNG}
-- Hệ thống chữa cháy họng nước vách tường được thiết kế chung với mạng đường ống hệ thống Sprinkler. Đây là hệ thống chữa cháy bán tự động, Công trình được sử dụng cuộn vòi D50 -- L = 30 m, lăng phun có đường kính miệng d = 13mm với lưu lượng phun
là 2,5 l/s, số vòi phun cho một đám cháy xẩy ra đồng thời là 2 vậy lưu lượng cần thiết là 5l/s. Đường ống đến các họng nước được rẽ nhánh từ trục chính tại các tầng có đường kính là D150 mm, hoặc D100 mm.

-- Cuộn vòi phải được chấp thuận và phải tương đương với cuộn vòi đã được chấp thuận bởi cơ quan Phòng cháy chữa cháy địa phương. Áp lực làm việc của cuộn vòi trong điều kiện bình thường phải đạt 10bar.

-- Trừ khi có chỉ định khác, tâm họng nước đặt ở độ cao cách mặt sàn 1,25 m. Toàn bộ họng nước đặt trong hộp chữa cháy đặt chìm trong tường, những nơi hộp chữa cháy nằm ở vị trí vách kính, tường bêtông thì hệ thống ống và hộp đi nổi bên ngoài và ống được sơn màu đỏ. (chi tiết bố trí, lắp đặt xem bản vẽ thiết kế).

-- Van góc đường kính D50mm. Áp suất làm việc 16 bars. Các van góc phải đạt tiêu chuẩn an toàn PCCC về khớp nối, van chữa cháy.

-- Bảng hướng dẫn sử dụng được để ở vị trí ngay sát với cuộn vòi ở vị trí chính diện dễ thấy.

\subsection{LỰA CHỌN MÁY BƠM CHỮA CHÁY}
\subsubsection{Áp lực công tác bơm}
\begin{equation*}
	H_{ct} = H_{hh} + H_{td} + \sum H
\end{equation*}

Trong đó:
\begin{itemize}
	\item $H_{hh}$: chiều cao mực nước thấp nhất trong bể chứa đến vòi chữa cháy cao và xa nhất là 110m.
	\item $H_{td}$: áp lực tự do cần thiết cho hệ thống đầu phun là 10m.
	\item $\sum H$: tổng tổn thất áp lực từ miệng hút máy bơm đến vòi nước chữa cháy cao và xa nhất (bao gồm tổn thất áp lực theo chiều dài và tổn thất cục bao qua các thiết bị van, tê, cút, ống vải, lăng phun …) là 90m.
\end{itemize}

Như vậy, $H_{ct}$ có kết quả là:
\begin{equation*}
	\begin{split}
		H_{ct} &= 110 + 10 + 90\\ 
		&= 210 m
	\end{split}
\end{equation*}

\subsubsection{Lưu lượng của bơm}
\vspace{-0.5cm}
\begin{equation*}
	Q_{ct} = 48 + 1 = 49 = 176.4 m^3/h
\end{equation*}

\subsubsection{Chọn bơm}
Từ các thông số trên, chọn bơm như sau: Q = 185 m$^3$/h, H = 210m (gồm 2 bơm chữa cháy, 1 bơm dự phòng và 1 bơm bù áp).
\begin{figure}[H]
	\centering
	\caption{Sơ đồ nguyên lý hệ thống chữa cháy toà nhà}
	\includegraphics[width=\textwidth]{pccc.png}
\end{figure}

\section{ĐẦU BÁO KHÓI}
-- Là thiết bị giám sát trực tiếp, phát hiện ra dấu hiệu khói để chuyển các tín hiệu khói về trung tâm xử lý. Thời gian các đầu báo khói nhận và truyền thông tin đến trung tâm báo cháy không quá 30s. Mật độ môi trường từ 15\% đến 20\%. Nếu nồng độ của khói trong môi trường tại khu vực vượt qua ngưỡng cho phép (10\% -20\%) thì thiết bị sẽ phát tín hiệu báo động về trung tâm để xử lý.

-- Các đầu báo khói thường được bố trí tại các phòng làm việc, hội trường, các kho quỹ, các khu vực có mật độ không gian kín và các chất gây cháy thường tạo khói trước.

\break
-- Đầu báo khói được chia làm 2 loại như sau:
\subsection{ĐẦU BÁO KHÓI DẠNG ĐIỂM}
-- Được lắp tại các khu vực mà phạm vi giám sát nhỏ, trần nhà thấp (văn phòng, chung cư ...).
\begin{enumerate}
	\item Đầu báo khói Ion : Thiết bị tạo ra các dòng ion dương và ion âm chuyển động, khi có khói, khói sẽ làm cản trở chuyển động của các ion dương và ion âm, từ đó thiết
	bị sẽ gởi tín hiệu báo cháy về trung tâm xử lý.
	\item Đầu báo khói quang (photo): Thiết bị bao gồm một cặp đầu báo (một đầu phát tín hiệu, một đầu thu tín hiệu) bố trí đối nhau, khi có khói xen giữa 2 đầu báo, khói sẽ làm cản trở đường truyền tín hiệu giữa 2 đầu báo, từ đó đầu báo sẽ gởi tín hiệu báo cháy về trung tâm xử lý.
\end{enumerate}

\subsection{ĐẦU BÁO KHÓI DẠNG BEAM}
-- Gồm một cặp thiết bị được lắp ở hai đầu của khu vực cần giám sát. Thiết bị chiếu phát chiếu một chùm tia hồng ngoại, qua khu vực thuộc phạm vi giám sát rồi tới một thiết bị nhận có chứa một tế bào cảm quang có nhiệm vụ theo dõi sự cân bằng tín hiệu của chùm tia sáng. Đầu báo này hoạt động trên nguyên lý làm mờ ánh sáng đối nghịch với nguyên lý tán xạ ánh sáng (cảm ứng khói ngay tại đầu báo).

-- Đầu báo khói loại Beam có tầm hoạt động rất rộng (15m x 100m), sử dụng thích hợp tại những khu vực mà các loại đầu báo khói quang điện tỏ ra không thích hợp, chẳng hạn như tại những nơi mà đám khói tiên liệu là sẽ có khói màu đen.

-- Hơn nữa đầu báo loại Beam có thể đương đầu với tình trạng khắc nghiệt về nhiệt độ, bụi bặm, độ ẩm quá mức, nhiều tạp chất,... Do đầu báo dạng Beam có thể đặt đằng sau cửa sổ có kiếng trong, nên rất dễ lau chùi, bảo quản.

-- Đầu báo dạng Beam thường được lắp trong khu vực có phạm vi giám sát lớn, trần nhà quá cao không thể lắp các đầu báo điểm (các nhà xưởng...).

\section{ĐẦU BÁO NHIỆT}
-- Đầu báo nhiệt là loại dùng để dò nhiệt độ của môi trường trong phạm vi bảo vệ, khi nhiệt độ của môi trường không thỏa mãn những quy định của các đầu báo nhiệt do nhà sản xuất quy định, thì nó sẽ phát tín hiệu báo động gởi về trung tâm xử lý.

-- Các đầu báo nhiệt được lắp đặt ở những nơi không thể lắp được đầu báo khói (nơi chứa thiết bị máy móc, Garage, các buồng điện động lực, nhà máy, nhà bếp,...).

\begin{figure}[H]
	\centering
	\includegraphics[width=0.7\textwidth]{pccc_daubaonhiet.jpg}
	\caption{Đầu báo nhiệt}
\end{figure}

\subsection{ĐẦU BÁO NHIỆT CỐ ĐỊNH}
-- Là loại đầu báo bị kích hoạt và phát tín hiệu báo động khi cảm ứng nhiệt độ trong bầu không khí chung quanh đầu báo tăng lên ở mức độ nhà sản xuất quy định (57$^{\circ}$, 70$^{\circ}$, 100$^{\circ}$...).
\subsection{ĐẦU BÁO NHIỆT GIA TĂNG}
-- Là loại đầu báo bị kích hoạt và phát tín hiệu báo động khi cảm ứng hiện tượng bầu không khí chung quanh đầu báo gia tăng nhiệt độ đột ngột khoảng 9$^{\circ}$C/phút.

\section{ĐẦU BÁO LỬA}
-- Là thiết bị cảm ứng các tia cực tím phát ra từ ngọn lửa, nhận tín hiệu, rồi gởi tín hiệu báo động về trung tâm xử lý khi phát hiện lửa.

-- Được sử dụng chủ yếu ở các nơi xét thấy có sự nguy hiểm cao độ, những nơi mà ánh sáng của ngọn lửa là dấu hiệu tiêu biểu cho sự cháy (ví dụ như kho chứa chất lỏng dễ cháy).

-- Đầu báo lửa rất nhạy cảm đối với các tia cực tím và đã được nghiên cứu tỉ mỉ để tránh tình trạng báo giả. Đầu dò chỉ phát tín hiệu báo động về trung tâm báo cháy khi có 2 xung cảm ứng tia cực tím sau 2 khoảng thời gian, mỗi thời kỳ là 5s.

\break
\section{CÔNG TẮC KHẨN}
\begin{wrapfigure}[10]{l}{0.4\textwidth}
	\vspace{-1cm}
	\centering
	\includegraphics[width=0.48\textwidth]{CONG-TAC-KHAN.jpg}
	\caption{Công tắc vuông}
\end{wrapfigure}

-- Được lắp đặt tại những nơi dễ thấy của hành lang cầu thang để sử dụng khi cần thiết. Thiết bị này cho phép người sử dụng chủ động truyền thông tin báo cháy bằng cách nhấn hoặc kéo vào công tắc khẩn, báo động khẩn cấp cho mọi người đang hiện diện trong khu vực đó được biết để có biện pháp xử lý hỏa hoạn và di chuyển ra khỏi khu vực nguy hiểm bằng các lối thóat hiểm.

-- Gồm có các loại công tắc khẩn như sau:
\begin{enumerate}[leftmargin=2.2cm]
	\item Khẩn tròn, vuông.
	\item Khẩn kính vỡ.
	\item Khẩn giật.
\end{enumerate}

\section{CHUÔNG BÁO CHÁY}
-- Được lắp đặt tại phòng bảo vệ, các phòng có nhân viên trực ban, hành lang, cầu thang hoặc những nơi đông người qua lại nhằm thông báo cho những người xung quanh có thể biết được sự cố đang xảy ra để có phương án xử lý, di tản kịp thời.

-- Khi xảy ra sự cố hỏa hoạn, chuông báo động sẽ phát tín hiệu báo động giúp cho nhân viên bảo vệ nhận biết và thông qua thiết bị theo dõi sự cố hỏa hoạn (bảng hiển thị phụ) sẽ biết khu vực nào xảy ra hỏa hoạn, từ đó thông báo kịp thời đến các nhân viên có trách nhiệm phòng cháy chữa cháy khắc phục sự cố hoặc có biện pháp xử lý thích hợp.

\begin{figure}[H]
	\centering
	\includegraphics[width=0.5\textwidth]{bell_firing.png}
	\caption{Chuông báo cháy}
\end{figure}

\section{CÒI BÁO CHÁY}
-- Có tính năng và vị trí lắp đặt giống như chuông báo cháy, tuy nhiên còi được sử dụng khi khoảng cách giữa nơi phát thông báo đến nơi cần nhận thông báo báo động quá xa.
\begin{figure}[H]
	\vspace{-1cm}
	\centering
	\includegraphics[width=0.45\textwidth]{coibaochay.jpg}
	\caption{Còi báo cháy}
\end{figure}

\section{ĐÈN BÁO CHÁY}
-- Có công dụng phát tín hiệu báo động, mỗi lọai đèn có chức năng khác nhau và được lắp đặt ở tại các vị trí thích hợp để phát huy tối đa tính năng của thiết bị này.

-- Gồm có các lọai đèn:
\subsection{ĐÈN CHỈ LỐI THOÁT HIỂM}
- Đèn chỉ dẫn thoát nạn Exit lắp đặt ở độ cao 2,5m. Đèn chỉ dẫn thoát nạn Exit được cấp nguồn AC 220V. Để duy trì đèn Exit luôn luôn sáng có 1 nguồn DC dự phòng tự động chuyển nguồn khi nguồn AC không có. Tuỳ từng vị trí lắp đặt, các đèn Exit phải có mũi tên chỉ hướng thoát nạn.

\begin{wrapfigure}[10]{r}{0.4\textwidth}
	\vspace{-0.5cm}
	\centering
	\includegraphics[width=0.3\textwidth]{exit.jpg}
	\caption{Đèn chỉ lối thoát hiểm}
\end{wrapfigure}
- Hệ thống chỉ dẫn lối thoát nạn và chiếu sáng sự cố chỉ dẫn cho người thoát ra khỏi công trình nhanh chóng khi có sự cố cháy xảy ra nhằm giảm thương vong về con người. Đèn hoạt động theo nguyên tắc: Khi chưa có sự cố mất điện, đèn hoạt động nhờ nguồn điện cấp từ tủ điện ánh sáng của tầng 220VAC. Ngoài ra các hộp đèn chỉ dẫn thoát nạn (EXIT) đều có nguồn ắc quy dự phòng, tự cung cấp điện cho đường chỉ dẫn khi mất hai nguồn trên trong một thời gian tối thiểu là 2 giờ.

- Đèn chiếu sáng sự cố lắp đặt trên lối thoát nạn: hành lang, cầu thang, chỗ khó di chuyển, chỗ rẽ. Khoảng cách không quá 30m.

- Đèn chiếu sáng sự cố có cường độ chiếu sáng ban đầu là 10 lux, và cường độ chiếu sáng tại bất kỳ điểm nào trên lối thoát nạn không nhỏ hơn 1 lux.

\subsection{ĐÈN BÁO CHÁY}
Được đặt bên trên công tắc khẩn của mỗi tầng. Đèn báo cháy sẽ sáng lên mỗi khi công t khẩn hoạt động, đồng thời đây cũng là đèn báo khẩn cấp cho những người hiện diện trong tòa nhà được biết. Điều này có ý nghĩa quan trọng, vì trong lúc bối rối do sự cố cháy, thì người sử dụng cần phân biệt rõ ràng công tác khẩn nào còn hiệu lực được kích hoạt máy bơm chữa cháy.
\begin{figure}[H]
	\centering
	\includegraphics[width=0.6\textwidth]{denbaochay.jpg}
	\caption{Đèn báo cháy}
\end{figure}





	\fancyhead[L]{\leftmark}
%trang bìa
\chapmoi{IoT HVAC - KHÁI QUÁT \& TRIỂN KHAI}
\section{KHÁI QUÁT VỀ IoT CHO HVAC}
-- Một trong những ứng dụng phổ biến của IoT chính là Nhà Thông Minh, cụ thể hơn là trong nền công nghiệp Sửi ấm, thông gió \& điều hoà không khí (HVAC). Theo báo cáo của Zion Market Research, thị trường HVAC thông minh sẽ đạt giá trị 28.3 tỉ đô vào năm 2025 so với 8.3 tỉ đô vào năm 2018. Sự kết hợp giữa HVAC và hệ thống IoT trước tiên sẽ cung cấp cho khách hàng một dịch vụ điều khiển toàn diện. Tiến thêm một bước nữa, hệ thống còn có khả năng cung cấp những tiên đoán về hệ thống dựa trên lịch sử.

-- IoT sẽ thay đổi ngành HVAC trong việc giúp sử dụng năng lượng hiệu quả hơn, thông minh hơn, và kết nối các thiết bị trong toà nhà tốt hơn. Ưu điểm lớn nhất của IoT HVAC chính là các thiết bị cảm biến sẽ rẻ hơn do tính chất open source.

\section{CẤU TRÚC HỆ THỐNG IoT}
-- Một hệ thống IoT bao gồm các thiết bị end users, các node và các gateway, cuối cùng là cloud (mạng internet). 
\begin{figure}[H]
	\centering
	\includegraphics[width=.9\textwidth]{HVAC_IOT_TOPOLOGY}
	\caption{Sơ đồ hệ thống IoT HVAC}
\end{figure}
-- Ở hình trên, chúng ta có thể thấy hệ thống bao gồm những thành phần sau:
\begin{enumerate}
	\item Bên trái ngoài cùng chính là các thiết bị cảm biến, các thiết bị giám sát.
	\item Phía trên là các Node (còn được gọi là các nút - được sử dụng như 1 bộ trung tâm kiểm soát các thiết bị bên dưới, có vai trò như một giáo viên chủ nhiệm trong lớp học).
	\item Trên nữa là GATEWAY, thiết bị này có nhiệm vụ đưa lượng dữ liệu lấy được từ các Node lên trên cloud để xử lý.
	\item Cuối cùng và quan trọng nhất chính là cloud (còn được gọi là Internet), thành phần này đóng vai trò xử lý lượng dữ liệu được đưa lên bằng cách phân tích dữ liệu dựa vào các thuật toán mà người dùng yêu cầu. Sau đó trả về dưới dạng các loại dữ liệu visual như biểu đồ.
\end{enumerate}

-- Từ những điều ở trên, ta có thể thấy rằng để triển khai một hệ thống IoT cho HVAC đòi hỏi một kỹ năng kèm theo đó là lượng kiến thức vô cùng khổng lồ đến từ nhiều lĩnh vực. 
\section{KHÓ KHĂN KHI TRIỂN KHAI IoT HVAC}
-- IoT là một hệ thống đa ngành vì vậy rất khó để triển khai một cách toàn diện nếu chỉ biết duy nhất một lĩnh vực.

-- Những lĩnh vực khác yêu cầu ngoài kiến thức về hệ thống HVAC:
\begin{enumerate}
	\item Hệ thống nhúng, thiết kế mạch PCB (mạch in).
	\item Cơ sở hạ tầng mạng.
	\item Data Analysis, AI.
	\item Hệ thống HVAC.
\end{enumerate}

-- Những kiến thức từ những lĩnh vực trên đều được áp dụng một cách trọn vẹn vào trong hệ thống IoT nói chung và HVAC IoT nói riêng. 

\subsection{HỆ THỐNG NHÚNG, THIẾT KẾ PCB}
-- Đối với hệ thống IoT thì lập trình nhúng, mạch PCB chính là cốt lõi của hệ thống vì IoT đòi hỏi phải làm việc với các loại cảm biến, các giao tiếp, giao thức, chống nhiễu cho thiết bị. Nếu không có kiến thức về lĩnh vực này sẽ rất khó để vận hành, bảo trì cũng như xác định lỗi của hệ thống. 

-- Khó khăn khi làm việc với hệ thống nhúng chính là độ ổn định của hệ thống vì nó liên quan trực tiếp với điều khiển - nguồn của thiết bị. Nếu độ ổn định không cao, sẽ làm hệ thống chạy sai lệch gây ra hao tổn năng lượng. Thẩm chí, có thể làm hư hỏng cả thiết bị. Chỉ cần vài dòng code sai từ bộ điều khiển, máy nén có thể chạy một cách quá đà gây ra hư hỏng máy nén. 

-- Thiết kế mạch PCB kém có thể gây ra tình trạng nhiễu trên mạch, gửi/nhận tín hiệu rất khó khăn. Thường xuyên bị rớt các gói. Đỉnh điểm chính là việc không bảo vệ được mạch PCB, linh kiện chết thường xuyên, phải thay thế và hàn lại (đôi khi phải bỏ nguyên cả mạch mà thay mới). 

\subsection{CƠ SỞ HẠ TẦNG MẠNG}
-- Cơ sở hạ tầng mạng đóng vai trò như một mạng lưới giao thông giữa các Node với Gateway, và giữa Gateway với Cloud. Một cơ sở hạ tầng mạng được thiết kế tốt sẽ giúp hệ thống hoạt động một cách trơn tru và xuyên suốt. Đặc biệt, nếu có kiến thức vững trong lĩnh vực sẽ giảm thiểu rủi ro khi hệ thống hoạt động Offline. Lĩnh vực này giúp tăng tính ổn định cho hệ thống và là ``huyết mạch'' của toàn hệ thống.

-- Khó khăn lớn nhất đối với lĩnh vực này chính là tính bảo mật. Mức độ bảo mật phải đảm bảo đủ tốt vì nó liên quan tới điều khiển và giám sát. Nếu bảo mật kém sẽ khiến hệ thống dễ bị hack. Và hậu quả tệ nhất là hacker có thể toàn quyền kiểm soát HVAC của toà nhà và hơn thế nữa.

\subsection{DATA ANALYSIS, ARTIFICIAL INTELLIGENCE}
-- Nếu ví hệ thống nhúng như những cơ bắp, cơ sở hạ tầng mạng như huyết mạch của một cơ thể. Thì Data analysis chính là ``bộ não'' của cơ thể. Lĩnh vực này sẽ đóng vai trò như 1 chuyên gia phân tích \& đưa quyết định đến các hệ thống điều khiển, giám sát. Ở lĩnh vực này, không chỉ giúp chúng ta đưa quyết định mà hơn nữa còn đưa ra được các dự đoán tương lai khi mà chúng ta cần xem xét hoặc ước lượng. Đỉnh cao nhất chính là việc nó dự đoán thói quen sử dụng của toà nhà và đưa ra các biện pháp tiết kiệm điện hiệu quả nhất (ví dụ như biết chính xác khi nào nên bật điều hoà trong phòng nào đó để khi có người sử dụng, nó sẽ ở nhiệt độ thích hợp, tránh mở quá muộn gây bất tiện cho người dùng hoặc quá sớm gây lãng phí điện năng).

--Phân tích dữ liệu cũng được áp dụng một cách triệt để khi nó có thể lấy dữ liệu từ các toà nhà khác để làm một cơ sở tham khảo; cũng như giúp hệ thống giao tiếp giữa các toà nhà. Điều này cực kỳ hữu ích khi các toà nhà có thể biết được thông tin thời tiết của vùng xunh quanh mình (như mưa, hướng gió, lốc, bão, v.v...) từ đó tạo nên một mạng lưới dựa báo thời tiết. 

\subsection{HỆ THỐNG HVAC}
-- Bản thân kiến thức để xây dựng 1 hệ thống HVAC IoT cũng đã là một thử thách khi nó đòi hỏi người thiết kế phải có một sự am hiểu tốt về HVAC, về những cảm biến cần lắp đặt và những địa điểm nên lắp đặt cảm biến. Đó còn chưa kể đến việc họ phải là những người đặt ra thuật toán để xử lý dữ liệu, tìm được những dữ liệu tốt (dữ liệu có nghĩa) và loại bỏ đi những dữ liệu làm nhiễu database.

$\Rightarrow$ Như vậy để thiết kế một hệ thống IoT HVAC hoàn chỉnh, đòi hỏi phải kết hợp rất nhiều kiến thức từ những lĩnh vực khác nhau. Sẽ rất khó cho một nhóm hoàn thành điều này mà là sự đòi hỏi ở một đơn vị có số lượng lớn nhân viên đến từ những lĩnh vực nêu trên.

\section{TRIỂN KHAI IoT HVAC TRONG PHẠM VI LUẬN VĂN NÀY}
-- Như đã nói ở trên, để một người trong lĩnh vực làm được hoàn chỉnh điều này sẽ tốn rất nhiều thời gian và công sức. Nên trong phạm vi luận văn này, chúng em sẽ chỉ làm mô phỏng một phần, cụ thể là mô phỏng việc lấy dữ liệu và thể hiện nó lên thành visual cho người dùng. Bỏ qua việc thiết kế mạch in, bảo mật \& xử lý dữ liệu. 

\subsection{MÔ TẢ DỰ ÁN}
-- Sử dụng một cảm biến nhiệt độ \& độ ẩm phổ thông - DHT11. Để gửi tín hiệu về một bộ vi điều khiển - ESP8266 Node MCU - thông qua giao thức One-wire. Bộ vi điều khiển sau đó sẽ sử dụng sóng RF - module nRF24L01+ - để đẩy dữ liệu lên một bộ vi điều khiển trung tâm đóng vai trò là Gateway - ESP32 Devkit.

-- Từ vi điều khiển trung tâm, dữ liệu sẽ được gửi lên database trên cloud - dịch vụ database Firebase của Google.

-- Tiếp đến dữ liệu sẽ được 1 đoạn script viết bằng Python \& sử dụng module firebase để lấy dữ liệu xuống và vẽ biểu đồ nhiệt độ - thời gian theo thời gian thực.

-- Dữ liệu được đoạn script Python lấy xuống sẽ được lưu vào trong một file Excel để sử dụng khi cần tra cứu sau này.

\subsection{CÁC LINH KIỆN ĐIỆN TỬ ĐƯỢC SỬ DỤNG TRONG LUẬN VĂN}
\subsubsection{Thông số kỹ thuật các linh kiện}
-- Cảm biến nhiệt độ \& độ ẩm DHT11:
\begin{figure}[H]
	\centering
	\includegraphics[width=.3\textwidth]{dht11.jpg}
	\caption{DHT11}
\end{figure}

$\circledast$ Thông số kỹ thuật của cảm biến:
\begin{itemize}
	\item Điện áp hoạt động: 5VDC.
	\item Chuẩn giao tiếp: TTL, 1 wire.
	\item Khoảng đo độ ẩm: 20\%-80\%RH sai số $ \pm $ 5\%RH.
	\item Khoảng đo nhiệt độ: 0-50°C sai số $ \pm $ 2°C.
	\item Tần số lấy mẫu tối đa 1Hz (1 giây / lần).
	\item Kích thước: 28mm x 12mm x 10m.
\end{itemize}

-- Vi điều khiển ESP8266 Node MCU:
\begin{figure}[H]
	\centering
	\includegraphics[width=.5\textwidth]{esp8266_nodemcu.jpg}
	\caption{ESP8266 NODE MCU}
\end{figure}

$\circledast$ Đặc tính nổi bật vi điều khiển ESP8266:
\begin{itemize}
	\item Tích hợp 2 nút nhấn
	\item Tích hợp chip chuyển usb – uart CP2102
	\item Full IO : 10 GPIO, 1 Analog, 1SPI , 2 UART, 1 I2C/I2S, PWM,v.v….
	\item Được hỗ trợ bởi cộng đồng lớn mạnh Nodemcu.
\end{itemize}

$\circledast$ Thông số kỹ thuật của ESP8266:
\begin{itemize}
	\item Tương thích các chuẩn wifi : 802.11 b/g/n
	\item Hỗ trợ: Wi-Fi Direct (P2P), soft-AP
	\item Tích hợp TCP/IP protocol stack
	\item Tích hợp TR switch, balun, LNA, power amplifier and matching network
	\item Tích hợp bộ nhân tần số, ổn áp, DCXO and power management units
	+25.dBm output power in 802.11b mode
	\item Power down leakage current of <10uA
	\item Integrated low power 32-bit CPU could be used as application processor
	\item SDIO 1.1/2.0, SPI, UART
	\item STBC, 1×1 MIMO, 2×1 MIMO
	\item A-MPDU \& A-MSDU aggregation \& 0.4ms guard interval
	\item Wake up and transmit packets in < 2ms
	\item Dòng tiêu thụ ở Standby Mode < 1.0mW (DTIM3)
\end{itemize}

-- NRF24L01 kết hợp với đế ra chân:
\begin{figure}[H]
	\centering
	\includegraphics[width=.6\textwidth]{nrf24l01_derachan.jpg}
	\caption{NRF24L01 + đế ra chân}
\end{figure}

$\circledast$ Thông số kỹ thuật của đế ra chân NRF24L01 với ic ổn áp:
\begin{itemize}
	\item Điện áp: Áp ngõ vào: 4.8VDC – 8.7VDC
	\item Module giảm áp: AMS1117 – 3V3
	\item Kích thước: 16mm * 19mm * 11.3mm
	\item Trọng lượng: 2g
\end{itemize}

-- NRF24L01:
\begin{figure}[H]
	\centering
	\includegraphics[width=.5\textwidth]{nrf24l01.jpg}
	\caption{NRF24L01}
\end{figure}

$\circledast$ Thông số kỹ thuật của NRF24L01:
\begin{itemize}
	\item Điện thế hoạt động: 1.9V – 3.6V
	\item Có sẵn anthena sứ 2.4GHz.
	\item Truyền được 100m trong môi trường mở với 250kbps baud.
	\item Tốc độ truyền dữ liệu qua sóng: 250kbps to 2Mbps.
	\item Tự động bắt tay (Auto Acknowledge).
	\item Tự động truyền lại khi bị lỗi (auto Re-Transmit).
	\item Multiceiver – 6 Data Pipes.
	\item Bộ đệm dữ liệu riêng cho từng kênh truyền nhận: 32 Byte separate TX and RX FIFOs.
	\item Các chân IO đều chịu được điện áp vào 5V.
	\item Lập trình được kênh truyền sóng trong khoảng 2400MHz đến 2525MHz (chọn được 125 kênh).
	\item Thứ tự chân giao tiếp : GND,VCC,CS,CSN,SCK,MOSI,MISO,IQR
\end{itemize}

-- ESP32:
\begin{figure}[H]
	\centering
	\includegraphics[width=.4\textwidth]{esp32.jpg}
	\caption{ESP32}
\end{figure}

$\circledast$ Thông số kỹ thuật của ESP32:
\begin{itemize}
	\item Model: Wifi BLE SoC ESP32 ESP-WROOM-32E (Tương thích hoàn toàn với phiên bản cũ ESP-WROOM-32 hiện đã ngưng sản xuất).
	\item ESP32-D0WD-V3 embedded, Xtensa® dual-core 32-bit LX6 microprocessor, up to 240 MHz
	\item Điện áp sử dụng: 3~3.6VDC
	\item Dòng điện sử dụng: ~90mA
	\item 448 KB ROM for booting and core functions
	\item 520 KB SRAM for data and instructions
	\item 16 KB SRAM in RTC
	\item Kiểu Antenna: PCB
	\item WiFi
	\begin{itemize}
		\item 802.11b/g/n
		\item Bit rate: 802.11n up to 150 Mbps
		\item A-MPDU and A-MSDU aggregation
		\item 0.4 µs guard interval support
		\item Center frequency range of operating channel: 2412 ~ 2484 MHz 
	\end{itemize}
	\item Bluetooth
	\begin{itemize}
		\item Bluetooth V4.2 BR/EDR and Bluetooth LE specification
		\item Class-1, class-2 and class-3 transmitter
		\item AFH
		\item CVSD and SBC
	\end{itemize}
	\item Hardware
	\begin{itemize}
		\item Interfaces: SD card, UART, SPI, SDIO, I2C, LED PWM, Motor PWM, I2S, IR, pulse counter, GPIO, capacitive touch sensor, ADC, DAC, Two-Wire Automotive Interface (TWAI®, compatible with ISO11898-1)
		\item 40 MHz crystal oscillator
		\item 4 MB SPI flash
	\end{itemize}
	\item Kích thước: 18 x 25.5 x 3.1mm
\end{itemize}

\subsubsection{Sơ đồ đi dây của mạch}
-- Thiết kế hệ thống này bao gồm 2 mạch:
\begin{enumerate}
	\item Mạch thu tín hiệu (bao gồm ESP32 nhận tín hiệu từ nRF24L01).
	\item Mạch phát tín hiệu (bao gồm ESP8266 phát tín hiệu đi bằng nRF24L01).
\end{enumerate}
\begin{figure}[H]
	\centering
	\includegraphics[width=.8\textwidth]{topology_iot_hvac}
	\caption{Text}
\end{figure}

-- Ở đây, mạch thu tín hiệu đóng vai trò như 1 Gateway để từ đó truyền dữ liệu lên database.
\begin{figure}[H]
	\centering
	\includegraphics[width=1.2\textwidth]{ESP32-nRF}
	\caption{Text}
\end{figure}
\begin{itemize}
	\item AA
\end{itemize}
-- 

\lstinputlisting[language=Python, firstline=2, lastline=26]{Kivy_Android_1.py}





	\chapmoi{THIẾT KẾ DÀN BAY HƠI}


\section{TÍNH TOÁN NHIỆT - THIẾT KẾ - TRỞ KHÁNG}


\section{MÔ HÌNH 3D SỬ DỤNG \textit{INVENTOR}}


	\chapmoi{TÍNH TOÁN - THIẾT KẾ BÌNH NGƯNG}
\section{PHÂN TÍCH LỰA CHỌN THIẾT BỊ NGƯNG TỤ}
Bình ngưng dùng để truyền nhiệt lượng của tác nhân lạnh ở nhiệt độ cao cho môi chất giải nhiệt. Hơi đi vào bình ngưng thường là hơi quá nhiệt, cho nên trước tiên nó phải được làm lạnh đến nhiệt độ bão hòa, rồi đến quá trình ngưng tụ, sau cùng là bị quá lạnh vài độ trước khi ra khỏi bình ngưng.

Theo cách giải nhiệt của bình ngưng có thể chia ra làm 4 nhóm sau:
\begin{itemize}
	\item Bình ngưng giải nhiệt bằng nước.
	\item Bìng ngưng giải nhiệt bằng không khí.
	\item Bình ngưng giải nhiệt bằng nước và không khí.
	\item Bình ngưng giải ngiệt bằng môi chất sôi trong máy lạnh bậc thang hoặc giải nhiệt bằng sản phẩm công nghệ.
\end{itemize}

Trong công trình HA NOI PLAZA chọn bình ngưng giải nhiệt bằng nước vì bình ngưng giải nhiệt bằng nước giải nhiệt tốt hơn bình ngưng giải nhiệt bằng không khí.
\section{TÍNH TOÁN THIẾT KẾ BÌNH NGƯNG}
\subsection{CÁC THÔNG SỐ BAN ĐẦU}
-- Hiệu Entanpy vào và ra khỏi bình ngưng:
\begin{equation*}
	\Delta i = q_{k} = i_{2} - i_{3'} = 435.65 - 255.95 = 179.70(kJ/kg)
\end{equation*}

-- Phụ tải nhiệt của bình ngưng:
\begin{equation*}
    Q_{k} = G\times q_{k} = 3.49 \times 179.70 = 626.33(kw)
\end{equation*}

-- Nhiệt độ bên ngoài tra theo TCVN 5687-2010” Tiêu chuẩn Thiết Kế Thông Gió Và Điều Hòa Không Khí.
$t_{N}$ = 36$^{\circ}$C , $\varphi_{N}$ = 74\% , $t_{u}$ = 31.7$^{\circ}$C 

Tra trên ẩm đồ I-d ta có:
\begin{itemize}
	\item $i_{1}$ =  108.7(kJ/kg)
	\item $d_{1}$ =  28.2(g/kg)
\end{itemize}

-- Nhiệt độ nước vào bình ngưng:
\begin{equation*}
	t_{w1} = t_{u} + (3\div 4)
\end{equation*}

Ta chọn $t_{w1}$ = 35$^{\circ}$C

-- Theo tài liệu [2], ta chọn độ chênh nhiệt độ giữa nước vào và ra bình ngưng là: 
\begin{equation*}
	\Delta t_{w} = (3\div 6)
\end{equation*}

Ta chọn: $\Delta t_{w}$ = 5$^{\circ}$C

-- Nhiệt độ nước ra bình ngưng:
\begin{equation*}
	\begin{split}
		 t_{w2} &= t_{w1} + \Delta t_{w} \\
		 &= 40^{\circ}C
	\end{split}
\end{equation*}

-- Nhiệt độ ngưng tụ: $t_{k} = t_{w1} + (3\div 5)$ =  43$^{\circ}$C

Bề mặt truyền nhiệt của bình ngưng được chọn là chùm ống đồng có cánh bố trí so le, trên mặt sàn có các thông số:
\begin{itemize}
	\item Đường kính cánh D = 0.02 (m)
	\item Đường kính ngoài của ống d{\scriptsize ng} = 0.018 (m)
	\item Đường kính trong của ống d{\scriptsize tr} = 0.016 (m)
	\item Bước cánh S{\scriptsize c} = 0.00118 (m)
	\item Bước ống S = 0.026 (m)
	\item Bề dày đầu cánh $\delta_{d}$ = 0.0002 (m)
	\item Bề dày chân cánh $\delta_{c}$ = 0.0003 (m)
\end{itemize}

\subsection{TÍNH TOÁN}
Các thông số khác của chùm ống:

- Diện tích bề mặt đứng của 1m ống có cánh:
\begin{equation*}
	\begin{split}
		F_{d} &= \dfrac{\pi\times (D^2 - d^2_{ng})}{2\times S_{c}}\\
		&=\dfrac{3.14 \times (0.02^2 - 0.016^2)}{2 \times 0.00118} = 0.1011(m^2/m)
	\end{split}
\end{equation*}

-- Diện tích bề mặt ngang của 1m ống:
\begin{equation*}
\begin{split}
		F_{n} &= \pi\times d_{ng}\times \left(1-\dfrac{\delta_{c}}{S_{c}}\right) + \dfrac{\pi\times D\times \delta_{d}}{S_{c}}\\
		&= 3.14 \times 0.018 \times (1-\dfrac{0.0003}{0.00118}) + \dfrac{3.14 \times 0.02 \times 0.0002}{0.00118} = 0.053 (m^2/m)
\end{split}
\end{equation*}

-- Diện tích mặt ngoài 1m ống có cánh:
\begin{equation*}
	\begin{split}
		F_{ng} &= F_{n} + F_{tr}\\
		&= 0.053 + 0.1011 = 0.154(m^2/m)
	\end{split}
\end{equation*}

-- Diện tích bề mặt trong của 1m ống có cánh:
\begin{equation*}
	\begin{split}
		F_{tr} &= \pi\times d_{tr}\\
		&= 3.14 \times 0.016 = 0.05024(m^2/m)
	\end{split}
\end{equation*}	

-- Hệ số làm cánh:
\begin{equation*}
	\begin{split}
		\beta &= \dfrac{F_{ng}}{F_{tr}}\\
		&= \dfrac{0.154}{0.05024}= 3.06
	\end{split}
\end{equation*}	

-- Nhiệt độ trung bình nước giải nhiệt bình ngưng:
\begin{equation*}
	\begin{split}
		t_{w} &= \dfrac{t_{w1} + t_{w2}}{2}\\
		&= \dfrac{35 + 40}{2} = 37.5^{\circ}C
	\end{split}
\end{equation*}	

Các tính chất nhiệt vật lý của nước giải nhiệt trong bình ngưng ở nhiệt độ trung bình $t_{w}$ = 37.5$^{\circ}$C

Tra theo phụ lục 28 tài liệu [2] trang 613 ta được:

Ở nhiệt độ $ t_{w} $ = 37.5$^{\circ}$C
\begin{itemize}
	\item $\gamma$ = 0.0000007028$(m^2/s$)
	\item $\rho$ =  993.25$(kg/m^3$)
	\item $Pr$ =  	5.087
	\item $C_{w}$ =  4.174(kJ/kg)	
	\item $\lambda$ = 0.6224(W/mK)
\end{itemize}


Tính chất vật lý của R134a ở nhiệt độ ngưng tụ $ t_{k} $ = 43$^{\circ}$C:
\begin{itemize}
	\item $\gamma$ = 0.00000023$(m^2/s$)
	\item $\rho$ = 56.22$(kg/m^3$)
	\item $\lambda$ = 1.17(W/mK)
\end{itemize}

Phụ tải nhiệt bình ngưng: Q{\scriptsize k} = 626.33(kw)

Lượng nước giải nhiệt đi qua bình ngưng:
\begin{equation*}
	\begin{split}
		G_{w} &= \dfrac{Q_{k}}{C_{w}\times \Delta t_{w}}\\
		&=\dfrac{626.33}{4.174 \times 5} = 30.01(kg/s)
	\end{split}
\end{equation*}

Để tính toán truyền nhiệt về phía nước ta chọn sơ bộ vận tốc nước chuyển động trong ống là $\omega$ = 2 (m/s) ở nhiệt độ là 37$^{\circ}$C:
\begin{equation*}
	\begin{split}
		n_{1} &= \dfrac{4\times G_{w}}{\pi\times\rho\times\omega\times d_{tr}^2}\\
		&= \dfrac{4 \times 30.01}{3.14 \times 993.25 \times 2 \times 0.016^2}= 75.18
	\end{split}
\end{equation*}

Số ống trong một đường nước là 76

Tính lại vận tốc nước theo $n_{1}$ = 76

Ta có:
\begin{equation*}
	\begin{split}
		\omega &= \dfrac{4\times G_{w}}{\pi\times\rho\times n_{1}\times d_{tr}^2}\\
		&= \dfrac{4 \times 30.01}{3.14 \times 993.25 \times 75.18 \times 0.016^2} = 1.999  \approx 2 (m/s)
	\end{split}
\end{equation*}

Hệ số Reynolds:
\begin{equation*}
	\begin{split}
		Re &= \dfrac{\omega\times d_{tr}}{\nu}\\
		&= \dfrac{2 \times 0.016}{0.7028 \times 10^-6} = 45532.16
	\end{split}
\end{equation*}

Từ kết quả này ta có $ Re $ =  ta suy ra đây là chế độ chảy .

Hệ số Nusselt:
\begin{equation*}
	\begin{split}
		Nu &= 0.021\times Re^{0.8} \times Pr^{0.43} \times\varepsilon_{1}\\
		&= 0.021 \times 45532.16^0.8 \times 5.087^0.43 \times 1 = 225.24
	\end{split}
\end{equation*}

Hệ số toả nhiệt về phía nước:
\begin{equation*}
	\begin{split}
		\alpha_{w} &= \dfrac{Nu\times \lambda}{d_{tr}}\\
		&=\dfrac{225.24 \times 0.6224}{0.016} = 8761.98(W/m^2K)
	\end{split}
\end{equation*}

Độ chênh nhiệt độ trung bình logarit:
\begin{equation*}
	\begin{split}
		\theta_{m} &= \dfrac{t_{w2} - t_{w1}}{\ln\left(\dfrac{t_{k} - t_{w1}}{t_{k} - t_{w2}}\right)}\\
		&=\dfrac{40-35}{\ln\dfrac{43-35}{43-40}}=11.74^{\circ}C
	\end{split}
\end{equation*}

Phương trình xác định mật độ dòng nhiệt về phía nước:
\begin{equation*}
	\begin{split}
		q_{w} &= \dfrac{\theta_{m} - \theta_{a}}{\dfrac{1}{\alpha_{w}}+\Sigma\dfrac{\delta_{i}}{\lambda_{i}}}\\
		&=3741.29 \times (11.74 - \theta_{a})
	\end{split}
\end{equation*}

$\Sigma\dfrac{\delta_{i}}{\lambda_{i}}$ : tổng trở nhiệt của lớp cáu và vách ống. Theo tài liệu [2] trang 257, vật liệu vách ống là đồng thì ta chọn $\Sigma\dfrac{\delta_{i}}{\lambda_{i}}$ = 0.26 $\times 10^-3

Để có thể xác định mật độ dòng nhiệt qtr, cần lưa chọn sơ bộ kết cấu của bình ngưng:

Sơ bộ ta có thể chọn: $\theta_{a}$ = 0.3 $\theta_{m}$

Khi đó:
\begin{equation*}
	\begin{split}
		q'_{tr} &= 3741.29 \times (11.74 - 0.3 \times 11.74)=30748.22(w/m^2)
	\end{split}
\end{equation*}

Ống được bố trí trên mặt sàn theo các cạnh của hình lục giác đều và trên các đỉnh của hình tam giác đều. Nên số ống bố trí theo đường chéo lớn nhất của lục giác ngoài cùng là m có thể được xác định sơ bộ theo công thức:

Theo tài liệu [2] ta được:
\begin{equation*}
	\begin{split}
		m &= 0.75\times\sqrt[3]{\dfrac{Q_{k}}{q'_{tr}\times s\times d_{tr}\times k}} \\
		&= 0.75 \times \sqrt[3]{\dfrac{626.33 \times 10^3}{30748.22 \times 0.026 \times 0.016 \times 8}}
	\end{split}
\end{equation*}

Trong đó:
\begin{itemize}
	\item s: Bước ống (s = 1.3\times D = 1.3 \times 0.02 = 0.026).\\
`	\item D: Đường kính cánh (D = 0.02).
\end{itemize}

Ta chọn: $k = l/d = $ 8

Ta suy ra:
\begin{equation*}
	\begin{split}
		m &= 0.75\times\sqrt[3]{\dfrac{Q_{k}}{q'_{tr}\times s\times d_{tr}\times k}} \\
		&=0.75 \times \sqrt[3]{\dfrac{626.33 \times 10^3}{30748.22 \times 0.026 \times 0.016 \times 8}} = 13.72
	\end{split}
\end{equation*}

Từ kết quả trên ta chọn: $m$ = 15

Vậy số hàng ống theo chiều đứng là: $n_{z} = 15 = $ và $\dfrac{n_{z}}{2} = $ \dfrac{15}{2}$ = 7.5

Hệ số toả nhiệt ngưng tụ R134a tính theo bề mặt trong của ống được xác định theo công thức:
\begin{equation*}
	\begin{split}
		\alpha_{a} &= 0.72\times\sqrt[4]{\dfrac{\Delta h\times \rho\times\lambda^3\times g}{\nu\times d_{ng}}}\times\left(\dfrac{n_{z}}{2}\right)^{-0.167}\times\beta\times\theta^{-0.25}_{a}\times\Psi_{c} \\
	\end{split}
\end{equation*}

Trong đó:
\begin{itemize}
	\item $\Delta h = q_{k} = $ 179.70(kJ/kg)
	\item $\Psi_{c}$: hệ số tính đến các điều kiện khác khi ngưng tụ trên đoạn ống có cánh.
\end{itemize}
\begin{equation*}
	\Psi_{c} = 1.3\times\dfrac{F_{d}}{F_{ng}}\times E^{0.75}\times\left(\dfrac{d_{ng}}{h'}\right)^{0.25} + \dfrac{F_{d}}{F_{ng}}
\end{equation*}

Với:
\begin{itemize}
	\item $E$: hiệu suất của cánh (E = 1, ống đồng có cánh).
	\item $h'$: chiều cao quy ước của cánh.
\end{itemize}
\begin{equation*}
	\begin{split}
		h' &= \dfrac{\pi}{4}\times\left(\dfrac{D^2 - d_{ng}^2}{D}\right)\\
		&=\dfrac{\pi}{4} \times \dfrac{0.02^2-0.018^2}{0.02}=0.002983(m)
	\end{split}
\end{equation*}

Ta suy ra:
\begin{equation*}
\begin{split}
		\Psi_{c} &= 1.3\times\dfrac{F_{d}}{F_{ng}}\times E^{0.75}\times\left(\dfrac{d_{ng}}{h'}\right)^{0.25} + \dfrac{F_{d}}{F_{ng}}\\
		&=1.3 \times \dfrac{0.1011}{0.154} \times 1^{0.75} \times (\dfrac{0.018}{0.002983})^{0.25} + \dfrac{0.1011}{0.154}= 1.6816
\end{split}
\end{equation*}

Mật độ dòng nhiệt về phía R134A:
\begin{equation*}
	\begin{split}
		\alpha_{a} &= 0.72\times\sqrt[4]{\dfrac{\Delta h\times \rho\times\lambda^3\times g}{\nu\times d_{ng}}}\times\left(\dfrac{n_{z}}{2}\right)^{-0.167}\times\beta\times\theta^{-0.25}_{a}\times\Psi_{c} \\
		&=0.72\times\sqrt[4]{\dfrac{\Delta h\times \rho\times\lambda^3\times g}{\nu\times d_{ng}}}\times\left(\dfrac{n_{z}}{2}\right)^{-0.167}\times\beta\times\theta^{-0.25}_{a}\times\Psi_{c}
	\end{split}
\end{equation*}

Ta có: 
\begin{equation*}
\begin{split}
		q_{a} &= \alpha_{a}\times\theta_{a}\\
		&=37073.930\theta_{a}
\end{split}
\end{equation*}

Ta có hệ phương trình để xác định $q_{tr}$:
\begin{equation*}
	\begin{cases}
		q_{w} &= 3741.29\times(11.74 - \theta_{a}) \\
		q_{a} &= 37073.930\times\theta_{a}^{0.75}
	\end{cases}
\end{equation*}

Giải phương trình bằng phương pháp lặp:
\begin{equation*}
	\begin{split}
		q_{tr} &= \dfrac{(x - 1)\times q_{tr}^{x} + \theta_{m}\times B^{x}}{x\times q_{tr}^{x-1} + \dfrac{B^{x}}{A}}\\
	\end{split}
\end{equation*}

Trong đó:
\begin{itemize}
\begin{multicols}{3}
	\item $x = 1/k = $ 1.33
	\item $A = $ 3741.29
	\item $B = $ 37073.93
	\item $\theta_{m} = $ 11.74
	\item $q'_{tr} = $ 30748.22
\end{multicols}
\end{itemize}

Ta suy ra:
\begin{equation*}
	\begin{split}
	q_{tr1} &= \dfrac{(x - 1)\times q_{tr}^{x} + \theta_{m}\times B^{x}}{x\times q_{tr}^{x-1} + \dfrac{B^{x}}{A}}\\
	& = \dfrac{1.33-1 \times 30748.22^{1.33} + 11.74 \times 37073.93^{1.33}}{1.33 \times 30748.22^{1.33-1}+\dfrac{37073.93^{1.33}}{3741.29}}=39856.012(W/m^2)\\
	q_{tr2} &= \dfrac{(x - 1)\times q_{tr}^{x} + \theta_{m}\times B^{x}}{x\times q_{tr}^{x-1} + \dfrac{B^{x}}{A}}\\
	& = \dfrac{1.33-1 \times 39856.012^{1.33} + 11.74 \times 37073.93^{1.33}}{1.33 \times 39856.012^{1.33-1}+\dfrac{37073.93^{1.33}}{3741.29}}=39808.994(W/m^2)\\
	\end{split}
\end{equation*}

Ta thấy sai số 0,00118\% nên ta có $q_{tr}$ = 39808.994$(W/m^2$)

Diện tích truyền nhiệt bề mặt trong của ống:
\begin{equation*}
	F_{tr} = \dfrac{Q_{k}}{q_{tr}} = \dfrac{626.33 \times 10^3}{39808.994}=15.73(m^2)
\end{equation*}

Tổng chiều dài ống trong bình ngưng:
\begin{equation*}
	L = \dfrac{F_{tr}}{\pi\times d_{tr}} = \dfrac{15.73}{3.14 \times 0.016}= 313.16(m)
\end{equation*}

Sơ bộ ta đã tính và chọn m = 15 vậy tổng số ống là:
\begin{equation*}
	\begin{split}
		n &= 0.75\times (m^2 - 1) + 1\\
		&=0.75 \times (15^2-1)+1 = 169(ống)
	\end{split}
\end{equation*}

Số đường nước trong bình ngưng:
\begin{equation*}
	z = \dfrac{n}{n_{1}} = \dfrac{169}{75.18} = 2.25 
\end{equation*}

Chọn $z$ = 3

Khi đó: $n = z\times n_{1} = $ 3 \times 75.18=225.53( ống ) 



Để sử dụng phần dưới bình ngưng làm bình chứa chúng ta phải bỏ bớt 2 hàng ống dưới cùng.

Số ống bỏ đi là:
\begin{equation*}
	\begin{split}
		n' &= i\times \dfrac{m + 1}{2} + \sum_{n=1}^{n=i-1}n_{i}\\
		&=  2 \times \dfrac{15+1}{2}+1 = 17(ống)
	\end{split}
\end{equation*}

Với $i$ là số hàng ống bỏ đi.

Vậy số hàng ống còn lại là: $n'' = n - n' = $ 169-17 = 152(ống)

Chiều dài ống:
\begin{equation*}
	l = \dfrac{L}{n} = \dfrac{313.16}{152}= 2.06(m)
\end{equation*}

Bước ống:
\begin{equation*}
	S = 1.3\times D = 1.3 \times 0.02 = 0.026(m)
\end{equation*}

Đường kính mặt sàng:
\begin{equation*}
	D = m\times S = 15 \times 0.026 = 0.39(m)
\end{equation*}

Tỉ số:
\begin{equation*}
	K = \dfrac{l}{D} =\dfrac{2.06}{0.39}=5.28
\end{equation*}

Vậy k = 5.28 nằm trong khoảng cho phép (4$\div$8)

\section{TÍNH TOÁN THUỶ ĐỘNG CHO BÌNH NGƯNG}
Ngoài việc tính toán truyền nhiệt trong bình ngưng, ta còn tính trở lực của nước lạnh khi qua bình ngưng. Theo công thức 9.25 trang 358 TL[2]:

Trở lực về phía nước qua bình ngưng:
\begin{equation*}
	\begin{split}
		\Delta P &= \left(\lambda\times\dfrac{L}{d_{tr}} +\xi_{v} + 1 + \dfrac{\xi_{v} + 1}{z}\right)\times \dfrac{\omega^2\times\rho}{2}\times z\\
		&=  
	\end{split}
\end{equation*}

Trong đó:
\begin{itemize}
	\item $\xi_{v}$ - hệ số trở lực cục bộ khi nước vào ống: $\xi_{v}$ = 0.5
	\item $L$ - chiều dài thân bình giữa 2 mặt sàng: $L$ = 2.06(m)
	\item $d_{tr}$ - đường kính trong của ống: $d_{tr}$ = 0.016(m)
	\item $z$ - số đường nước trong thiết bị: $z$ = 3
	\item $\omega$ - vận tốc dòng nước trong ống: $\omega$ = 2 m/s
	\item $\rho$ - khối lượng riêng của nước lạnh: $\rho$ = 999.71 $kg/m^3$
	\item $\lambda$ - hệ số ma sát	
\end{itemize}

Do nước trong ống ở trạng thái chảy rối, nên đối với các ống đồng hệ số ma sát được tính như sau:
\begin{equation*}
	\begin{split}
		\lambda &= \dfrac{0.3164}{Re^{0.25}}\\
		&=  \dfrac{0.3164}{45532.16^{0.25}}=0.0217
	\end{split}
\end{equation*}

Vậy trở lực của nước qua bình ngưng:
\begin{equation*}
	\begin{split}
		\Delta P &= \left(\lambda\times\dfrac{L}{d_{tr}} +\xi_{v} + 1 + \dfrac{\xi_{v} + 1}{z}\right)\times \dfrac{\omega^2\times\rho}{2}\times z\\
		&=  (0.0217 \times \dfrac{2.06}{0.016} + 0.5 + 1 + \dfrac{0.5+1}{3}) \times \dfrac{2^2 \times 999.71}{2} \times 3 = 28540.69(Pa)
	\end{split}
\end{equation*}

\section{TÍNH BỀN CHO BÌNH NGƯNG}
\subsection{TÍNH TOÁN BỀN CHO THÂN BÌNH}
Thiết bị ngưng tụ trong hệ thống điều hoà không khí là thiết bị chịu áp lực phía cao áp. Do đó, ta phải tính toán bền cho thiết bị để đảm bảo an toàn cho thiết bị khi vận hành…

Do kết cấu của bình ngưng dạng hình trụ, vì thế chịu được áp lực đều. Bề dày của thân hình trụ S được chọn phải thỏa mãn điều kiện sau:
\begin{equation*}
	S \geq \dfrac{P_{R}\times D_{tr}}{2\times [\sigma]\times\varphi_{d} - P_{R}} + C
\end{equation*}

Trong đó:
\begin{itemize}
	\item $P_{R}$ - áp suất tính toán của thiết bị, MPa. Theo bảng 10.1 trang 360 TL[2] ta chọn: $P_{R}$ = 16 bar = 1.6 MPa.
	\item $[\sigma]$ - ứng suất cho phép của kim loại chế tạo thân bình, MPa. Theo bảng 10.2 trang 361 TL[2], chọn vật liệu chế tạo thân bình ngưng là thép cacbon chất lượng thường CCT38, với nhiệt độ tính toán của vách là: t = 42 $^{\circ}$C có $[\sigma]$ = 138.35(MPa)
	\item $D_{tr}$ - đường kính trong của thân bình ngưng: $D_{tr}$ = 0.39(m)
	\item $\varphi_{d}$ - hệ số bền mối hàn dọc, $\varphi_{d}$ = 0.8
	\item $C$ - chiều dày bổ sung, mm
\end{itemize}
\begin{equation*}
	C = C_{1} + C_{2} + C_{3}
\end{equation*}
\begin{itemize}
	\item $C_{1}$ - phần bề dày bổ sung để bù cho sự ăn mòn khi tiếp xúc với các	chất độc hại $C_{1}$ = 0,001 m
	\item $C_{2}$ - chiều dày bổ sung đề bù dung sai âm bề dày $C_{4}$ = 0,001 m
	\item $C_{3}$ - phần bề dày bổ sung do bề dày thân bình bị mỏng đi trong quá trình gia công kéo, dập , uốn… $C_{3}$ = 0,001m
\end{itemize}

Vậy:
\begin{equation*}
	S \geq \dfrac{P_{R}\times D_{tr}}{2\times [\sigma]\times\varphi_{d} - P_{R}} + C =\dfrac{1.6 \times 0.39}{2 \times 138.35 \times 0.8 - 1.6} + 0.003 = 0.0058
\end{equation*}

Ta chọn: $S$ = 0.006(m)

Bình ngưng có kích thước như sau:
\begin{itemize}[label={$\diamond$}]
	\item $D_{tr}$ = 0.39(m)
	\item $D_{ng} = D_{tr} + 2\times S =$ 0.39 + 2 \times  0.006$ = 0.4020(m)
	      
\end{itemize}

\subsection{TÍNH TOÁN BỀ DÀY MẶT SÀNG}
Bề dày mặt sàng $S_{m}$ phải đảm bảo có thể núc được ống và phải thõa mãn điều kiện:
\begin{equation*}
	S_{m} \geq 0.5\times D_{E}\times \sqrt{\dfrac{|P_{O} - P_{R}|}{[\sigma]}} + C
\end{equation*}

Trong đó:
\begin{itemize}
	\item $P_{R}$ - áp suất tính toán của thiết bị, MPa. Theo bảng 10.1 trang 360 TL[2] ta chọn $P_{R}$ = 16 bar = 1.6  MPa.
	\item $P_{O}$ - áp suất tính toán bên trong ống $P_{O}$ = 1.5 bar = 0.15 MPa.
	\item $[\sigma]$ - ứng suất cho phép của kim loại chế tạo thân bình, MPa. Theo bảng 10.2 trang 361 TL[2], chọn vật liệu chế tạo thân bình ngưng là thép cacbon chất lượng thường CCT38, với nhiệt độ tính toán của vách là t = 36$^{\circ}$C có $[\sigma]$ = 138.35(MPa)	
	\item $D_{E}$ - đường kính của vòng tròn có thể chứa được trong diện tích không có ống lớn nhất trên mặt sàng $D_{E}$ = 0.115(m)
	\item $C$ - bề dày bổ sung $C$ = 0.003(m)
\end{itemize}

Vậy:
\begin{equation*}
\begin{split}
		S_{m} &\geq 0.5\times D_{E}\times \sqrt{\dfrac{|P_{O} - P_{R}|}{[\sigma]}} + C\\
		&\geq 0.5 \times 0.115 \times \sqrt{\dfrac{|0.15-1.6}{138.35}} + 0.003 = 0.0089(m)
\end{split}
\end{equation*}

Ta chọn chiều dày mặt sàng: $S_{m}$ = 0.009(m)

\subsection{TÍNH TOÁN BỀN CHO ĐÁY}
Với thiết bị ngưng tụ dạng hình trụ, ta sử dụng đáy cong có thể tháo mở được để lắp ghép với bích ở 2 đầu thân hình trụ. Ta chọn loại đáy cong cho thiết bị là đáy cong hình tròn không bo mép (hình 10-4 c, trang 370 TL[2]).

Bề dày loại đáy tròn được xác định như sau: (công thức trang 370 TL[2])
\begin{equation*}
	S_{n} \geq \dfrac{P_{R}\times R}{2\times \phi_{d}\times[\sigma]} + C
\end{equation*}

Trong đó:
\begin{itemize}
<<<<<<< HEAD
	\item $D_{tr}$ = 
	\item $H_{tr} = 0.25\times D_{tr} = $
	\item $R$ bán kính của đáy cong, m.
=======
	\item $D_{tr}$ = 0.39(m)
	\item $H_{tr} = 0.25\times D_{tr} = $ 0.25 \times 0.39$ = 0.0975(m)
	\item $R$ bán kính của nắp cong, m.
>>>>>>> 00efaa603391f1ea0298b70e3c82f7b57ed6939a
	
	$R = \dfrac{D_{tr}^2}{4\times H_{tr}} = $ \dfrac{0.39^2}{4 \times 0.0975} $= 0.39(m) 
	\item $ \varphi_{d} $ - hệ số bền mối hàn dọc, $\varphi_{d}$ = 0.8
	\item $P_{R}$ - áp suất tính toán của thiết bị: $P_{R}$ = 1.6 MPa
	\item $[\sigma]$ - ứng suất cho phép của kim loại chế tạo đáy $[\sigma]$ = 138.35 MPa 
	\item $C$ - chiều dày bổ sung $C$ = 0.03(m)
\end{itemize}

Vậy: 
\begin{equation*}
	S_{n} \geq \dfrac{P_{R}\times R}{2\times \phi_{d}\times[\sigma]} + C = \dfrac{1.6 \times 0.39}{2 \times 0.8 \times 138.35}+0.003 = 0.0058(m)
\end{equation*}

Ta chọn chiều dày đáy: $S_{n}$ = 0.006(m)

	\chapmoi{TÍNH TOÁN - THIẾT KẾ BÌNH BAY HƠI}
\section{GIỚI THIỆU VỀ THIẾT BỊ BAY HƠI}
\subsection{CHỨC NĂNG}
Thiết bị bay hơi là thiết bị chính quan trọng trong hệ thống lạnh, dùng để làm lạnh chất tải lạnh (nước hay dung dịch NaCl, CaCl{\scriptsize 2} ). Các chất tải lạnh này được dẫn vào dàn lạnh và làm lạnh không khí trong không gian cần làm lạnh.

Trong thiết bị bay hơi có sự trao đổi nhiệt giữa tác nhân lỏng và chất tải lạnh từ dàn lạnh trở về. Chất tải lạnh chuyển động trong ống, còn tác nhân lạnh chuyển động bên ngoài ống. Kết quả của sự truyền nhiệt là tác nhân lạnh chuyển thành hơi, chất tải lạnh làm lạnh xuống nhiệt độ cần thiết.

\subsection{PHÂN LOẠI}
Theo cách phân loại về mức độ hoán chỗ của tác nhân trong thiết bị bay hơi, chia làm 2 loại sau:
\begin{itemize}
	\item Thiết bị bay hơi kiểu ngập lỏng.
	\item Thiết bị bay hơi kiểu ngập lỏng nửa.
	\item Thiết bị bay hơi kiểu trực tiếp.
\end{itemize}
\subsection{PHÂN TÍCH \& LỰA CHỌN THIẾT BỊ BAY HƠI}
Thiết bị bay hơi kiểu ngập lỏng: tác nhân lạnh lỏng bao phủ toàn bộ bề mặt trao đổi nhiệt, tác nhân lạnh lỏng được cấp từ phía dưới. Với loại này, nước lạnh chuyển động trong ống, còn tác nhân lạnh lỏng chuyển động bên ngoài ống. Hệ số truyền nhiệt cao.

Thiết bị bay hơi kiểu không ngập lỏng: tác nhân lạnh lỏng chỉ bao phủ một bề mặt trao đổi nhiệt, phần còn lại của bề mặt trao đổi nhiệt dùng để quá nhiệt hơi hút vế máy nén. Ở loại này, tác nhân lạnh lỏng được cấp từ phía trên của thiết bị bay hơi và chuyển động bên ngoài ống, còn nước lạnh chuyển động trong ống. Loại này có hệ số truyền nhiệt cao.

Thiết bị bay hơi kiểu trực tiếp: tác nhân lạnh lỏng chuyển động trong ống, còn nước lạnh chuyển động bên ngoài ống. Với loại này thì tổn thất áp về phía nước nhỏ, lượng tác nhân lạnh nạp vào hệ thống tương đối ít. Nhưng hệ số truyền nhiệt không cao.

Từ sự phân tích ở trên ta chọn thiết bị bay hơi là loại ngập lỏng, do hệ số trruyền nhiệt cao nên kích thước thiết bị sẻ nhỏ ứng với năng suất lạnh lớn. Do công trình cần điều hoà không khí có tổn thất nhiệt là $Q_{O}$ = 539.44 kW nên việc chọn lựa thiết bị trên là hợp lý. Với loại bình bay hơi này, nước chuyển động trong có thể xem như kín nên không khí ít lọt vào hệ thống, do đó giảm được sự ăn mòn thiết bị.

\section{TÍNH TOÁN - THIẾT KẾ BÌNH BAY HƠI}
\subsection{THÔNG SỐ BAN ĐẦU}
Tổng tải lạnh của công trình $Q_{O}$ = 539.44 kW.

Trong hệ thống ta sử dụng thiết bị bay hơi với chất tải lạnh là nước chuyển động trong ống, còn tác nhân lạnh R134a chuyển động bên ngoài ống. Trong thiết bị bay hơi sử dụng ống đồng có làm cánh bên ngoài.

Nhiệt độ nước vào bình bay hơi: $t_{s1}$ = 12$^{\circ}$C

Nhiệt độ nước ra bình bay hơi: $t_{s2}$ = 7$^{\circ}$C

Các thông số về ống đồng sử dụng như sau:
\begin{itemize}
	\item Đường kính trong của ống : $ d_{tr} $ = 0.016 (m)
	\item Đường kính chân cánh : $ d_{ng} $ = 0.019 (m)
	\item Đường kính cánh : $D_{C}$ = 0.022 (m)
	\item Bước cánh : $ S_{C} $ =  0.00118(m)
	\item Bề dày chân cánh : $ \delta_{o} $ = 0.0003(m) 
	\item Bề dày đầu cánh : $ \delta_{d} $ = 0.0002(m)
	\item Bước ống : $S$ = 0,027 (m)
\end{itemize}

Diện tích bề mặt trong của 1m ống:
\begin{equation*}
	F_{tr} = \pi\times d_{tr} = 3.14 \times 0.016 = 0.05024(m^2/m)
\end{equation*}

Diện tích bề mặt đứng của 1m ống:
\begin{equation*}
	\begin{split}
		F_{d} &= \dfrac{\pi\times(D_{C}^2 - d_{ng}^2)}{2\times S_{C}}\\
		&= \dfrac{3.14 \times (0.022^2 - 0.019^2)}{2 \times 0.00118}=0.16365(m^2/m)
	\end{split} 
\end{equation*}

Diện tích mặt ngoài của 1m ống:
\begin{equation*}
	\begin{split}
		F_{ng} &= \pi\times d_{ng}\times\left(1 - \dfrac{\delta_{o}}{S_{C}}\right) + \dfrac{\pi\times D_{C}\times d_{ng}}{S_{C}} \\
		&= 3.14 \times 0.019 \times (1 - \dfrac{0.0003}{0.00118}) + \dfrac{3.14 \times 0.022 \times 0.019}{0.00118}= 0.05620(m^2/m)
	\end{split}
\end{equation*}

Tổng diện tích mặt ngoài ứng với 1m ống:
\begin{equation*}
	F = F_{d} + F_{ng} = 0.16365 + 0.05620 = 0.21985(m^2/m)
\end{equation*}

Hệ số làm cánh:
\begin{equation*}
	\beta = \dfrac{F}{F_{tr}} = \dfrac{0.21985}{0.05024}=4.376
\end{equation*}

\subsection{TÍNH TOÁN VỀ PHÍA NƯỚC}
Nhiệt độ trung bình của nước:
\begin{equation*}
	\overline{t_{w}} = \dfrac{1}{2}\times(t_{w1} + t_{w2}) = \dfrac{1}{2} \times (12+7)=9.5{^\circ}C
\end{equation*}

Tra phụ lục 28 trang 613 TL[2], các tính chất vật lý của nước ta được:
\begin{itemize}
	\item Nhiệt dung riêng : $C_{\rho n}$ = 4.191 kJ/kg.K
	\item Khối lượng riêng : $ \rho_{n} $ = 999.7 kg/m3
	\item Độ nhớt động học : $ \nu_{n} $ = 1.306$\times10^{-6}$ m2/s
	\item Hệ số dẫn nhiệt : $ \lambda_{n} $ = 57.45$\times10^{-2}$ W/m.K
	\item Hệ số Prandtl : $ Pr_{n} $ = 9.52
\end{itemize}

Lưu lượng nước lạnh cần cung cấp cho 1 giây:
\begin{equation*}
	G_{n} = \dfrac{Q_{o}}{C\times\Delta t} = \dfrac{539.44}{4.191 \times 5}=25.74(l/s)
\end{equation*}

Với chất lượng nước ở nước ta không cao, để đảm nước chảy rối thích hợp trong bình ngưng, để giảm tổn thất năng lượng cho bơm, giảm khả năng mài mòn ống thì tốc độ nước không nên chọn quá lớn. Tốc độ nước được chọn nằm trong khoảng (1$\div$2.5m/s).

Cụ thể ta chọn tốc độ nước: $\omega_{n}$ = 2 m/s.

Số ống trong một đường nước:
\begin{equation*}
	n_{1} = \dfrac{4\times G_{n}}{\pi\times\rho_{n}\times\omega_{n}\times d_{tr}^2} = \dfrac{4 \times 25.74}{3.14 \times 999.7 \times 2 \times 0.016^2}=64.07(ống)
\end{equation*}

Chọn số ống là: $n_{1}$  = 65(ống)

Ta xác định lại vận tốc nước chảy trong ống:
\begin{equation*}
	n_{1} = \dfrac{4\times G_{n}}{\pi\times\rho_{n}\times n_{1}\times d_{tr}^2} = \dfrac{4 \times 25.74}{3.14 \times 999.7 \times 65 \times 0.016^2}=1.971(m/s)
\end{equation*}

Trị số Reynold của nước:
\begin{equation*}
	Re = \dfrac{\omega_{n}\times d_{tr}}{\nu_{n}} = \dfrac{1.971 \times 0.016}{1.306 \times 10^{-6}}=24151
\end{equation*}

Do đó chế độ chảy của nước trong ống là chảy rối.

Trị số Nusselt của nước:
\begin{equation*}
	Nu = 0.017\times Re^{0.8}\times Pr^{0.4}\times\left(\dfrac{d_{ng}}{d_{tr}}\right)^{0.18} 
\end{equation*}

Xem nhiệt độ giữa bề mặt vách ống và nước trong ống là không quá lớn, nên xem $\left(\dfrac{Pr_{f}}{Pr_{w}}\right)$ = 1. Do $\dfrac{l}{d} >$ 50, nên xem $\varepsilon_{1}$ = 1, với ống sử dụng là ống thẳng nên $\varepsilon_{R}$ = 1.

Do đó:
\begin{equation*}
	Nu = 0.017\times x^{0.8}\times y^{0.4}\times\left(\dfrac{u}{i}\right)^{0.18} = 0.017 \times 24151^{0.8} \times 9.52^{0.4} \times (\dfrac{0.019}{0.016})^{0.18}=138.58
\end{equation*}

Hệ số tỏa nhiệt về phía nước giải nhiệt:
\begin{equation*}
	\alpha_{s.tr} = \dfrac{Nu\times\lambda}{d_{tr}} =\dfrac{138.58 \times 0.5745}{0.016}=4975.75(W/m^2K)
\end{equation*}

\subsection{TÍNH TOÁN VỀ PHÍA TÁC NHÂN LẠNH}
Với nhiệt sôi của tác nhân $t_{o}$ = 5$^{\circ}$C, nhiệt độ trung bình logarit của bình bay hơi được xác định như sau:
\begin{equation*}
	\theta_{m} = \dfrac{t_{s1} - t_{s2}}{\ln{\dfrac{t_{s1} - t_{s0}}{t_{s2} - t_{s0}}}} = \dfrac{12-7}{\ln{\dfrac{12-5}{7-5}}}=9.19{^\circ}C
\end{equation*}

Tổng trở nhiệt của lớp cáu:
$\Sigma \dfrac{\delta_{i}}{\lambda_{i}} = (0.12 \div 0.15)\times 10^{-3} m^2.K/W$

Ta chọn tổng nhiệt trở để tính toán: $\Sigma \dfrac{\delta_{i}}{\lambda_{i}} = (0.13)\times 10^{-3} m^2.K/W$

Mật độ dòng nhiệt về phía nước:
\begin{equation*}
	\begin{split}
		q_{s.tr} &= \dfrac{t_{V} - t_{W}}{\dfrac{1}{\alpha_{s.tr}} + \Sigma \dfrac{\delta_{i}}{\lambda_{i}}} = \dfrac{\theta_{m} - \theta}{\dfrac{1}{\alpha_{s.tr}} + \Sigma \dfrac{\delta_{i}}{\lambda_{i}}}\\
		&= \dfrac{9.19 - \theta}{\dfrac{1}{4975.75}+0.00013}\\
		&= 2868.36 \times (9.19 - \theta)
	\end{split}
\end{equation*}

Mật độ dòng nhiệt của tác nhân R134A khi sôi trên bề mặt có cánh được tính quy về phía nước (bề mặt trong) như sau:
\begin{equation*}
	q_{a.tr} = 335\times p_{o}^{0.5}\times\theta^2\times\varepsilon_{n}\times\varepsilon_{d}\times\dfrac{F}{F_{tr}}
\end{equation*}

\pagebreak
Trong đó:
\begin{itemize}
	\item $p_{o}$ - áp suất sôi của tác nhân ở $t_{o}$ =5 $^{\circ}$C.
	\item $\varepsilon_{n}$ - hệ số xét đến ảnh hưởng của chùm ống có cánh, trong bình bay hơi làm lạnh nước ta lấy $\varepsilon_{n}$ = 1.
	\item $\varepsilon_{d}$ - hệ số xét đến ảnh hưởng của dầu bôi trơn hoà tan trong tác nhân R143A, ta lấy $\varepsilon_{d}$ = 0.82.
\end{itemize}

Vậy:
\begin{equation*}
	\begin{split}
		q_{a.tr} &= 335\times p_{o}^{0.5}\times\theta^2\times\varepsilon_{n}\times\varepsilon_{d}\times\dfrac{F}{F_{tr}}\\
		&= 335 \times 314.8^{0.5} \times \theta^2 \times 1 \times 0.82 \times 4.376\\
		&= 21328.45 \times \theta^2
	\end{split}
\end{equation*}

Ta có hệ phương trình:
\begin{equation*}
	\begin{cases}
		q_{s.tr} &= 2868.36\times(9.19 - \theta)\\
		q_{a.tr} &= 21328.45\times\theta^2
	\end{cases}
\end{equation*}
\begin{equation*}
	\begin{split}
		21328.45\times\theta^2 &= 2868.36\times(9.19 - \theta)\\
		\Rightarrow \theta &= 1.0465^{\circ}C\\
		q_{tr} &=23358.11(W/m^2)
	\end{split}
\end{equation*}

Diện tích truyền nhiệt bề mặt trong:
\begin{equation*}
	F_{tr} = \dfrac{Q_{o}}{q_{tr}} = \dfrac{539.44 \times 10^{3}}{23358.11}=23.09(m^2)
\end{equation*}

Tổng chiều dài ống của bình bốc hơi:
\begin{equation*}
	L_{\Sigma} = \dfrac{F_{tr}}{\pi\times d_{tr}} = \dfrac{23.09}{3.14 \times 0.016}=459.68(m) 
\end{equation*}

Chọn số đường nước z = 3.

Do đó, tổng số ống trong bình bay hơi là: $n = n_{1}\times z = 64.07 \times 3=192.21$(ống)

Chọn cách bố trí ống trong bình bay hơi theo hình lục giác đều. Ta chọn số ống bố trí theo đường chéo lớn nhất là: m = 17 ống

Tổng số ống trong bình bay hơi là:
\begin{equation*}
	n = 0.75\times(m^2 - 1) + 1 = 0.75 \times (17^2-1)+1=217(ống)
\end{equation*}

Vậy còn thiếu 25 ống nữa là đủ 217 ống ta sẽ bố trí chúng ở khoảng trống bình bay hơi.

Chiều dài của một đoạn ống trong bình bay hơi:
\begin{equation*}
	l = \dfrac{L}{n} = \dfrac{459.68}{217}=2.12(m)
\end{equation*}

Đường kính mặt sàng:
\begin{equation*}
	D = m\times S = 17 \times 0.027=0.459(m)
\end{equation*}

Tỉ số: $k = \dfrac{L}{D} = \dfrac{2.12}{0.459}= 4.6$
Ta thấy tỉ số $k$ là chấp nhận được do nằm trong khoảng (3,5$\div$10).
\section{TÍNH TOÁN THUỶ ĐỘNG CHO BÌNH BAY HƠI}
Ngoài việc tính toán truyền nhiệt trong bình bay hơi, ta còn tính trở lực của nước lạnh khi qua bình bay hơi. Theo công thức 9.25 trang 358 TL[2]:
Trở lực về phía nước qua bình bay hơi:
\begin{equation*}
	\Delta P = \left(\lambda\times\dfrac{L}{d_{tr}} +\xi_{v} + 1 + \dfrac{\xi_{v} + 1}{z}\right)\times \dfrac{\omega^2\times\rho}{2}\times z
\end{equation*}
Trong đó:
\begin{itemize}
	\item $\xi_{v}$ - hệ số trở lực cục bộ khi nước vào ống: $\xi_{v}$ = 0.5
	\item $L$ - chiều dài thân bình giữa 2 mặt sàng: $L$ = 2.12(m)
	\item $d_{tr}$ - đường kính trong của ống: $d_{tr}$ = 0.016(m)
	\item $z$ - số đường nước trong thiết bị: $z$ = 3
	\item $\omega$ - vận tốc dòng nước trong ống: $\omega$ = 2 m/s
	\item $\rho$ - khối lượng riêng của nước lạnh: $\rho$ = 999.71 $kg/m^3$
	\item $\lambda$ - hệ số ma sát	
\end{itemize}
Do nước trong ống ở trạng thái chảy rối, nên đối với các ống đồng hệ số ma sát được tính như sau:
\begin{equation*}
	\begin{split}
		\lambda &= \dfrac{0.3164}{Re^{0.25}}\\
		&=  \dfrac{0.3164}{24151^{0.25}}=0.0254
	\end{split}
\end{equation*}
Vậy trở lực của nước qua bình bay hơi:
\begin{equation*}
	\begin{split}
		\Delta P &= \left(\lambda\times\dfrac{L}{d_{tr}} +\xi_{v} + 1 + \dfrac{\xi_{v} + 1}{z}\right)\times \dfrac{\omega^2\times\rho}{2}\times z\\
		&=  (0.0254 \times \dfrac{2.12}{0.016}+0.5+1+ \dfrac{0.5+1}{3} \times \dfrac{2^2 \times 999.71}{2} \times 3) = 32152.40(Pa)
	\end{split}
\end{equation*}
\subsection{TÍNH TOÁN BỀN CHO THÂN BÌNH}
Thiết bị bốc hơi trong hệ thống điều hoà không khí là thiết bị chịu áp lực phía hạ áp. Do đó, ta phải tính toán bền cho thiết bị để đảm bảo an toàn cho thiết bị khi vận hành…
Do kết cấu của bình bốc hơi dạng hình trụ, vì thế chịu được áp lực đều. Bề dày của thân hình trụ S được chọn phải thỏa mãn điều kiện sau:
\begin{equation*}
	S \geq \dfrac{P_{R}\times D_{tr}}{2\times [\sigma]\times\varphi_{d} - P_{R}} + C
\end{equation*}
Trong đó:
\begin{itemize}
	\item $P_{R}$ - áp suất tính toán của thiết bị, MPa. Theo bảng 10.1 trang 360 TL[2] ta chọn: $P_{R}$ = 16 bar = 1.6 MPa.
	\item $[\sigma]$ - ứng suất cho phép của kim loại chế tạo thân bình, MPa. Theo bảng 10.2 trang 361 TL[2], chọn vật liệu chế tạo thân bình ngưng là thép cacbon chất lượng thường CCT38, với nhiệt độ tính toán của vách là: t = 36$^{\circ}$C có $[\sigma]$ = 138.8(MPAa)
	\item $D_{tr}$ - đường kính trong của thân bình ngưng: $D_{tr}$ = 0.459(m)
	\item $\varphi_{d}$ - hệ số bền mối hàn dọc, $\varphi_{d}$ = 0.8
	\item $C$ - chiều dày bổ sung, mm
\end{itemize}
\begin{equation*}
	C = C_{1} + C_{2} + C_{3}
\end{equation*}
\begin{itemize}
	\item $C_{1}$ - phần bề dày bổ sung để bù cho sự ăn mòn khi tiếp xúc với các	chất độc hại $C_{1}$ = 0,001 m
	\item $C_{2}$ - chiều dày bổ sung đề bù dung sai âm bề dày $C_{4}$ = 0,001 m
	\item $C_{3}$ - phần bề dày bổ sung do bề dày thân bình bị mỏng đi trong quá trình gia công kéo, dập , uốn… $C_{3}$ = 0,001m
\end{itemize}
Vậy:
\begin{equation*}
	S \geq \dfrac{P_{R}\times D_{tr}}{2\times [\sigma]\times\varphi_{d} - P_{R}} + C = \dfrac{1.6 \times 0.459}{2 \times 138.8 \times 0.8 - 1.6}+0.003 = 0.006(m)
\end{equation*}
Ta chọn: $S$ = 0.007(m)
Bình bay hơi có kích thước như sau:
\begin{itemize}[label={$\diamond$}]
	\item $D_{tr}$ = 0.459(m)
	\item $D_{ng} = D_{tr} + 2\times S = 0.459 + 2 \times 0.006 = 0.473(m)$
\end{itemize}

\subsection{TÍNH TOÁN BỀ DÀY MẶT SÀNG}
Bề dày mặt sàng $S_{m}$ phải đảm bảo có thể núc được ống và phải thõa mãn điều kiện:
\begin{equation*}
	S_{m} \geq 0.5\times D_{E}\times \sqrt{\dfrac{|P_{O} - P_{R}|}{[\sigma]}} + C
\end{equation*}

Trong đó:
\begin{itemize}
	\item $P_{R}$ - áp suất tính toán của thiết bị, MPa. Theo bảng 10.1 trang 360 TL[2] ta chọn $P_{R}$ = 16 bar = 1.6  MPa.
	\item $P_{O}$ - áp suất tính toán bên trong ống $P_{O}$ = 1.5 bar = 0.15 MPa.
	\item $[\sigma]$ - ứng suất cho phép của kim loại chế tạo thân bình, MPa. Theo bảng 10.2 trang 361 TL[2], chọn vật liệu chế tạo thân bình ngưng là thép cacbon chất lượng thường CCT38, với nhiệt độ tính toán của vách là t = 36$^{\circ}$C có $[\sigma]$ = 138.8(MPa)	
	\item $D_{E}$ - đường kính của vòng tròn có thể chứa được trong diện tích không có ống lớn nhất trên mặt sàng $D_{E}$ = 0.115(m)
	\item $C$ - bề dày bổ sung $C$ = 0.003(m)
\end{itemize}

Vậy:
\begin{equation*}
	\begin{split}
		S_{m} &\geq 0.5\times D_{E}\times \sqrt{\dfrac{|P_{O} - P_{R}|}{[\sigma]}} + C\\
		&\geq 0.5 \times 0.115 \times \sqrt{\dfrac{|0.15-1.6|}{138.8}} + 0.003 = 0.0089(m)
	\end{split}
\end{equation*}

Ta chọn chiều dày mặt sàng: $S_{m}$ = 0.009(m)

\subsection{TÍNH TOÁN BỀN CHO ĐÁY}
Với thiết bị ngưng tụ dạng hình trụ, ta sử dụng đáy cong có thể tháo mở được để lắp ghép với bích ở 2 đầu thân hình trụ. Ta chọn loại đáy cong cho thiết bị là đáy cong hình tròn không bo mép (hình 10-4 c, trang 370 TL[2]).

Bề dày loại nắp tròn được xác định như sau: (công thức trang 370 TL[2])
\begin{equation*}
	S_{n} \geq \dfrac{P_{R}\times R}{2\times \phi_{d}\times[\sigma]} + C
\end{equation*}

\pagebreak
Trong đó:
\begin{itemize}
	\item $D_{tr}$ = 0.459(m)
	\item $H_{tr} = 0.25\times D_{tr} = 0.25 \times 0.39 = 0.11475(m)$
	\item $R$ bán kính của nắp cong, m.
	
	$R = \dfrac{D_{tr}^2}{4\times H_{tr}} = \dfrac{0.459^2}{4 \times 0.11475} = 0.459(m) $
	\item $ \varphi_{d} $ - hệ số bền mối hàn dọc, $\varphi_{d}$ = 0.8
	\item $P_{R}$ - áp suất tính toán của thiết bị: $P_{R}$ = 1.6 MPa
	\item $[\sigma]$ - ứng suất cho phép của kim loại chế tạo đáy $[\sigma]$ = 138.8 MPa 
	\item $C$ - chiều dày bổ sung $C$ = 0.03(m)
\end{itemize}

Vậy: 
\begin{equation*}
	S_{n} \geq \dfrac{P_{R}\times R}{2\times \phi_{d}\times[\sigma]} + C = \dfrac{1.6 \times 0.459}{2 \times 0.8 \times 138.8}+0.003 = 0.006(m)
\end{equation*}

Ta chọn chiều dày đáy: $S_{n}$ = 0.007(m)
	\chapmoi{TÍNH TOÁN - THIẾT KẾ THÁP GIẢI NHIỆT}

\section{NGUYÊN LÝ HOẠT ĐỘNG CỦA THÁP GIẢI NHIỆT}

Nguyên lý hoạt động và quá trình truyền nhiệt xảy ra trong tháp giải nhiệt như sau: Nước nóng sau khi giải nhiệt cho tác nhân lạnh và ra khỏi bình ngưng sẽ được bơm lên tháp rồi sẽ phun qua các lỗ nhỏ giúp quá trình tiếp xúc giữa nước và khog6 khí giải nhiệt được tốt hơn. Các giọt nước này rơi trên các tấm chắn ( giàn tổ ong ) và dưới tác dụng của quạt gió sẽ tạo thành các hạt nhỏ hơn rồi chảy thành từng lớp mỏng trên bề mặt tấm chắn từ trên xuống dưới. Mặt khác không kí từ bên ngoài tháp ( là không khí ẩm chưa bão hòa $\varphi$ < 100\% ) nhờ quạt được hút vào từ phía dưới và ra khỏi tháp ở phía trên. Khi không khí tiếp xúc với nước sẽ thực hiện quá trình trao đổi nhiệt và trao đổi chất. Nước sẽ tỏa nhiệt cho không khí, giảm nhiệt độ và quay về bình chứa ở dưới. Sau đó được bơm về bình ngưng. Qúa trình truyền nhiệt giữa nước và không khí được thực hiền bằng hai phương thức : \\
+ Phương thức thứ nhất là truyền nhiệt bằng đối lưu do có độ chênh lệch nhiệt độ $\Delta_{t}$ giữa nhiệt độ nước $t_{n}$ và nhiệt độ không khí $t_{k}$. Khi $\Delta t$ tăng thì truyền nhiệt đối lưu giữa nước và không khí tăng lên và ngược lại.\\
+ Phương thức thứ hai là truyền nhiệt bằng truyền chất nghĩa là do nước bay hơi vào không. Thực tế trong các tháp giải nhiệt thì sự truyền nhiệt từ nước vào không khí bằng phương thức bay hơi là chủ yếu.\\

\begin{figure}[H]
	\centering
	\includegraphics[width=0.8\textwidth]{thapgiainhiet2}
	\caption{\textbf{Tháp giải nhiệt}}
	\label{thapgiainhiet2}
\end{figure}

\section{CHỌN THÁP GIẢI NHIỆT}
Thông số ban đầu :
\begin{itemize}
	\item Năng suất tháp giải nhiệt : Q = $ Q_{k}$ = 539.44(kW).
	\item Nhiệt độ nước vào tháp : $t_{w1}$ = 40$^{\circ}$C.
	\item Nhiệt độ nước ra khỏi tháp : $t_{w2}$ = 35$^{\circ}$C.
\end{itemize}

Với các thông số trên Catalogue Tháp giải nhiệt, ta chọn tháp giải nhiệt phù hợp : 

Lượng nước giải nhiệt qua bình ngưng:
\begin{equation*}
	\begin{split}
		G_{W2} = \dfrac{Q'_{k}}{C_{W} \times \Delta t_{W}}= \dfrac{539.44}{4.174 \times 5}=25.85(kg/s)
	\end{split}
\end{equation*}

Phương trình cân bằng trong tháp giải nhiệt:
\begin{itemize}
	\item $G_{w2}$= $G_{w} + G_{xả}$.
	\item $G_{w}$ =$ G_{w1} + G' + G''$
	\item $G_{bs}$ = $G' + G'' + G_{x}$
\end{itemize}

Trong đó :
\begin{itemize}
	\item $G_{w}$ : Lượng nước đi vào hệ thống làm mát.
	\item $G_{w1}$ : Lượng nước đi vào bình chứa sau khi được làm mát.
	\item $G_{w2}$ : Lượng nước đi vào bình ngưng.
	\item $G_{x}$ : Lượng nước xả để đảm bảo lượng nước sạch cho hệ thống giải nhiệt.
	\item $G_{bs}$ : Lượng nước bổ sung vào tháp để bù đắp tất cả các tổn thất.
	\item $G'$ : Lượng nước tổn thất do bốc hơi.
	\item $G''$ : Lượng nước tổn thất do gió mang đi.
\end{itemize}

Để đảm bảo các ống dẫn nước giải nhiệt ít bị ăn mòn, độ cứng của nước không nên vượt quá giới hạn cho phép. Ta phải xả đi một phần nước giải nhiệt, theo trang 311 TL[2], tỉ lệ lượng nước xả:
\begin{equation*}
  \begin{split}
  	\dfrac{g_{x}}{G_{w}} = G_{x} > (3 - 4 \% )
  \end{split}	
\end{equation*}

\newpage

Ta chọn: $g_{x}$ = 5\%

Lượng nước do gió mang đi : $G_{x}$ = 0.05 $\times$ $G_{w}$

Lượng nước đi vào hệ thống làm mát: $G_{w2}$ = 1.05 $\times$ $G_{w}$

\begin{equation*}
	 G = \dfrac{G_{w2}}{1.05}= \dfrac{25.85}{1.05}=24.62(kg/s).
\end{equation*}

Lượng nước xả:
\begin{equation*}
	G' = G_{w2} - G = 25.85 - 24.62 = 1.23(kg/s).
\end{equation*}

Lượng nước do gió mang đi : $g''$ = $\dfrac{G''}{G_{w}} = (0.3 - 0.5\%)$.

Ta chọn: g'' = 0.005.

$G''$ = 0.005 $\times$ G = 0.0005 $\times$ 24.62 = 0.1231(kg/s).

Lượng không khí cần thiết qua tháp:
$G_{kk}$ = $\dfrac{Q}{\Delta_{i}}$ (TL 2, CT 8.12 trang 333).

Trong đó:

+ Q : Năng suất tháp giải nhiệt, Q = $Q'_{k}$ = 539.44(kW).

+ $\Delta_{i}$ : Độ chênh lệch của không khí trong tháp, $\Delta_{i}$ = $i_{1} - i_{2}$.  

Để thuận lợi cho việc tính toán ta chọn:
\begin{equation*}
	\Delta_{i} = C_{w} \times \Delta t_{w} = 4.174 \times 5 = 20.87(kg/s).
\end{equation*}

Để cho lượng nước qua tháp và lượng không khí qua tháp bằng nhau: $\dfrac{G_{w}}{G_{kk}} = 1$.

Vậy lương không khí qua tháp :$G_{kk}$ = 25.85(kg/s).

Nhiệt độ trung bình của nước vào ra khỏi tháp:
\begin{equation*}
	\begin{split}
		t_{tb}& = 0.5 \times (t_{w1} + t_{w2})\\ 
		      & = 0.5 \times (40 + 35) = 37.5^{\circ}C.
	\end{split}
\end{equation*}

Các thông số trạng thái không khí :

+ Khi vào tháp, với $t_{1}$ = 36$^{\circ}$C, $\varphi_{1}$ = 74\%.
\begin{itemize}
	\item $d_{1}$ = 0.0282(g/kg).
	\item $i_{1}$ = 108.7(kJ/kg).
\end{itemize}

+ Ở trạng thái bão hòa ứng với $t_{tb}$ = 37.5$^{\circ}$C, $\varphi_{w}$ = 100\%.
\begin{itemize}
	\item $d_{w}$ = 0.0426(g/kg).
	\item $I''_{w}$ = 147.19(kJ/kg).
\end{itemize}

\newpage
+ Khi ra khỏi tháp:

	$i_{2}$ = $i_{1}$ + $\Delta_{i}$ = 108.7 + 20.87 = 129.57(kJ/kg).
	
	$t_{2} = t_{1} + (t_{w} - t_{1}) \times \dfrac{i_{2} - i_{1}}{i''_{w} - i_{1}}$.
	
Trong đó:
\begin{itemize}
	\item $i_{1}$ : Entanpy của không khí vào tháp, $i_{1}$ = 108.7(kJ/kg).
	\item $i_{2}$ : Entanpy của không khí ra tháp, $i_{2}$ = 130.94(kJ/kg).
	\item $i''_{w}$ : Entanpy của không khí bão hòa có nhiệt độ ứng với $t_{tb}$ = 37.5$^{\circ}$C.
	\item $t_{1}$ : Nhiệt độ không khí vào tháp, $t_{1}$ = 36$^{\circ}$C.
	\item $t_{2}$ : Nhiệt độ không khí khỏi tháp
\end{itemize}
 Vậy:
 \begin{equation*}
 	t_{2} = 36 + (37.5 - 36 ) \times \dfrac{129.57 - 108.7}{147.19 - 108.7} = 36.8^{\circ}C.
 \end{equation*}

Tra theo phần mềm:
\begin{itemize}
	\item $i_{2}$ = 130.94(kJ/kg).
	\item $d_{2}$ = 0.03657(kg/kg).
	\item $\varphi_{2}$ = 90\%.
\end{itemize}

Độ chênh entanpy trung bình logarit trong tháp giải nhiệt:
\begin{equation*}
	\Delta_{i} = \dfrac{(i''_{w1} - i_{2}) - (i''_{w2} - i_{1}) }{\ln \dfrac{i''_{w1} - i_{2}}{i''_{w2} - i_{1}}}.
\end{equation*}

Trong đó:
\begin{itemize}
	\item $i_{1}$ : Entanpy của không khí bảo hòa ứng với $t_{1}$ = 36$^{\circ}$C, 108.7(kJ/kg).
	\item $i''_{w1}$ : Entanpy của không khí bảo hòa ứng với $t_{w1}$ = 40$^{\circ}$C,167(kJ/kg).
	\item $i_{2}$ : Entanpy của không khí bảo hòa ứng với $t_{2}$ = 36.8$^{\circ}$C, 130.94(kJ/kg).
	\item $i''_{w2}$: Entanpy của không khí bảo hòa ứng với $t_{w2}$ = 35$^{\circ}$C, 129.62(kJ/kg).
\end{itemize}

\begin{equation*}
	\Delta i_{L} = \dfrac{(167-130.94)-(129.62-108.7)}{\ln \dfrac{167-130.94}{129.62-108.7}} = 27.81(kJ/kg).
\end{equation*}

Chọn dạng bề mặt xối tưới cho tháp giải nhiệt là bề mặt xối tưới loại tổ ong hình gợn sóng với chất liệu giấy có tẩm epôxi, có đặc tính kỹ thuật như sau (TL2 trang 320, bảng 8.2):
\begin{itemize}
	\item $\omega$ : Vận tốc không khí trong bề mặt chính diện, $\omega$ = 3.4(m/s).
	\item $F_{v}$ : Diện tích riêng $F_{v}$ = 640($m^2/m^3$).
	\item $d_{dt}$ : Đường kính tương đương, $d_{dt}$ = 5.35(mm).
	\item $V_{0}$: Thể tích tự do, $V_{0}$ = 0.91($m^3/m^3$).
	\item H : Chiều cao lớp tổ ong, H = 0.25(m).
\end{itemize}

Hệ số bốc hơi đối với bề mặt tổ ong:
\begin{equation*}
	\sigma = 0.284 \times (\omega \rho)^{0.57} \times g_{L}^{0.29} \times \dfrac{H}{d_{dt}}^{-0.515}
\end{equation*}

Trong đó:
\begin{itemize}
	\item $\omega_{\rho}$ : Vận tốc khối của không khí trong bề mặt xối tưới,($Kg/m^2.s$).
	\item $g_{L}$ : Mật độ xối tưới trên 1m chu vi của tiết diện bị thấm nước, (Kg/m.s).
	\item H : chiều cao (chiều dài rãnh) tổ ong, (m).
	\item $d_{dt}$ : Đường kính tương đương của tổ ong, (m).
\end{itemize}

Ta có :  $\omega_{\rho}$ = $\dfrac{G_{k}}{F \times V}$, (TL2, trang 336, CT 8.22).

Trong đó:
\begin{itemize}
	\item $G_{kk}$ : Lưu lượng không khí qua tháp, $G_{kk}$ = 25.85(kg/s).
	\item $V_{0}$ : Thể tích tự do của bề mặt xối tưới, $V_{o}$ = 0.91($m^3/m^3$).
	\item F : Diện tích mặt cắt ngang của tháp, ($m^2$).
\end{itemize}

\newpage
Ta có : F = $\dfrac{G_{w}}{g_{w}}$.(TL2, trang 312, CT 8.5).

Trong đó:
$g_{w}$ : Mật độ xối tưới, đối với tháp làm lạnh bằng quạt gió mật độ xối tưới có thể đạt đến ( 4- 5 )($kg/m^2.s$).

\begin{equation*}
	\begin{split}
		F& = \dfrac{G_{w}}{g_{w}}= \dfrac{24.62}{4.5} = 5.47(m^2).\\
		\omega_{\rho}& = \dfrac{G_{kk}}{F \times V}\\ 
		             & = \dfrac{25.85}{5.47 \times 0.91} = 5.19(kg/m^2.s).\\
		g_{L}& = \dfrac{\omega_{\rho}}{F_{v}}\\
		     & = \dfrac{5.19}{640}=0.008113(kg/m.s) \\ 
		\dfrac{H}{d_{td}} = \dfrac{0.25}{0.00535}= 46.73             
	\end{split}
\end{equation*}

Vậy hệ số bốc hơi của bề mặt xối tưới :
\begin{equation*}
	\sigma = 0.284 \times 5.19^{0.57} \times 0.008113^{0.29} \times 46.73^{-0.515} = 0.02483 (kg/m^2.s)
\end{equation*}

Diện tích bề mặt xối tưới tổ ong của tháp:
\begin{equation*}
	F_{x} = \dfrac{Q}{\sigma \times \Delta_{i}} =\dfrac{G_{kk} \times \Delta_{i}}{\sigma \times \Delta_{iL}} = \dfrac{25.85 \times 20.87}{0.02483 \times 27.87} = 781.44(m^2/s)
\end{equation*}

Thể tích chứa bề mặt xối tưới:
\begin{equation*}
	V = \dfrac{F_{x}}{F_{v}} = \dfrac{781.44}{640} = 1.22(m^3)
\end{equation*}

Chiều cao bề mặt xối tưới
\begin{equation*}
	H = \dfrac{V_{x}}{F}=\dfrac{1.22}{5.47} =0.22(m)
\end{equation*}

Đường kính trong của tháp:
\begin{equation*}
	D_{tr} = \sqrt{\dfrac{4F}{\pi}} = \sqrt{\dfrac{4 \times 5.47}{3.14}} =2.6(m)
\end{equation*}

Chọn đường kính trong là : $D_{tr}$ = 2.6(m)

Diện tích tiết diện bề mặt xối tưới không cho không khí đi qua:
\begin{equation*}
	f = F \times V_{0} = 5.47 \times 0.91 = 4.98(m^2)
\end{equation*}

\newpage
\section{TÍNH KHÍ ĐỘNG CHO THÁP GIẢI NHIỆT}
Để chọn quạt cho tháp ta cần phải tính toán khí động cho tháp nghĩa là ta tính trở lực của không khí qua tháp để từ đó chọn quạt phù hợp.

Vận tốc chuyển động của không khí ở các tiết diện khác nhau của tháp đều có quan hệ với nhau theo phương trình liên túc:

\begin{equation}
	\dfrac{G_{kk}}{\rho} = f1 \times \omega_{1} = f2 \times \omega_{2} = f3 \times \omega_{3} =f4 \times \omega_{4} 
\end{equation}

Trong đó
\begin{itemize}
	\item $f_{1}$ : Diện tích tiết diện tại cửa vào.
	\item Chọn sơ bộ chiều cao cửa gió vào tháp: $h_{1}$ = 0.5(m).
	\item $\Rightarrow$ $f_{1}$ = $\pi \times D_{tr} \times h_{1}$ =$3.14 \times 2.6 \times 0.5$ = 4.145($m^2$)
	\item $f_{2}$ : Diện tích thân thap, $f_{2} = F$ = 5.47($m^2$).
	\item $f_{3}$ : Diện tích ngay tại lớp tổ ong, $f_{3} = f$ = 4.98($m^2$).
	\item $f_{4}$ : Diện tích ngay tại cửa tháp.
\end{itemize}

Chọn sơ bộ đường kính trong cửa ra của tháp:
\begin{itemize}
	\item $D_{4} = 0.6 \times D_{tr} = 0.6 \times 2.6 $= 1.58(m).
	\item $\Rightarrow$ $f_{4} = \dfrac{\pi \times D_{4}^{2}}{4} = \dfrac{3.14 \times 1.58^2}{4}$ = 1.97($m^2$).
\end{itemize}

$\rho$ : Khối lượng riêng của không khí, $\rho$ = 1.1391(Kg/$m^3$) ứng với nhiệt độ không khí trung bình:
\begin{equation}
	t_{tb} = 0.5 \times (t_{1} + t_{2}) = 0.5 \times (40 + 35) = 37.5^{\circ}C
\end{equation}

\newpage

Từ các tiết diện trên ta xác định vận tốc của dòng không khí tại các tiết diện khác nhau  được xác định như sau:
\begin{equation*}
	\begin{split}
		\omega_{1}& = \dfrac{G_{k}}{\rho \times f_{1}}\\ 
				  & = \dfrac{24.62}{1.1391 \times 4.145 } = 5.21(m/s)\\
		\omega_{2}& = \dfrac{G_{k}}{\rho \times f_{2}}\\ 
		& = \dfrac{24.62}{1.1391 \times 5.47 } = 3.95(m/s)\\
		\omega_{3}& = \dfrac{G_{k}}{\rho \times f_{3}}\\ 
		& = \dfrac{24.62}{1.1391 \times 4.98 } = 4.34(m/s)\\
		\omega_{4}& = \dfrac{G_{k}}{\rho \times f_{4}}\\ 
		& = \dfrac{24.62}{1.1391 \times 1.97 } = 10.97(m/s)\\
	\end{split}
\end{equation*}

Tổng trở lực của dòng không khí qua tháp:
\begin{equation*}
	\Delta P_{\sum} = \Delta P_{v} +\Delta P_{n} + \Delta P_{x} + \Delta P_{p} +\Delta P_{e} + \Delta P_{k} + \Delta P_{ra} (TL2, trang 337, CT 8.24)
\end{equation*}

Trở lực tại cửa vào
\begin{equation*}
	\Delta P_{v} = 0.55 \times \rho \times \dfrac{\omega_{1}^{2}}{2} = 0.55 \times 1.1391 \times \dfrac{5.21^2}{2} = 8.52(Pa)
\end{equation*}

Trở lực tại chỗ ngoặc của dòng không khí
\begin{equation*}
	\Delta P_{ng} = 0.55 \times \rho \times \dfrac{\omega_{2}^{2}}{2} = 0.55 \times 1.1391 \times \dfrac{3.95^2}{2} = 4.89(Pa)
\end{equation*}

Trở lực của bề mặt xối tưới, với $\omega$\ $\rho$ = 5.19(Kg/$m^2$.s) > 4.5(Kg/$m^2$.s)
\begin{equation*}
	\Delta P_{x} = 5.85 \times (\omega \rho)^{2.1}  \times g_{L} \times  (\dfrac{H}{d_{dt}})^{0.66} = 5.85 \times (5.19)^{2.1} \times 0.008113 \times (46.73)^{0.66} = 47.23(Pa)
\end{equation*}

Trở lực tại tiết diện có vòi phun nước, ta có : $\xi_{p}$ = 0.65
\begin{equation*}
	\Delta P_{p} = \xi_{p} \times \rho 
	\dfrac{\omega_{2}^2}{2} = 0.65 \times 1.1397 \times \dfrac{3.95^2}{2} = 5.78(Pa)
\end{equation*}

Trở lực tại cửa chóp giữ nước, ta có $\xi_{c}$ = 0.7
\begin{equation*}
	\Delta P_{c} = \xi_{c} \times \rho \times  
	\dfrac{\omega_{2}^2}{2} = 0.7 \times 1.1391 \times \dfrac{3.95^2}{2} = 6.22(Pa)
\end{equation*}

\newpage

Trở lực trong đoạn tháp hình côn, với hệ số trở lực hình côn:
\begin{equation*}
	\begin{split}
		\xi_{k}& = 0.5 \times (1 - \dfrac{f_{4}}{f_{2}})\\
			& = 0.5 \times (1 - \dfrac{1.97}{5.47}) = 0.32\\
		\Delta_{k}& = \xi_{k} \times \rho \times \dfrac{\omega_{2}^2}{2}	\\
				  & = 0.96 \times 1.1391 \times \dfrac{3.95^2}{2} = 2.84(Pa).		 		
	\end{split}
\end{equation*}

Trở lực tại cửa ra của tháp:
\begin{equation*}
	\Delta P_{a} = \rho \dfrac{\omega_{4}^2}{2} = 1.1391 \times \dfrac{10.97^2}{2} = 68.58(Pa).
\end{equation*}

$\Rightarrow$ $\Delta P_{\sum} = 8.52 + 4.89 + 47.23 + 5.78 + 6.22 + 2.84 + 68.58$ = 144.06(Pa).

Công suất quạt gió ( công suất động cơ điện kéo quạt):
\begin{equation*}
	N = \dfrac{1.2 \times G_{kk} \times (\Delta P_{\sum}) 10^{-3}}{\rho \times \eta } (KW).
\end{equation*}

\begin{itemize}
	\item $G_{kk}$ : Lượng không khí thổi qua tháp, Kg/s.
	\item $\Delta P_{\sum}$ : Tổng trợ lực khí động của tháp, Pa.
	\item $\rho$ : Khối lượng riêng của không khí, Kg/$m^3$. 
	\item $\eta$ : Hiệu suất cúa quạt gió, chọn $\eta$ = 0.65
\end{itemize}

$\Rightarrow$ N = $\dfrac{1.2 \times 25.85 \times (144.06)10^{-3}}{1.1391 \times 0.65}$ = 6.035(KW)

\section{CHỌN THÁP GIẢI NHIỆT}
\begin{itemize}
	\item V = 25.85(l/s) = 1550.85(l/phút)
	\item $t_{w1}$ = 40$^{\circ}$C
	\item $t_{w2}$ = 35$^{\circ}$C
	\item $T_{ngoài}$ = 36$^{\circ}$C; RH = 74\%
\end{itemize}

\newpage
\begin{figure}[H]
	\centering
	\includegraphics[width=1.1\textwidth]{chonthapgiainhiet}
	\caption{\textbf{CHỌN THÁP GIẢI NHIỆT}}
	\label{chonthapgiainhiet}
\end{figure}

	\chapmoi{THIẾT KẾ HỆ THỐNG ĐƯỜNG ỐNG}
\section{ỐNG NƯỚC}
\subsection{PHƯƠNG PHÁP TÍNH}
Để tính toán đường ống phân phối nước lạnh đến các tầng và đến các dàn lạnh ta làm theo các phương sau:

\textbf{a. Bước 1}

Tính lưu lượng nước lạnh qua từng đoạn ống:
\begin{equation*}
	G = \dfrac{Q_{0}}{C \times \Delta_{t}}. )Kg/s.
\end{equation*}

Trong đó:
\begin{itemize}
	\item $Q_{0}$ : Là năng suất lạnh yêu cầu của không gian cần điều hòa.
	\item $\Delta_{t}$: Là độ chênh lệnh nhiệt độ của nước vào và ra khỏi dàn lạnh.
	\item C : Nhiệt dung riêng của nước ở nhiệt độ trung bình, KJ/kg.
\end{itemize}

\textbf{b.Bước 2} : Chọn vận tốc sơ bộ, $\omega$ = 0.5 $\approx$ 2 (m/s).

\textbf{c.Bước 3} : Xác định kích thước đường kính trong của đoạn ống:
\begin{equation*}
	d_{tr} = \sqrt{\dfrac{4 \times G}{\pi \times \omega \times \rho }}
\end{equation*}

Trong đó:
\begin{itemize}
	\item G : Lưu lượng nước trong đường ống, kg/s.
	\item $\omega$ : Vận tốc nước chuyển động trong ống, m/s.
	\item $\rho$ : Khối lượng riêng của nước ở nhiệt độ trung bình.
\end{itemize}

\textbf{d.Bước 4} : Chọn đường kính danh nghĩa và thông số $d_{tr}, d_{ng}$ tương ứng.

\textbf{e.Bước 5} : Tính lại vận tốc thức trong ống theo đường kính trong vừa chọn:
\begin{equation*}
	\omega = \dfrac{4 \times G}{\rho \times \pi \times d_{tr}^{2}}
\end{equation*} 

\subsection{TÍNH TOÁN ỐNG CHÍNH CẤP NƯỚC CHO CÁC TẦNG}

\textbf{Tính toán ống chính cấp cho tầng 2-3 (tầng điển hình)}.
\begin{equation*}
	G = \dfrac{Q_{0}}{C \times \Delta_{t}} =\dfrac{144.28}{4.186 \times 5} = 6.89(kg/s).
\end{equation*}

\newpage
Trong đó:
\begin{itemize}
	\item $Q_{0}$ : Năng suất lạnh yêu cầu của không gian cần điều hòa.
	\item $\Delta_{t}$ = 5$^{\circ}$C : Độ chênh nhiệt độ của nước vào và ra khỏi dàn lạnh.
	\item C = 4.186 (kJ/kg.$^{\circ}$K) : Nhiệt dung riêng của nước,
\end{itemize}

Chọn vận tốc sơ bộ : $\omega$ = 1.5(m/s)

Xác định kích thước đường kính của đoạn ống:
\begin{equation*}
	d_{tr} = \sqrt{\dfrac{4 \times 6.89}{3.14 \times 1.5 \times 999.71}} = 0.0765(m)
\end{equation*}

Chọn đường kính danh nghĩa $d_{N}$ = 80(mm)
\begin{itemize}
	\item $d_{tr}$ = 90.1(mm).
	\item $d_{ng}$ = 101.6(mm)
\end{itemize}

Tính lại vận tốc thực theo đường kính trong:
\begin{equation*}
	\omega = \dfrac{4 \times 144.28 }{999.71 \times 3.14 \times 0.0901^{2}} = 1.08(m/s)
\end{equation*}

\begin{figure}[H]
 	\centering
 	\includegraphics[width=1\textwidth]{kichthuocongchinhcap}
 	\caption{\textbf{Kích thước ống chính cấp cho các tầng}}
	\label{kichthuocongchinhcap}	 
\end{figure}

\subsection{TÍNH TOÁN ỐNG NƯỚC KẾT NỐI VỚI CÁC FCU }
Tính lưu lượng nước lạnh qua từng đoạn ống:
\begin{equation*}
	G = \dfrac{Q_{0}}{C \times \Delta_{t}} =\dfrac{17.71}{4.186 \times 5} = 0.85(kg/s).
\end{equation*}

Trong đó:
\begin{itemize}
	\item $Q_{0}$ : Là năng suất lạnh yêu cầu của không gian cần điều hòa.
	\item $\Delta_{t}$: Là độ chênh lệnh nhiệt độ của nước vào và ra khỏi dàn lạnh.
	\item C : Nhiệt dung riêng của nước ở nhiệt độ trung bình, KJ/kg.
\end{itemize}

Chọn vận tốc sơ bộ : $\omega$ = 1.5(m/s)

Xác định kích thước đường kính của đoạn ống:
\begin{equation*}
	d_{tr} = \sqrt{\dfrac{4 \times 0.8462}{3.14 \times 1.5 \times 999.71}} = 0.0268(m)
\end{equation*}

Chọn đường kính danh nghĩa $d_{N}$ = 32(mm)
\begin{itemize}
	\item $d_{tr}$ = 35.1(mm).
	\item $d_{ng}$ = 42.1(mm)
\end{itemize}

Tính lại vận tốc thực theo đường kính trong:
\begin{equation*}
	\omega = \dfrac{4 \times 0.85 }{999.71 \times 3.14 \times 0.0351^{2}} = 0.88(m/s)
\end{equation*}

\begin{figure}[H]
	\centering
	\includegraphics[width=1\textwidth]{kichthuocongketnoitb}
	\caption{\textbf{Kích thước ống kết nối với các thiết bị}}
	\label{kichthuocongketnoitb}	 
\end{figure}

\subsection{TÍNH TOÁN ĐƯỜNG ỐNG CHO CÁC TẦNG}
** \textbf{TẦNG 1}

\textbf{Tính toán cho ống chính}

Tính lưu lượng nước lạnh qua từng đoạn ống:
\begin{equation*}
	G = \dfrac{Q_{0}}{C \times \Delta_{t}} =\dfrac{221.55}{4.186 \times 5} = 10.59(kg/s).
\end{equation*}

Trong đó:
\begin{itemize}
	\item $Q_{0}$ : Là năng suất lạnh yêu cầu của không gian cần điều hòa.
	\item $\Delta_{t}$: Là độ chênh lệnh nhiệt độ của nước vào và ra khỏi dàn lạnh.
	\item C : Nhiệt dung riêng của nước ở nhiệt độ trung bình, KJ/kg.
\end{itemize}

Chọn vận tốc sơ bộ : $\omega$ = 1.5(m/s)

Xác định kích thước đường kính của đoạn ống:
\begin{equation*}
	d_{tr} = \sqrt{\dfrac{4 \times 10.59}{3.14 \times 1.5 \times 999.71}} = 0.0948(m)
\end{equation*}

Chọn đường kính danh nghĩa $d_{N}$ = 125(mm)
\begin{itemize}
	\item $d_{tr}$ = 97.2(mm).
	\item $d_{ng}$ = 114.3(mm)
\end{itemize}

Tính lại vận tốc thực theo đường kính trong:
\begin{equation*}
	\omega = \dfrac{4 \times 0.85 }{999.71 \times 3.14 \times 0.0972^{2}} = 1.43(m/s)
\end{equation*}

-- \textbf{NHÁNH 1}

Tính lưu lượng nước lạnh qua từng đoạn ống:
\begin{equation*}
	G = \dfrac{Q_{0}}{C \times \Delta_{t}} =\dfrac{19.88}{4.186 \times 5} = 0.95(kg/s).
\end{equation*}

Trong đó:
\begin{itemize}
	\item $Q_{0}$ : Là năng suất lạnh yêu cầu của không gian cần điều hòa.
	\item $\Delta_{t}$: Là độ chênh lệnh nhiệt độ của nước vào và ra khỏi dàn lạnh.
	\item C : Nhiệt dung riêng của nước ở nhiệt độ trung bình, KJ/kg.
\end{itemize}

Chọn vận tốc sơ bộ : $\omega$ = 1.5(m/s)

Xác định kích thước đường kính của đoạn ống:
\begin{equation*}
	d_{tr} = \sqrt{\dfrac{4 \times 0.95}{3.14 \times 1.5 \times 999.71}} = 0.0284(m)
\end{equation*}

Chọn đường kính danh nghĩa $d_{N}$ = 32(mm)
\begin{itemize}
	\item $d_{tr}$ = 35.1(mm).
	\item $d_{ng}$ = 42.1(mm)
\end{itemize}

Tính lại vận tốc thực theo đường kính trong:
\begin{equation*}
	\omega = \dfrac{4 \times 0.95 }{999.71 \times 3.14 \times 0.0351^{2}} = 0.98(m/s)
\end{equation*}

\newpage

--\textbf{NHÁNH 2}

Tính lưu lượng nước lạnh qua từng đoạn ống:
\begin{equation*}
	G = \dfrac{Q_{0}}{C \times \Delta_{t}} =\dfrac{29.82}{4.186 \times 5} = 1.42(kg/s).
\end{equation*}

Trong đó:
\begin{itemize}
	\item $Q_{0}$ : Là năng suất lạnh yêu cầu của không gian cần điều hòa.
	\item $\Delta_{t}$: Là độ chênh lệnh nhiệt độ của nước vào và ra khỏi dàn lạnh.
	\item C : Nhiệt dung riêng của nước ở nhiệt độ trung bình, KJ/kg.
\end{itemize}

Chọn vận tốc sơ bộ : $\omega$ = 1.5(m/s)

Xác định kích thước đường kính của đoạn ống:
\begin{equation*}
	d_{tr} = \sqrt{\dfrac{4 \times 1.42}{3.14 \times 1.5 \times 999.71}} = 0.0348(m)
\end{equation*}

Chọn đường kính danh nghĩa $d_{N}$ = 40(mm)
\begin{itemize}
	\item $d_{tr}$ = 40.9(mm).
	\item $d_{ng}$ = 48.2(mm)
\end{itemize}

Tính lại vận tốc thực theo đường kính trong:
\begin{equation*}
	\omega = \dfrac{4 \times 1.42 }{999.71 \times 3.14 \times 0.0409^{2}} = 1.09(m/s)
\end{equation*}

--\textbf{NHÁNH 3}

Tính lưu lượng nước lạnh qua từng đoạn ống:
\begin{equation*}
	G = \dfrac{Q_{0}}{C \times \Delta_{t}} =\dfrac{24.78}{4.186 \times 5} = 1.18(kg/s).
\end{equation*}

Trong đó:
\begin{itemize}
	\item $Q_{0}$ : Là năng suất lạnh yêu cầu của không gian cần điều hòa.
	\item $\Delta_{t}$: Là độ chênh lệnh nhiệt độ của nước vào và ra khỏi dàn lạnh.
	\item C : Nhiệt dung riêng của nước ở nhiệt độ trung bình, KJ/kg.
\end{itemize}

Chọn vận tốc sơ bộ : $\omega$ = 1.5(m/s)

Xác định kích thước đường kính của đoạn ống:
\begin{equation*}
	d_{tr} = \sqrt{\dfrac{4 \times 1.18}{3.14 \times 1.5 \times 999.71}} = 0.0317(m)
\end{equation*}

Chọn đường kính danh nghĩa $d_{N}$ = 32(mm)
\begin{itemize}
	\item $d_{tr}$ = 935.1(mm).
	\item $d_{ng}$ = 42.1(mm)
\end{itemize}

Tính lại vận tốc thực theo đường kính trong:
\begin{equation*}
	\omega = \dfrac{4 \times 0.1.18 }{999.71 \times 3.14 \times 0.0351^{2}} = 1.22(m/s)
\end{equation*}

**\textbf{TẦNG M}

\textbf{Tính toán cho ống chính}

Tính lưu lượng nước lạnh qua từng đoạn ống:
\begin{equation*}
	G = \dfrac{Q_{0}}{C \times \Delta_{t}} =\dfrac{124.57}{4.186 \times 5} = 5.95(kg/s).
\end{equation*}

Trong đó:
\begin{itemize}
	\item $Q_{0}$ : Là năng suất lạnh yêu cầu của không gian cần điều hòa.
	\item $\Delta_{t}$: Là độ chênh lệnh nhiệt độ của nước vào và ra khỏi dàn lạnh.
	\item C : Nhiệt dung riêng của nước ở nhiệt độ trung bình, KJ/kg.
\end{itemize}

Chọn vận tốc sơ bộ : $\omega$ = 1.5(m/s)

Xác định kích thước đường kính của đoạn ống:
\begin{equation*}
	d_{tr} = \sqrt{\dfrac{4 \times 5.95}{3.14 \times 1.5 \times 999.71}} = 0.0711(m)
\end{equation*}

Chọn đường kính danh nghĩa $d_{N}$ = 70(mm)
\begin{itemize}
	\item $d_{tr}$ = 73.7(mm).
	\item $d_{ng}$ = 88.9(mm)
\end{itemize}

Tính lại vận tốc thực theo đường kính trong:
\begin{equation*}
	\omega = \dfrac{4 \times 5.95 }{999.71 \times 3.14 \times 0.0737^{2}} = 1.40(m/s)
\end{equation*}

-- \textbf{NHÁNH 1}

Tính lưu lượng nước lạnh qua từng đoạn ống:
\begin{equation*}
	G = \dfrac{Q_{0}}{C \times \Delta_{t}} =\dfrac{35.42}{4.186 \times 5} = 1.69(kg/s).
\end{equation*}

Trong đó:
\begin{itemize}
	\item $Q_{0}$ : Là năng suất lạnh yêu cầu của không gian cần điều hòa.
	\item $\Delta_{t}$: Là độ chênh lệnh nhiệt độ của nước vào và ra khỏi dàn lạnh.
	\item C : Nhiệt dung riêng của nước ở nhiệt độ trung bình, KJ/kg.
\end{itemize}

Chọn vận tốc sơ bộ : $\omega$ = 1.5(m/s)

Xác định kích thước đường kính của đoạn ống:
\begin{equation*}
	d_{tr} = \sqrt{\dfrac{4 \times 1.69}{3.14 \times 1.5 \times 999.71}} = 0.0379(m)
\end{equation*}

Chọn đường kính danh nghĩa $d_{N}$ = 40(mm)
\begin{itemize}
	\item $d_{tr}$ = 40.9(mm).
	\item $d_{ng}$ = 42.1(mm)
\end{itemize}

Tính lại vận tốc thực theo đường kính trong:
\begin{equation*}
	\omega = \dfrac{4 \times 1.69 }{999.71 \times 3.14 \times 0.0409^{2}} = 1.29(m/s)
\end{equation*}

-- \textbf{NHÁNH 2}

Tính lưu lượng nước lạnh qua từng đoạn ống:
\begin{equation*}
	G = \dfrac{Q_{0}}{C \times \Delta_{t}} =\dfrac{19.88}{4.186 \times 5} = 0.95(kg/s).
\end{equation*}

Trong đó:
\begin{itemize}
	\item $Q_{0}$ : Là năng suất lạnh yêu cầu của không gian cần điều hòa.
	\item $\Delta_{t}$: Là độ chênh lệnh nhiệt độ của nước vào và ra khỏi dàn lạnh.
	\item C : Nhiệt dung riêng của nước ở nhiệt độ trung bình, KJ/kg.
\end{itemize}

Chọn vận tốc sơ bộ : $\omega$ = 1.5(m/s)

Xác định kích thước đường kính của đoạn ống:
\begin{equation*}
	d_{tr} = \sqrt{\dfrac{4 \times 0.95}{3.14 \times 1.5 \times 999.71}} = 0.0284(m)
\end{equation*}

Chọn đường kính danh nghĩa $d_{N}$ = 32(mm)
\begin{itemize}
	\item $d_{tr}$ = 35.1(mm).
	\item $d_{ng}$ = 42.1(mm)
\end{itemize}

Tính lại vận tốc thực theo đường kính trong:
\begin{equation*}
	\omega = \dfrac{4 \times 0.95 }{999.71 \times 3.14 \times 0.0351^{2}} = 0.98(m/s)
\end{equation*}

**\textbf{ (TẦNG 2-3)}.
\begin{equation*}
	G = \dfrac{Q_{0}}{C \times \Delta_{t}} =\dfrac{144.28}{4.186 \times 5} = 6.89(kg/s).
\end{equation*}

Trong đó:
\begin{itemize}
	\item $Q_{0}$ : Năng suất lạnh yêu cầu của không gian cần điều hòa.
	\item $\Delta_{t}$ = 5$^{\circ}$C : Độ chênh nhiệt độ của nước vào và ra khỏi dàn lạnh.
	\item C = 4.186 (kJ/kg.$^{\circ}$K) : Nhiệt dung riêng của nước,
\end{itemize}

Chọn vận tốc sơ bộ : $\omega$ = 1.5(m/s)

Xác định kích thước đường kính của đoạn ống:
\begin{equation*}
	d_{tr} = \sqrt{\dfrac{4 \times 6.89}{3.14 \times 1.5 \times 999.71}} = 0.0765(m)
\end{equation*}

Chọn đường kính danh nghĩa $d_{N}$ = 80(mm)
\begin{itemize}
	\item $d_{tr}$ = 90.1(mm).
	\item $d_{ng}$ = 101.6(mm)
\end{itemize}

Tính lại vận tốc thực theo đường kính trong:
\begin{equation*}
	\omega = \dfrac{4 \times 144.28 }{999.71 \times 3.14 \times 0.0901^{2}} = 1.08(m/s)
\end{equation*}

-- \textbf{NHÁNH 1-2}

Tính lưu lượng nước lạnh qua từng đoạn ống:
\begin{equation*}
	G = \dfrac{Q_{0}}{C \times \Delta_{t}} =\dfrac{28.2}{4.186 \times 5} = 1.35(kg/s).
\end{equation*}

Trong đó:
\begin{itemize}
	\item $Q_{0}$ : Là năng suất lạnh yêu cầu của không gian cần điều hòa.
	\item $\Delta_{t}$: Là độ chênh lệnh nhiệt độ của nước vào và ra khỏi dàn lạnh.
	\item C : Nhiệt dung riêng của nước ở nhiệt độ trung bình, KJ/kg.
\end{itemize}

Chọn vận tốc sơ bộ : $\omega$ = 1.5(m/s)

Xác định kích thước đường kính của đoạn ống:
\begin{equation*}
	d_{tr} = \sqrt{\dfrac{4 \times 1.35}{3.14 \times 1.5 \times 999.71}} = 0.0338(m)
\end{equation*}

Chọn đường kính danh nghĩa $d_{N}$ = 40(mm)
\begin{itemize}
	\item $d_{tr}$ = 40.9(mm).
	\item $d_{ng}$ = 48.2(mm)
\end{itemize}

Tính lại vận tốc thực theo đường kính trong:
\begin{equation*}
	\omega = \dfrac{4 \times 1.35 }{999.71 \times 3.14 \times 0.0409^{2}} = 1.03(m/s)
\end{equation*}

\subsection{TÍNH TOÁN ỐNG NƯỚC TRONG PHÒNG CHILLER}

**\textbf{ỐNG KẾT NỐI VỚI CHILLER}

Tính lưu lượng nước lạnh qua từng đoạn ống:
\begin{equation*}
	G = \dfrac{Q_{0}}{C \times \Delta_{t}} =\dfrac{539.44}{4.186 \times 5} = 25.77(kg/s).
\end{equation*}

Trong đó:
\begin{itemize}
	\item $Q_{0}$ : Là năng suất lạnh yêu cầu của không gian cần điều hòa.
	\item $\Delta_{t}$: Là độ chênh lệnh nhiệt độ của nước vào và ra khỏi dàn lạnh.
	\item C : Nhiệt dung riêng của nước ở nhiệt độ trung bình, KJ/kg.
\end{itemize}

Chọn vận tốc sơ bộ : $\omega$ = 1.5(m/s)

Xác định kích thước đường kính của đoạn ống:
\begin{equation*}
	d_{tr} = \sqrt{\dfrac{4 \times 25.77}{3.14 \times 1.5 \times 999.71}} = 0.1480(m)
\end{equation*}

Chọn đường kính danh nghĩa $d_{N}$ = 150(mm)
\begin{itemize}
	\item $d_{tr}$ = 154.1(mm).
	\item $d_{ng}$ = 168.3(mm)
\end{itemize}

Tính lại vận tốc thực theo đường kính trong:
\begin{equation*}
	\omega = \dfrac{4 \times 25.77 }{999.71 \times 3.14 \times 0.1541^{2}} = 1.38(m/s)
\end{equation*}

**\textbf{ỐNG GÓP}

Tính lưu lượng nước lạnh qua từng đoạn ống:
\begin{equation*}
	G = \dfrac{Q_{0}}{C \times \Delta_{t}} =\dfrac{25.85}{4.186 \times 5} = 1.23(kg/s).
\end{equation*}

Trong đó:
\begin{itemize}
	\item $Q_{0}$ : Là năng suất lạnh yêu cầu của không gian cần điều hòa.
	\item $\Delta_{t}$: Là độ chênh lệnh nhiệt độ của nước vào và ra khỏi dàn lạnh.
	\item C : Nhiệt dung riêng của nước ở nhiệt độ trung bình, KJ/kg.
\end{itemize}

Chọn vận tốc sơ bộ : $\omega$ = 1.5(m/s)

Xác định kích thước đường kính của đoạn ống:
\begin{equation*}
	d_{tr} = \sqrt{\dfrac{4 \times 1.23}{3.14 \times 1.5 \times 999.71}} =0.0324 (m)
\end{equation*}

Chọn đường kính danh nghĩa $d_{N}$ = 40(mm)
\begin{itemize}
	\item $d_{tr}$ = 40.9(mm).
	\item $d_{ng}$ = 48.2(mm)
\end{itemize}

Tính lại vận tốc thực theo đường kính trong:
\begin{equation*}
	\omega = \dfrac{4 \times 1.23 }{999.71 \times 3.14 \times 0.0409^{2}} = 0.94(m/s)
\end{equation*}

\subsection{TÍNH TOÁN ĐƯỜNG ỐNG CHO THÁP GIẢI NHIỆT}

**\textbf{ỐNG KẾT NỐI TỚI THÁP GIẢI NHIỆT}

Lưu lượng nước lạnh: G = 24.62(kg/s)

Chọn vận tốc sơ bộ : $\omega$ = 1.5(m/s)

Xác định kích thước đường kính của đoạn ống:
\begin{equation*}
	d_{tr} = \sqrt{\dfrac{4 \times 24.62}{3.14 \times 1.5 \times 999.71}} =0.14461 (m)
\end{equation*}

Chọn đường kính danh nghĩa $d_{N}$ = 150(mm)
\begin{itemize}
	\item $d_{tr}$ = 154.1(mm).
	\item $d_{ng}$ = 168.3(mm)
\end{itemize}

Tính lại vận tốc thực theo đường kính trong:
\begin{equation*}
	\omega = \dfrac{4 \times 24.62 }{999.71 \times 3.14 \times 0.1541^{2}} = 1.32(m/s)
\end{equation*}

**\textbf{ỐNG GÓP}

Lưu lượng nước lạnh: G = 73.85(kg/s)

Chọn vận tốc sơ bộ : $\omega$ = 1.5(m/s)

Xác định kích thước đường kính của đoạn ống:
\begin{equation*}
	d_{tr} = \sqrt{\dfrac{4 \times 73.85}{3.14 \times 1.5 \times 999.71}} =0.2505 (m)
\end{equation*}

Chọn đường kính danh nghĩa $d_{N}$ = 250(mm)
\begin{itemize}
	\item $d_{tr}$ = 254.5(mm).
	\item $d_{ng}$ = 273(mm)
\end{itemize}

Tính lại vận tốc thực theo đường kính trong:
\begin{equation*}
	\omega = \dfrac{4 \times 73.85 }{999.71 \times 3.14 \times 0.2545^{2}} = 1.45(m/s)
\end{equation*}
	\chapmoi{THIẾT KẾ HỆ THỐNG CẤP THOÁT NƯỚC}
	\chapmoi{HỆ THỐNG CUNG CẤP ĐIỆN}
\section{TỔNG QUAN}
\textbf{Nhiệm vụ thết kế cung cấp điện:}

Mục tiêu chính của thiết kế cung cấp điện là đảm bảo cho tòa nhà văn phòng tiêu thụ luôn luôn đủ điện năng với chất lượng nằm trong phạm vi cho phép. Phải thỏa mãn những yêu cầu sau:
\begin{itemize}
	\item Vốn đầu tư nhỏ, chú ý đến tiết kiệm được ngoại tệ quý và vật tư hiếm.
	\item Đảm bảo độ tin cậy cung cấp điện cao tùy theo tính chất tiêu thụ.
	\item Chi phí vận hành hàng năm thấp.
	\item Đảm bảo an toàn cho người và thiết bị.
	\item Đảm bảo tính kinh tế.
	\item Thuận tiện cho vận hành, sửa chữa v.v…
\end{itemize}

Ngoài ra, khi thiết kế cung cấp điện phải chú ý đến những yêu cầu khác như:
\begin{itemize}
	\item Dự báo được khả năng phát triển phụ tải sau này.
	\item Rút ngắn thời gian xây xựng.
\end{itemize}

Ngày nay, điện năng được sử dụng rất rộng rãi trong các ngành như: điện tử, giao thông vận tải.v.v.. Do đó mà vai trò của điện đối với đời sống xã hội, điện năng được xem là chỉ tiêu, là thước đo về sự phát triển của một quốc gia.

\section{TÍNH TOÁN PHỤ TẢI ĐIỆN CHO TÒA NHÀ}
\subsection{TÍNH TOÁN PHỤ TẢI CHIẾU SÁNG}
\subsubsection{Đèn hiệu:}
Yêu cầu: Lựa chọn, thiết kế và cung cấp các vật phản quang, các phụ kiện và thiết bị kiểm soát theo sự giới thiệu của nhà sản xuất và cho phép các loại đèn đạt chất lượng thực hiện theo tài liệu kỹ thuật được xuất bản của nhà sản xuất.
\subsubsection{Chiếu sáng nhân tạo:}
Hệ thống chiếu sáng trong nhà được tính toán đủ ánh sáng khi không có chiếu sáng tự nhiên mà vẫn đảm bảo mọi hoạt động bình thường của con người trong công trình. Trong trường hợp khẩn cấp xảy ra vẫn có hệ thống chiếu sáng sự cố được bố trí dọc đường thoát nạn các loại đèn này có nguồn acquy riêng duy trì tối thiểu 2h.

Chiếu sáng ngoài nhà: Hệ thống chiếu sáng ngoài toà công trình được thiết kế với độ sáng 100 Lux. Toàn bộ dùng loại đèn pha có chao chụp phản quan bóng Sodiumnua 250$ W $ loại sử dụng tranpormer
\begin{itemize}
	\item Văn phòng độ rọi từ 300-400 Lux.
	\item Sảnh độ rọi từ 150-200 Lux.
	\item Khu hành lang, cầu thang, kho, vệ sinh độ rọi tối thiểu 100 Lux và được trang bị đèn khẩn cấp để thoát hiểm.
\end{itemize}

Điện áp sử dụng cho đèn là 220v nếu sử dụng đèn có điện áp 380v thì phải có dây nối đất. Với các loại đèn chiếu sáng sự cố, cục bộ, cầm tay ở trong các phòng nguy hiểm hoặc rất nguy hiểm thì điện áp nhỏ hơn 42v và nhỏ hơn 12v đối với các phòng ẩm ướt, chật chội dễ bị chạm vào những bề mặt kim loại lớn có nối đất.
\subsubsection{Thiết kế chiếu sáng:}
\textbf{CÁC CÔNG THỨC SẼ SỬ DỤNG TÍNH TOÁN CHIẾU SÁNG}
\begin{itemize}[label = $\blacktriangleright$]
	\item Chiều cao treo đèn: $H_{tt} = H - H' - H_{lv}$
	\item Phân bố bóng đèn: $K = \dfrac{a\times b}{H_{tt}\times (a+b)}$
	\item Tỷ số treo: $j =\dfrac{H'}{H' + H_{tt}}$
	\item Hệ số chiếu sáng: $U = u_{d}\times \eta_{d} + u_{i}\times \eta_{i}$
	\item Quang thông tổng cộng: {\Large $\Phi$}$_{\Sigma} = \dfrac{E_{tc}\times S\times d}{U}$
	\item Xác định số lượng bóng đèn: $N_{bd} = ${\Large $\dfrac{\Phi_{\Sigma}}{\Phi_{bd}}$}
	\item Kiểm tra độ rọi trung bình trên bề mặt làm việc: $E_{tb} = \dfrac{N_{bd}\times \Phi_{2}\times U}{S\times d}$
\end{itemize}






	\chapmoi{HỆ THỐNG TỰ ĐỘNG HOÁ}
	\chapmoi{TÍNH TOÁN KINH TẾ}

\section{DỰ TOÁN}


\section{SO SÁNH CHI PHÍ}


	\printbibliography[title={TÀI LIỆU THAM KHẢO}] %Tạo các trang nguồn tài liệu tham khảo
\end{document}