\fancyhead[L]{LỜI NÓI ĐẦU}
\begin{center}
	\textbf{{\Large \textcolor{red}{LỜI NÓI ĐẦU}}}
\end{center}

Nước ta hiện đang là một trong những quốc gia có mức độ công nghiệp hoá và hiện đại hoá đang phát triển rất mạnh mẽ trong những năm gần đây. Khi mà đời sống tăng cao thì nhu cầu về điều hoà càng cao, có thể nói hầu như trong tất cả các cao ốc, văn phòng, khách sạn, bệnh viện, nhà hàng, một số phân xưởng,… đã và đang xây dựng đều trang bị hệ thống điều hòa không khí. Mục đích của việc điều hòa không khí là tạo ra môi trường vi khí hậu thích hợp cho điều kiện sinh lý của con người và nâng cao độ tin cậy hoạt động của các trang thiết bị công nghệ.

Tuy nhiên, nếu chỉ đơn thuần là việc điều hoà không khí và vận hành thôi thì trong thời đại kỹ thuật số hiện nay là chưa đủ. Nhu cầu cấp tiến hơn cho các hệ thống lớn là ngoài việc vận hành phải đảm bảo được an toàn, ổn định thì còn phải đạt được tiêu chí tiết kiệm điện, giảm giá thành vận hành hệ thống. Hệ thống càng lớn thì mức độ tiết kiệm càng được để ý nhiều hơn. 

Với đề tài ``IoT HVAC''. Để thực nhiện đề tài này, chúng em đã vận dụng kiến thức, kinh nghiệp làm việc và các tài liệu liên quan để tính toán, thiết kế dưới sự hướng dẫn tận tình của thầy Đỗ Trí Nhựt.

Vì đây là lần đầu tiên thực hiện việc tính toán, thiết kế cho một công trình lớn, hơn nữa kiến thức chuyên môn còn hạn chế và chưa có kinh nghiệm thực tế nên sẽ có nhiều thiếu sót. Kính mong được sự chỉ bảo cũng như góp ý quý báo của quý thầy cô để chúng em có được kinh nghiệm và tiến bộ sau này. Chúng em xin chân thành cảm ơn!

\begin{flushright}
	Hồ Chí Minh, tháng 02 năm 2020
	
	Sinh viên thực hiện
	
	Nguyễn Phúc Thịnh
	
	Nguyễn Thành Nhân
	
	Nguyễn Vũ Trường
\end{flushright}
